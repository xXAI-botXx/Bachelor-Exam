% -------------------------------------------------------
% Daten für die Arbeit
% Wenn hier alles korrekt eingetragen wurde, wird das Titelblatt
% automatisch generiert. D.h. die Datei titelblatt.tex muss nicht mehr
% angepasst werden.

\newcommand{\hsmasprache}{en} % de oder en für Deutsch oder Englisch
% Für korrekt sortierte Literatureinträge, noch preambel.tex anpassen
% und zwar bei \usepackage[main=ngerman, english]{babel},
% \usepackage[pagebackref=false,german]{hyperref}
% und \usepackage[autostyle=true,german=quotes]{csquotes}

% Titel der Arbeit auf Deutsch
\newcommand{\hsmatitelde}{Untersuchung des Einflusses von Material- und 3d-Modellkombinationen auf die Instanzsegmentierung}

% Titel der Arbeit auf Englisch
% \newcommand{\hsmatitelen}{Investigation of the Impact of Material and 3D Model Combinations on Instance Segmentation}
% \newcommand{\hsmatitelen}{Assessing Bias Toward Material or Shape in Instance Segmentation}
% \newcommand{\hsmatitelen}{Investigation of Material and Shape Quantities on Instance Segmentation Accuracy}
%\newcommand{\hsmatitelen}{Investigation of Material and Shape Quantities and Their Bias Impact on Sim-to-Real Instance Segmentation Accuracy}
\newcommand{\hsmatitelen}{Depth Data and Shape-Texture Biases in Instance Segmentation}

% \newcommand{\hsmatitelen}{Using 3D-Information Improves Shape Awareness}



% Weitere Informationen zur Arbeit
\newcommand{\hsmaort}{Offenburg}    % Ort
\newcommand{\hsmaautorvname}{Tobia} % Vorname(n)
\newcommand{\hsmaautornname}{Ippolito} % Nachname(n)
\newcommand{\hsmadatum}{06 December} % Datum der Abgabe
\newcommand{\hsmajahr}{2024} % Jahr der Abgabe
\newcommand{\hsmafirma}{Optonic GmbH} % Firma bei der die Arbeit durchgeführt wurde
\newcommand{\hsmabetreuer}{Prof. Dr.-Ing. Janis Keuper, Offenburg University of Applied Sciences} % Betreuer an der Hochschule
\newcommand{\hsmazweitkorrektor}{Morith Sperling, Optonic GmbH} % Betreuer im Unternehmen oder Zweitkorrektor
\newcommand{\hsmafakultaet}{EMI} % Fakultät
\newcommand{\hsmastudiengang}{AKI} % Studiengangsabkürzung. 
% Diese wird in titelblatt.tex definiert. Bisher AI, EI, MK und INFM. Bitte ergänzen.

% Zustimmung zur Veröffentlichung
\setboolean{hsmapublizieren}{true}   % Einer Veröffentlichung wird zugestimmt
\setboolean{hsmasperrvermerk}{false} % Die Arbeit hat keinen Sperrvermerk

% -------------------------------------------------------
% Abstract

% Kurze (maximal halbseitige) Beschreibung, worum es in der Arbeit geht auf Deutsch
\newcommand{\hsmaabstractde}{Lorem ipsum dolor sit amet, consetetur sadipscing elitr, sed diam nonumy eirmod tempor invidunt ut labore et dolore magna aliquyam erat, sed diam voluptua. At vero eos et accusam et justo duo dolores et ea rebum. Stet clita kasd gubergren, no sea takimata sanctus est Lorem ipsum dolor sit amet. Lorem ipsum dolor sit amet, consetetur sadipscing elitr, sed diam nonumy eirmod tempor invidunt ut labore et dolore magna aliquyam erat, sed diam voluptua. At vero eos et accusam et justo duo dolores et ea rebum. Stet clita kasd gubergren, no sea takimata sanctus est Lorem ipsum dolor sit amet.}

% Kurze (maximal halbseitige) Beschreibung, worum es in der Arbeit geht auf Englisch

\newcommand{\hsmaabstracten}{Englische Version von Lorem ipsum dolor sit amet, consetetur sadipscing elitr, sed diam nonumy eirmod tempor invidunt ut labore et dolore magna aliquyam erat, sed diam voluptua. At vero eos et accusam et justo duo dolores et ea rebum. Stet clita kasd gubergren, no sea takimata sanctus est Lorem ipsum dolor sit amet. Lorem ipsum dolor sit amet, consetetur sadipscing elitr, sed diam nonumy eirmod tempor invidunt ut labore et dolore magna aliquyam erat, sed diam voluptua. At vero eos et accusam et justo duo dolores et ea rebum. Stet clita kasd gubergren, no sea takimata sanctus est Lorem ipsum dolor sit amet.}
