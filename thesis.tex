% Preambel mit Einstellungen importieren
\input{preambel}

% Dokumenteninfos importieren
% In docinfo.tex sind Titel, Autor, Abstract zu definieren
% -------------------------------------------------------
% Daten für die Arbeit
% Wenn hier alles korrekt eingetragen wurde, wird das Titelblatt
% automatisch generiert. D.h. die Datei titelblatt.tex muss nicht mehr
% angepasst werden.

\newcommand{\hsmasprache}{en} % de oder en für Deutsch oder Englisch
% Für korrekt sortierte Literatureinträge, noch preambel.tex anpassen
% und zwar bei \usepackage[main=ngerman, english]{babel},
% \usepackage[pagebackref=false,german]{hyperref}
% und \usepackage[autostyle=true,german=quotes]{csquotes}

% Titel der Arbeit auf Deutsch
% \newcommand{\hsmatitelde}{Bias, Generalisierung und Sim-zu-Real Leistung bei der Instanzsegmentierung: Eine Praktische Untersuchung von Menge und Tiefe}
\newcommand{\hsmatitelde}{Untersuchung des Einflusses von Tiefeninformationen und der Anzahl an Formen und Texturen auf den Bias und die Performance in der Instanz Segmentierung}

% Titel der Arbeit auf Englisch
% \newcommand{\hsmatitelen}{Investigation of the Impact of Material and 3D Model Combinations on Instance Segmentation}
% \newcommand{\hsmatitelen}{Assessing Bias Toward Material or Shape in Instance Segmentation}
% \newcommand{\hsmatitelen}{Investigation of Material and Shape Quantities on Instance Segmentation Accuracy}
%\newcommand{\hsmatitelen}{Investigation of Material and Shape Quantities and Their Bias Impact on Sim-to-Real Instance Segmentation Accuracy}
% \newcommand{\hsmatitelen}{Depth Data and Shape-Texture Biases in Instance Segmentation: Including the Quantity of Varying Shapes and Textures}
% \newcommand{\hsmatitelen}{Generalization and Bias of Shape-Texture and Sim-to-Real Performance in Instance Segmentation: An Analysis of Quantity and Depth}
% Shape-Texture Generalization, Bias and Sim-to-Real in Instance Segmentation: An Analysis of Quantity and Depth
% \newcommand{\hsmatitelen}{Bias, Generalization and Sim-to-Real Performance in Instance Segmentation:\\A Practical Investigation of Quantity and Depth}
% \newcommand{\hsmatitelen}{An Investigation of Depth and Shape-Texture Quantity in Instance Segmentation}
% \newcommand{\hsmatitelen}{Investigating the Influence of Depth Information on the Shape-Texture Bias in Instance Segmentation}
\newcommand{\hsmatitelencleaned}{Investigating the Influence of \\Depth Information and the Amount of \\Shapes and Textures on Bias and Performance in Instance Segmentation}
\newcommand{\hsmatitelen}{Investigating the Influence of Depth Information and the Amount of Shapes and Textures on Bias and Performance in Instance Segmentation}
% Investigating the Influence of Depth and Shape-Texture Quantity on Bias and Performance


% \newcommand{\hsmatitelen}{Using 3D-Information Improves Shape Awareness}



% Weitere Informationen zur Arbeit
\newcommand{\hsmaort}{Offenburg}    % Ort
\newcommand{\hsmaautorvname}{Tobia} % Vorname(n)
\newcommand{\hsmaautornname}{Ippolito} % Nachname(n)
\newcommand{\hsmadatum}{03 December 2024} % Datum der Abgabe
\newcommand{\hsmajahr}{2024} % Jahr der Abgabe
\newcommand{\hsmafirma}{Optonic GmbH} % Firma bei der die Arbeit durchgeführt wurde
\newcommand{\hsmabetreuer}{Prof. Dr.-Ing. Janis Keuper, Offenburg University of Applied Sciences} % Betreuer an der Hochschule
\newcommand{\hsmazweitkorrektor}{Moritz Sperling, Optonic GmbH} % Betreuer im Unternehmen oder Zweitkorrektor
\newcommand{\hsmafakultaet}{EMI} % Fakultät
\newcommand{\hsmastudiengang}{AKI} % Studiengangsabkürzung. 
% Diese wird in titelblatt.tex definiert. Bisher AI, EI, MK und INFM. Bitte ergänzen.

% Zustimmung zur Veröffentlichung
\setboolean{hsmapublizieren}{true}   % Einer Veröffentlichung wird zugestimmt
\setboolean{hsmasperrvermerk}{false} % Die Arbeit hat keinen Sperrvermerk

% -------------------------------------------------------
% Abstract

% Kurze (maximal halbseitige) Beschreibung, worum es in der Arbeit geht auf Deutsch
%\newcommand{\hsmaabstractde}{Lorem ipsum dolor sit amet, consetetur sadipscing elitr, sed diam nonumy eirmod tempor invidunt ut labore et dolore magna aliquyam erat, sed diam voluptua. At vero eos et accusam et justo duo dolores et ea rebum. Stet clita kasd gubergren, no sea takimata sanctus est Lorem ipsum dolor sit amet. Lorem ipsum dolor sit amet, consetetur sadipscing elitr, sed diam nonumy eirmod tempor invidunt ut labore et dolore magna aliquyam erat, sed diam voluptua. At vero eos et accusam et justo duo dolores et ea rebum. Stet clita kasd gubergren, no sea takimata sanctus est Lorem ipsum dolor sit amet.}

% Kurze (maximal halbseitige) Beschreibung, worum es in der Arbeit geht auf Englisch

%\newcommand{\hsmaabstracten}{coming soon...}



% Literatur-Datenbank
\addbibresource{literatur.bib}   % BibLaTeX-Datei mit Literaturquellen einbinden

\begin{document}
\frontmatter

% Römische Ziffern für die "Front-Matter"
\setcounter{page}{0}
\changefont{ptm}{m}{n}  % Times New Roman für den Fließtext
\renewcommand{\rmdefault}{ptm}

% Titelblatt
% -------------------------------------------------------
% In dieser Datei sollten eigentlich keine Veränderungen mehr
% notwendig sein.
% -------------------------------------------------------

\thispagestyle{empty}

% Fakultät
% -------------------------------------------------------
\ifthenelse{\equal{\hsmafakultaet}{EI}}%
  {\newcommand{\hsmafakultaetlangde}{Fakultät Elektrotechnik und Informationstechnik}%
   \newcommand{\hsmafakultaetlangen}{Department of Electrical Engineering and Computer Science}}{}
\ifthenelse{\equal{\hsmafakultaet}{EMI}}%
{\newcommand{\hsmafakultaetlangde}{Fakultät Elektrotechnik, Medizintechnik und Informatik}%
	\newcommand{\hsmafakultaetlangen}{Department of Electrical Engineering, Medical Engineering and Computer Science}}{}



\ifthenelse{\equal{\hsmastudiengang}{AI}}%
{\newcommand{\hsmastudienganglangde}{Angewandte Informatik}%
	\newcommand{\hsmastudienganglangen}{Applied Computer Science}%
	\newcommand{\hsmatypde}{BACHELORARBEIT}%
	\newcommand{\hsmatypen}{BACHELOR THESIS}%
	\newcommand{\hsmagrad}{\hsmabachelor}}{}
	
\ifthenelse{\equal{\hsmastudiengang}{AKI}}%
{\newcommand{\hsmastudienganglangde}{Angewandte Künstliche Intelligenz}%
	\newcommand{\hsmastudienganglangen}{Applied Artificial Intelligent}%
	\newcommand{\hsmatypde}{BACHELORARBEIT}%
	\newcommand{\hsmatypen}{BACHELOR THESIS}%
	\newcommand{\hsmagrad}{\hsmabachelor}}{}
	
\ifthenelse{\equal{\hsmastudiengang}{EI}}%
{\newcommand{\hsmastudienganglangde}{Elektrotechnik/Informationstechnik}%
	\newcommand{\hsmastudienganglangen}{Electrical Engineering/Information Technology}%
	\newcommand{\hsmatypde}{BACHELORARBEIT}%
	\newcommand{\hsmatypen}{BACHELOR THESIS}%
	\newcommand{\hsmagrad}{\hsmabachelor}}{}

\ifthenelse{\equal{\hsmastudiengang}{MK}}%
{\newcommand{\hsmastudienganglangde}{Mechatronik}%
	\newcommand{\hsmastudienganglangen}{Mechatronics}%
	\newcommand{\hsmatypde}{BACHELORARBEIT}%
	\newcommand{\hsmatypen}{BACHELOR THESIS}%
	\newcommand{\hsmagrad}{\hsmabachelor}}{}

\ifthenelse{\equal{\hsmastudiengang}{INFM}}%
  {\newcommand{\hsmastudienganglangde}{Informatik Master}%
  \newcommand{\hsmastudienganglangen}{Computer Science Master}%
  \newcommand{\hsmatypde}{MASTERARBEIT}%
  \newcommand{\hsmatypen}{MASTER THESIS}%
  \newcommand{\hsmagrad}{\hsmamaster}}{}

\newcommand{\hsmamaster}{Master of Science (M.Sc.)}

\newcommand{\hsmabachelor}{Bachelor of Science (B.Sc.)}


\newcommand{\hsmakoerperschaftde}{Hochschule für Technik, Wirtschaft und Medien Offenburg}
\newcommand{\hsmakoerperschaften}{Offenburg University}

\newcommand{\hsmaautorbib}{\hsmaautornname, \hsmaautorvname} % Autor Nachname, Vorname
\newcommand{\hsmaautor}{\hsmaautorvname \ \hsmaautornname} % Autor Vorname Nachname

\ifthenelse{\equal{\hsmasprache}{de}}%
  {\newcommand{\hsmatyp}{\hsmatypde}%
   \newcommand{\hsmathesistype}{zur Erlangung des akademischen Grades \hsmagrad}%
   \newcommand{\hsmakoerperschaft}{\hsmakoerperschaftde}%
   \newcommand{\hsmastudiengangname}{Studiengang \hsmastudienganglangde}%
   \newcommand{\hsmastudienganglang}{\hsmastudienganglangde}%
   \newcommand{\hsmatitel}{\hsmatitelde}%
   \newcommand{\hsmatutor}{Betreuer}%
   \newcommand{\hsmafakultaetlang}{\hsmafakultaetlangde}%
   \newcommand{\hsmalistoftables}{Tabellenverzeichnis}%
   \newcommand{\hsmalistoffigures}{Abbildungsverzeichnis}%
   \newcommand{\hsmalistings}{Quellcodeverzeichnis}%
   \newcommand{\hsmaindex}{Index}%
   \newcommand{\hsmaabbreviations}{Abkürzungsverzeichnis}%   
   \selectlanguage{ngerman}}%
  {\newcommand{\hsmatyp}{\hsmatypen}%
   \newcommand{\hsmathesistype}{for the acquisition of the academic degree \hsmagrad}%
   \newcommand{\hsmakoerperschaft}{\hsmakoerperschaften}%
   \newcommand{\hsmastudiengangname}{Course of Studies: \hsmastudienganglang}%
   \newcommand{\hsmastudienganglang}{\hsmastudienganglangen}%
   \newcommand{\hsmatitel}{\hsmatitelencleaned}%
   \newcommand{\hsmatutor}{Tutors}
   \newcommand{\hsmafakultaetlang}{\hsmafakultaetlangen}%
   \newcommand{\hsmalistoftables}{List of Tables}%
   \newcommand{\hsmalistoffigures}{List of Figures}%
   \newcommand{\hsmalistings}{Listings}%
   \newcommand{\hsmaindex}{Index}%
   \newcommand{\hsmaabbreviations}{List of Abbreviations}%
   \selectlanguage{english}}%


% Daten in die Standard-Felder von KOMA-Script eintragen
\titlehead{\hsmatyp\ in\  \hsmastudienganglang}
\subject{}
\title{\hsmatitel}
\author{\hsmaauthor}
\date{\small{\hsmadatum}}

% Daten für das fertige PDF-Dokument
\hypersetup{
  pdftitle={\hsmatitel},  % Titel des Dokuments
  pdfauthor={\hsmaautor},              % Autor
  pdfsubject={\hsmatyp\ in\ \hsmastudienganglang},                % Thema
  pdfkeywords={\hsmatitel}         % Schlüsselworte
}

\newlength{\bindekorrektur}
\newlength{\seitenanfang}
\newlength{\seitenbreite}
  
\setlength{\bindekorrektur}{-46mm}   % Korrektur der horizontalen Position
\setlength{\seitenanfang}{0mm}       % Korrektur der vertikalen Position
\setlength{\seitenbreite}{297mm}

%\noindent \includegraphics[width=7cm, left]{hso.png}\hfill \includegraphics[width=2cm, right]{edeka.png} \\
\captionsetup[figure]{labelformat=empty}
\noindent 
\begin{figure}
	%\includegraphics[width=10cm,center]{hso.jpg}
% Wenn ein Unternehmenslogo mit abgedruckt werden soll,
% kann dies wie folgt integriert werden.	
	\begin{subfigure}[b]{0.5\textwidth}
		\includegraphics[width=7cm,left]{hso.jpg}
	\end{subfigure} 
	\begin{subfigure}[b]{0.5\textwidth}
		\centering
		\includegraphics[width=4cm,right]{optonic.png}
	\end{subfigure} 
	\caption[]{}
\end{figure}
\captionsetup[figure]{labelformat=simple}
% Titel der Arbeit
\begin{textblock*}{128mm}(41mm,\seitenanfang + 62mm) % 4,5cm vom linken Rand und 6,0cm vom oberen Rand
  \centering\Large\sffamily
  \vspace{12mm} % Kleiner zusätzlicher Abstand oben für bessere Optik
  \textbf{\hsmatitel}
\end{textblock*}%

% Name
\begin{textblock*}{\seitenbreite}(\bindekorrektur,\seitenanfang + 108mm)
  \centering\large\sffamily
  \hsmaautor
\end{textblock*}

% Thesis
\begin{textblock*}{\seitenbreite}(\bindekorrektur,\seitenanfang + 130mm)
  \centering\large\sffamily
  \textbf{\hsmatyp}\\
  \begin{small}\hsmathesistype \end{small}\\
  \vspace{6mm}
  \hsmastudiengangname
\end{textblock*}

% Fakultät
\begin{textblock*}{\seitenbreite}(\bindekorrektur,\seitenanfang + 165mm)
  \centering\large\sffamily
  \hsmafakultaetlang\\
  \vspace{2mm}
  \hsmakoerperschaft
\end{textblock*}

% Datum
\begin{textblock*}{\seitenbreite}(\bindekorrektur,\seitenanfang + 190mm)
  \centering\large 
  \textsf{\hsmadatum}
\end{textblock*}

% Firma
\begin{textblock*}{\seitenbreite}(\bindekorrektur,\seitenanfang + 215mm)
  \centering\large 
  % \textsf{Durchgeführt bei der Firma \hsmafirma}
  \textsf{Performed at the company \hsmafirma}
\end{textblock*}

% Betreuer
\begin{textblock*}{\seitenbreite}(\bindekorrektur,\seitenanfang + 240mm)
  \centering\large\sffamily
  \hsmatutor \\
  \vspace{2mm}
  \hsmabetreuer\\
  \vspace{2mm}
  \hsmazweitkorrektor
\end{textblock*}

% Bibliographische Informationen
\null\newpage
\thispagestyle{empty}
  
\newcommand{\hsmabibde}{\begin{small}\textbf{\hsmaautorbib}: \\ \hsmatitelde \ / \hsmaautor. \ -- \\ \hsmatypde, \hsmaort : \hsmakoerperschaftde, \hsmajahr. \pageref{lastpage} Seiten.\end{small}}

\newcommand{\hsmabiben}{\begin{small}\textbf{\hsmaautorbib}: \\ \hsmatitelen \ / \hsmaautor. \ -- \\ \hsmatypen, \hsmaort : \hsmakoerperschaften, \hsmajahr. \pageref{lastpage} pages. \end{small}}

\ifthenelse{\equal{\hsmasprache}{de}}%
  {\hsmabibde \\ \vspace{0.5cm} \\ \hsmabiben}
  {\hsmabiben \\ \vspace{0.5cm} \\ \hsmabibde}


%Vorwort
\clearpage\setcounter{page}{1}
\thispagestyle{empty}
\textsf{\large\textbf{Acknowledgment}}

This work would not have been possible without the guidance of Optonic GmbH, especially Moritz Sperling and my university supervisor, Prof. Dr.-Ing. Janis Keuper.\\
The journey was challenging; I faced many difficulties. Including technological complications and personal health issues. \\
Although I often tried to solve problems independently, seeking help sooner eased some challenges. In the end, I could not be happier to have accepted these challenges and navigated through them with the expertise and dedication of my supervisors.


% Erklärung
\clearpage
\thispagestyle{empty}
\textsf{\large\textbf{Eidesstattliche Erklärung}}

Hiermit versichere ich eidesstattlich, dass die vorliegende Master-Thesis von mir selbststän-dig und ohne unerlaubte fremde Hilfe angefertigt worden ist, insbesondere, dass ich alle Stel-len, die wörtlich oder annähernd wörtlich oder dem Gedanken nach aus Veröffentlichungen, unveröffentlichten Unterlagen und Gesprächen entnommen worden sind, als solche an den entsprechenden Stellen innerhalb der Arbeit durch Zitate kenntlich gemacht habe, wobei in den Zitaten jeweils der Umfang der entnommenen Originalzitate kenntlich gemacht wurde. Ich bin mir bewusst, dass eine falsche Versicherung rechtliche Folgen haben wird.

Ich bin damit einverstanden, dass meine Arbeit veröffentlicht wird, d.\,h. dass die Arbeit elektronisch gespeichert, in andere Formate konvertiert, auf den Servern der Hochschule Offenburg öffentlich zugänglich gemacht und über das Internet verbreitet werden darf.

%\textsf{\large\textbf{Declaration on oath}}

%I hereby declare on oath that this Bachelor's thesis has been prepared by me independently and without unauthorized external assistance, in particular, that I have identified all passages taken verbatim or approximately verbatim or in spirit from publications, unpublished documents, and conversations as such at the appropriate places within the thesis utilizing quotations, whereby the scope of the original quotations taken has been indicated in the quotations. I am aware that a false statement will have legal consequences.

\vspace{1cm}
\hsmaort, \hsmadatum \\
\hsmaautor

\vspace{2.5cm}

\textsf{\large\textbf{AI-Tool Disclaimer}}

%Nowadays AI-Tools are everywhere and it is important to state what is human-made and what not. In reality, the work of AI and humans is woven together, and therefore, it is even more crucial to name and try to differentiate.\\
% This work was written by a human for humans. 
% The AI-Tool Grammarly \cite{Grammarly} was applied to correct grammar and spelling, for appealing language and best output for the readers. The text itself was not the creation of an AI nor AI assisted.\\
% ChatGPT \cite{ChatGPT} was used for supporting during coding to create and debug tasks efficiently. It still remains the handcraft and thinking of the author.

%I hereby affirm that the AI tool Grammarly \cite{Grammarly} was used solely to correct grammar, spelling, and to enhance the language for optimal clarity and appeal to the reader. The text itself, however, was not generated by AI, nor was it AI-assisted in its creation. Furthermore, I testify that ChatGPT \cite{ChatGPT} was employed as a supportive tool during the coding process, assisting with the creation and debugging of tasks in an efficient manner. Nevertheless, the work reflects the independent craftsmanship and reasoning of the author.
Ich versichere hiermit, dass das KI-Tool Grammarly \cite{Grammarly} ausschließlich dazu verwendet wurde, Grammatik und Rechtschreibung zu korrigieren und die Sprache zu verbessern, um eine optimale Klarheit und Attraktivität für den Leser zu erreichen. Der Text selbst wurde jedoch weder von der KI erstellt. Darüber hinaus bezeuge ich, dass ChatGPT \cite{ChatGPT} während des Programmierens als unterstützendes Werkzeug eingesetzt wurde, das die Erstellung und das Debugging von Aufgaben auf effiziente Weise unterstützte. Nichtsdestotrotz spiegelt die Arbeit das unabhängige Handwerk und die Überlegungen des Autors wider.

\ifthenelse{\boolean{hsmapublizieren} \and \not\boolean{hsmasperrvermerk}}%
{
\vspace{0.5cm}
%I agree that my work may be published, i.e. that the work may be stored electronically, converted into other formats, made publicly accessible on the servers of Offenburg University of Applied Sciences and distributed via the Internet.  
}{}%


\vspace{1cm}
\hsmaort, \hsmadatum \\
%\vspace{1.2cm}						                                      
\hsmaautor

\ifthenelse{\boolean{hsmasperrvermerk}}%
{%
\vspace{5cm}
\color{red}\textsf{\large\textbf{Sperrvermerk}}

Die vorliegende Abschlussarbeit beinhaltet vertrauliche Informationen und interne Daten des Unternehmens \hsmafirma.
Sie darf aus diesem Grund nur zu Prüfzwecken verwendet und ohne ausdrückliche Genehmigung durch die \hsmafirma weder Dritten zugänglich gemacht, noch ganz oder in Auszügen veröffentlicht werden. Die Sperrfrist endet 5 Jahre Jahre nach dem Einreichen der Arbeit bei der Hochschule Offenburg. Unbeschadet hiervon bleibt die Weitergabe der Arbeit und Einsicht in die Arbeit an die mit der Prüfung befassten Mitarbeiter der Hochschule und Prüfer möglich, die ihrerseits zur Geheimhaltung verpflichtet sind, sowie die Verwendung der Arbeit in eventuellen prüfungsrechtlichen Rechtsschutzverfahren nach Maßgabe der geltenden verwaltungsprozessualen Regeln.
\color{black}
}{}

\cleardoublepage

% Abstract
\thispagestyle{empty}
\textsf{\large\textbf{Zusammenfassung}}
\subsubsection*{\hsmatitelde}%\hsmaabstractde
In dieser Studie werden neun synthetische RGBD-Instanzsegmentierungsdatensätze für das Training und zusätzliche Datensätze für das Testen vorgestellt, wobei ein besonderer Schwerpunkt auf der Variation der Anzahl von Formen und Texturen liegt. Zusätzlich wird ein realer Bin-Picking-Datensatz mit unübersichtlichen und übersichtlichen Industrieteilen und Alltagsprodukten vorgestellt. Diese Datensätze und der bekannte OCID-Datensatz werden verwendet, um die Auswirkung von Tiefeninformationen und der Anzahl der Formen und Texturen auf den "Shape-Texture-Bias", die Leistung und die Generalisierung bei der Instanzsegmentierung zu untersuchen.\\
Die Ergebnisse zeigen, dass reine RGB-Modelle einen Bias in Richtung der Textur besitzen, während RGB-D-Modelle eine leichte Verschiebung in Richtung Shape-Bias aufweisen, obwohl der Textur Bias bestehen bleibt. Modelle, die RGB und Tiefe verwenden, schnitten etwas besser ab als mit nur RGB Daten, und ebenso auf einem der beiden realen Datensatz war das Segmentierungsergebnis deutlich besser mit der Tiefe als zusätzlichen Input. Allerdings könnte eine niedrige Qualität der Tiefeninformationen, eventuell die Schärfe, zu verzerrten Ergebnissen führen, die noch schlechter sind als bei reinen RGB-Modellen. Darüber hinaus zeigt diese Studie, dass reine RGB-Modelle massiv von der Qualität (Auflösung) der Eingabebilder beeinflusst werden.\\
Die Erhöhung der Anzahl der einzigartigen Formen in den Trainingsdaten verringert den Shape-Bias, erhöht die allgemeine Instanzsegmentierungsleistung, steigend mit der Anzahl der unbekannten Formen in den Daten und hat einen entscheidenden Einfluss auf die Generalisierungs von Formen. \\
Die Erhöhung der Anzahl der Texturen hatte keine signifikante Auswirkung auf die Shape-Texture-Bias, aber sie zeigt, dass sie auch die allgemeine Leistung mit der steigenden Anzahl unbekannter Texturen erhöht. Diese Studie legt nahe, dass eine höhere Anzahl an Texturen in den Trainingsdaten zu einer höheren Generalisierung von Texturen führt.

\clearpage
\thispagestyle{empty}
\textsf{\large\textbf{Abstract}}
\subsubsection*{\hsmatitelen}%\hsmaabstracten

This study presents nine synthetic RGBD instance segmentation datasets for training and additional datasets for testing with a special focus on varying the number of shapes and textures. Additionally, a real-world bin-picking dataset with cluttered and uncluttered industrial parts and everyday products is proposed. These datasets and the known OCID dataset are used to investigate the impact of depth information and shape-texture amount towards shape-texture bias, performance, and generalization in instance segmentation.\\
The findings reveal that RGB-only models are biased toward texture, while RGB-D models exhibit a slight shift towards shape bias, though texture bias persists. Models using RGB and depth performed slightly better, and one real-world dataset was significantly better. However, the depth information quality, eventually sharpness, could lead to distorted results, even worse than RGB-only. In addition, this study finds that RGB-only models are influenced massively by the quality (resolution) of the input images.\\
Increasing the number of unique shapes in train data decreases the shape bias, increases the general performance, rises with the number of unknown shapes in the data, and has a crucial impact on the generalization of shapes. \\
Increasing the number of textures had no significant effect on the shape-texture bias, but it shows that it also increases the general performance with the rising number of unknown textures. This study suggests that a higher texture amount in train data leads to higher generalization of textures.






% Inhaltsverzeichnis erzeugen
\cleardoublepage
\pdfbookmark{\contentsname}{Contents}
\tableofcontents

% Korrigiert Nummerierung bei mehrseitigem Inhaltsverzeichnis
\cleardoublepage
\newcounter{frontmatterpage}
\setcounter{frontmatterpage}{\value{page}}

% Arabische Zahlen für den Hauptteil
\mainmatter

% Den Hauptteil mit vergrößertem Zeilenabstand setzen
\onehalfspacing

% ------------------------------------------------------------------
% Hauptteil der Arbeit
% \listoftodos

% Instance Segmentation: A brief look at the current state
% The Need for Accurate Instance Segmentation
% INTRODUCTION
\chapter{Introduction}
\label{chap:kapitel1}



	\section{Objective and Importance}
	\label{sec:objective-and-importance}
		Since introducing instance segmentation in the years 2012-2014 \cite{Yang2012}\cite{Silbermann2012}\cite{Hariharan2014}, it has become one of the most important and complex tasks in computer vision \cite{Sharma2022}. \\
		Detecting and separating every object in an image can be found in many modern responsibilities. It ranges from grading prostate cancer \cite{Hassan2022}, to understanding cell division, cellular growth, and morphogenesis \cite{Kar2022}, to wildlife monitoring \cite{Haucke2021}, to segmenting unknown marine objects \cite{Hu2024}, to tooth segmentation in dental medicine \cite{Brahmi2023} and much more. The rising use cases also create a need for more precise segmentations. It can be challenging to achieve this goal because many factors, like the domain, data quality, quantity, \ac{dnn} architecture, and data augmentations, can be adjusted.\\
		For example, it is still being determined if the input images should be provided as RGB with depth or as only depth or RGB. First, it seems straightforward that depth is important and helpful information for segmentation, and it showed promising results in many cases \cite{Danielczuk2019}\cite{Xie2021}. However, recent research shows that RGB-only images as input, for instance segmentation, can achieve even better results \cite{Raj2023}. It doesn't seem very clear and leads to the question of which factors matter for precise generalization and whether depth data could improve this success even more.\\
		In addition, there are common difficulties in the field of instance segmentation, such as clutter environments with overlapping objects (also called occlusion), novel objects, and challenging materials like translucent or reflective ones.\\
		There also needs to be more clarity about the shape-texture bias in \ac{cnn}-based approaches. Some research suggests a bias towards texture as rewarding \cite{Qiu2024} while others recommend a shape-bias \cite{Geihors2019} and also a debiased approach can be successful in some scenarios \cite{Li2021}\cite{Co2021}\cite{Chung2023}. In turn, research shows that a bias alone cannot be the reason for a good or bad performance \cite{Gavrikov2024}. It is essential to notice that some factors of these researches vary, like sometimes the focus is on classification, sometimes on \ac{ood} data, but the current state of research still seems to be confusing and have a gap of research and clearness.\\
		Besides generalization optimization and the influence of bias, there is often the transfer from simulation to the real world. The need for simulations exists due to the data quantity that every \ac{dnn} requires to perform precisely and accurately \cite {Uchida2016}\cite{Alzubaidi2021}\cite{Csurka2023}. Labeling segmentation data is always very time-consuming and costly since every pixel must be labeled. To solve this challenge, using a virtual environment with automatic labeling is much cheaper and faster. However, there is a gap between the synthetic and real-world data. How to bridge from simulation to the real world still remains a problem of research \cite{Doersch2019}.\\
		Lastly, there is not enough qualitative research about the influence of the quantity of different shapes and textures on instance segmentation.\\
		\\
		\textbf{To conclude}, more research needs to be done on shape-texture bias in instance segmentation, and the existing research needs to be more precise and also in context with generalization. Consequently, it is crucial to continue research on generalization, bias, sim-to-real, depth-data, and shape-texture quantity to improve the performance of instance segmentation.
		
	
	
	\section{Core Focus of the Study}    % Research Gap and 
	\label{sec:core-focus-of-the-study}
		Generalization, shape-texture bias, and sim-to-real transfer are all fundamental and complex areas with many research possibilities. This study will focus on depth data and the quantity of shapes and textures in relation to these three areas. A practical approach to uncovering new and valuable insights is the main objective of this study.\\
		Understanding the impact of depth data on shape-texture bias, generalization, and sim-to-real performance of a \ac{dnn} is essential. Collecting depth data in real-world scenarios can be challenging, so knowing how much depth data enhances segmentation outcomes can be highly beneficial.\\
		Additionally, new information about the influence of shape-texture quantity could be relevant and valuable. Generating different shapes and textures with high quality and quantity, potentially in a domain-specific manner, is demanding and time-intensive. Determining the minimum necessary variety of shapes and textures for robust generalization can support the efficiency of \ac{dnn} development for instance-segmentation.\\
		\\
		An uncovered inspection is the combination of depth data and shape-texture quantity on generalization, shape-texture bias, and sim-to-real transfer, which could yield interesting insights.
		
	
	
	\section{Methodology Overview}
	\label{sec:methodology-overview}
		This work trained 18 different Mask R-CNN \cite{Kaiming2017} models on synthetic, in unreal engine \cite{Romero2022} created datasets, each with 20.000 RGB-, depth- and mask-images. The different shapes and textures vary between the datasets. Also, the models can only have RGB or RGB-depth as input data.\\
		With these 18 \ac{dnn}s, three different studies were performed for generalization, shape-texture bias, and sim-to-real transfer.\\
		A few outlier data shows, if the \ac{dnn} learned to prefer shape or texture.\\
		A small Test dataset with a structured combination of novel and known shapes and textures is proposed to test the generalization of shape and texture.\\
		Finally, an existing real-world dataset, for instance segmentation, is used to see the sim-to-real performance.
		
	
	
	\section{Scope and Delimitations}
	\label{sec:scope-and-delimitations}
		Investigating generalization, shape-texture bias, and sim-to-real transfer with a focus on depth data and shape-texture quantity must come with reduced scope and delimitations. Otherwise, it would be an excess of work. Furthermore, some delimitation is needed, thus keeping the influencing factors as small as possible.\\
		First, only Mask R-CNN \cite{Kaiming2017} is used as \ac{dnn} because many assumptions are made on top of research for \ac{cnn}-based approaches.\\
		Moreover, all used \ac{dnn}s are trained on the same type of datasets, where only the objects' shape, texture, and position vary.\\
		Both named delimitations are important in excluding influences with certainty. Another \ac{dnn} architecture can differ in bias, accuracy, and learning. Mask R-CNN's wide range of use, distribution, and flexibility made it a good choice as \ac{dnn} for this study.\\
		For the same reason, every \ac{dnn} used the same hyper-parameters for training. Even if there are better parameters, these parameters must be consistently equal during the studies to eliminate a possible influence.\\
		This work focuses primarily on instance segmentation for bin-picking. Which is a common task of instance segmentation and important for automation\cite{Raj2023}\cite{Danielczuk2019}\cite{Xie2021}. However, the research should be valid and vital for every instance of the segmentation approach.\\
		Another delimitation is the choice of different shapes and textures. The difference is difficult to measure. There are some methods, like the structural similarity index measure \cite{Wang2004}, but it is not a standard approach. In the end, a subjective choice was applied due to some misclassifications using objective methods.
		\\
		Due to the amount of studies and time limitations, a limited test-data size is given.\\
		This study can find indications and chances, while an underlying truth cannot be provided. Again, this is owed to the complexity and dimension of this research.\\
		Also, due to time limitations, a suboptimal number of epochs was chosen. More epochs would lead to even better results, but the experiments should not be influenced by the extra precision; still, it is a limitation.
		Next up, the number of different shapes and textures used is limited to a maximum of 160. The rare availability of high-quality shapes and textures justifies this.\\
		The usage of depth information also comes with delimitations. There are many ways to use depth information, as described in section \ref{sec:state-of-the-art}, but this study can only approach one way to use depth, or else the amount of \ac{dnn} and training would rise too vast. This study uses the straightforward approach of adding a fourth channel with depth information.\\
		Synthetic data often comes with quality issues, and this work's proposed datasets are not different. There are some major downsides which should be considered. The render quality should be higher than it is because of the method used to render and save. The spawning objects can also spawn into each other, sometimes leading to unfavorable data. \\
		Lastly, this study needs to meet the assumption that it will not find novel knowledge. This study tries to reinforce or question current state-of-the-art research to find practical results. In a broader context, that could provide valuable insights. A profound look is not possible through the wide research question.
	
	
	
	\section{Key Definitions}
	\label{sec:key-definitions}
	
		Here are some important definitions for this work to make a clear point and prevent misunderstandings.\\
		\\
		\textbf{Instance Segmentation:} Instance Segmentation is the task of finding and labeling every foreground object in an image pixel-wise. The classification of these objects is optional and plays no role in the segmentation.\\
		\textbf{Bin Picking:} During bin picking, a robot tries to grasp an object from a bin and uses or transports it to another place. Instance Segmentation is needed in bin picking to find all available objects in the bin.\\
		\textbf{Texture:} Texture is an object's visual appearance and pattern. This word is significant. Thus, it is part of the investigation. An object can have the texture of a rock, water, salt, skin, piece of wood, steel beam, and much more. A texture is local information. It describes specific areas on a detailed level and does not describe the overall shape of an object.\\
		\textbf{Material:} A material describes the physical properties of an object, like color, metalness, roughness, and the height of the surface. If a picture is taken of an object that has a material, on the picture, the texture of the object is visible, which is defined by its material. The properties of a material can be described as single number values or, for more complexity, as a bump map (image).\\
		\textbf{Shape:} A Shape defines the geometry of an object. It is also often referred to as mesh or 3D-model. Simple shapes are spheres and rectangles. A shape is global information. It defines the boundaries of an object and is only available in the broader view.\\
		\textbf{Depth data:} As depth mentioned, data are images that contain information about the distance from every object. The images have only one channel. Every pixel value states how far or near the given pixel in reality to the camera was/is.\\
		\textbf{Novel Objects:} Novel objects are objects that were not part of the \ac{dnn}'s training. Novel objects are a typical challenge for bin-picking since there are often unique and innovative products in production and automation pipelines.\\
		\textbf{Out of distribution data:} Data which differs essentially from the train data of a \ac{dnn}. A new domain with a completely different background, resolution, quality, and quantity is a \ac{ood} data.\\
		\textbf{Generalization:}The \ac{dnn}'s ability to generalize defines how well it learns the underlying task and patterns of the train data. Shape- or texture-generalization, in instance segmentation specifically, is the ability to learn to segment objects with their shape or texture, even if they are novel to the \ac{dnn}. Meaning that the network learned how to extract shape or texture and detect object boundaries with this information. \\
		\textbf{Bias:} A bias shows a preferred, learned strategy to solve a task. It is most likely caused by the train data itself but also influenced by the \ac{dnn} architecture. A shape- or texture bias is present when the \ac{dnn} prefers the global shape information or the local texture information to make mask predictions.\\
		\textbf{Sim-to-real:} Sim-to-real is the transfer from synthetic train data to the real world. Synthetic data is used through its cheap and fast production and the challenging collection of labeled real data. The quality of synthetic data fluctuates but is always much less complex than the natural world and does not reproduce reality precisely enough, thus causing a gap. The goal is that the \ac{dnn} generalize well enough to bridge this sim-to-real gap.\\
		\textbf{Occlusions:} If only a part of an object is visible thus another object lays on top, it called occlusion. Occlusion is a common problem in clutter scenes, appearing often in instance segmentation.
	
	
	
	\section{Structure}
	\label{sec:structure}
		In the \hyperref[chap:kapitel2]{next chapter}, a brief look at the current state of research and related work will be done; Which leads to the actual thesis.\\
		\hyperref[chap:kapitel3]{It follows} a description of used tools and environments. Furthermore, the Implementation of the used \ac{dnn} and data will be explained.\\
		\hyperref[chap:kapitel4]{Proceeding to} the concept of experiments and measurement of the results.\\
		\hyperref[chap:kapitel5]{The 'results' chapter} shows the results of the experiments, followed by \hyperref[chap:kapitel6]{the discussion and thoughts} about these and what maybe limitates the claims.\\
		\hyperref[chap:kapitel7]{Ultimately a summary} of the foundings is outlined.
		





 % Externe Datei einbinden

\chapter{Research Background and the Hypothesis}
\label{chap:kapitel2}

	The research background for this work is enormous through the many aspects of this study. Shape-texture bias, generalization, and sim-to-real are among the major research fields of instance segmentation. This chapter summarizes research on these topics (see section \ref{sec:state-of-the-art} and \ref{sec:related-work}) and states the hypothesis of this study (see section \ref{sec:hypothesis-statement}).

	\section{State of the Art}
	\label{sec:state-of-the-art}
		\paragraph{Methods:} Instance segmentation has become an essential field in computer vision, and much research has been done on different approaches to achieving precise mask predictions for foreground objects. Still, there are a lot of open questions and investigations \cite{Sharma2022}. The cited paper also states that there are four big types of state-of-the-art machine-learning methods for instance segmentation: Transformers, reinforcement learning, proposal-based (\ac{dnn}), and proposal-free (\ac{dnn}) methods.\\
		Reinforcement Learning is tricky to use for instance segmentation, and transformers are too inefficient for this complex task, but achieve part-wise high accuracy evidenced by the just mentioned paper.\\
		The paper from \citeauthor{Sharma2022} also confirms that proposal-free and proposal-based (\ac{dnn}), like the here used Mask R-CNN \cite{Kaiming2017}, are the most common used and practical useful approaches. With Proposal-based approaches as the baseline for instance segmentation.\\
		This work will not go further into the topic of \ac{dnn} architecture, but it is a necessary field with many open questions.
		\clearpage
		\paragraph{Shape Texture Bias:} Most instance segmentation methods tend to have a strong bias towards texture \cite{Theodoridis2022}, which is a common problem of \ac{cnn}-based approaches \cite{Geirhos2022}\cite{Baker2018}\cite{Tabak2023}.\\
		Some research tries to apply shape bias to \ac{cnn}-based approaches in similar tasks and show improved accuracy of the used \ac{dnn}s \cite{Geirhos2022}\cite{Hermann2020}; because, it comes closer to the human vision \cite{Geirhos2020}\cite{Mohla2022}\cite{Baker2020}.\\
		Recent research claims that shape bias leads not in every case to improved accuracy and generalization and that a \ac{dnn} with texture bias can achieve state-of-the-art \ac{ood} accuracy in many different scenarios \cite{Qiu2024}. Previous studies have reported similar findings \cite{Brochu2019}.\\
		Another approach is to de-bias the \ac{dnn}. So it includes shape and texture equally in its decision-making process \cite{Li2021}\cite{Co2021}\cite{Chung2023}. Which claims to achieve substantial improvements on ImageNet. \\
		The reason for the existence of texture bias is the shortcut learning from \ac{cnn}s \cite{Geirhos2020}, probably also influenced by unnatural data augmentation techniques \cite{Hermann2020}. Research found that shape information still plays an essential role in decision-making of many texture-biased \ac{dnn}s \cite{Tabak2023}.\\
		In contrast to past research, recent research states that a bias towards texture or shape can not explain the ability to generalize \cite{Gavrikov2024}. The relevance of bias is in question and is probably not as important as thought for generalization and maybe performance. %; otherwise, there would not be much research on shape texture bias.
		%\paragraph{Generalization:} ...
		\paragraph{Sim-to-real:} Using a synthetic dataset from a virtual environment became the standard procedure to quickly generate a massive amount of labeled data for instance segmentation \cite{Danielczuk2019}\cite{Xie2020}\cite{Xie2021}\cite{Shao2018}\cite{Toda2019}. To overcome the sim-to-real gap, the quality of texture plays a crucial role \cite{Tabak2023}\cite{Martinez2019}; That makes it appropriate to use a simulation software with a powerful and realistic-looking renderer for data generation, such as Unreal Engine 5 \cite{Romero2022}.\\
		Another successful technique to transfer well from sim to real is domain randomization \cite{Raj2023}. A variation to the synthetic data is applied to randomly modify shape, texture, colors, lightning (exposure), camera angles and cropping, blur, and noise. These random modifications bring the data closer to reality, where such modification also happens. Thus, the improvement is explainable. This variation can also be applied in a data-augmentation step after data generation \cite{Kar2022}.\\
		Many open questions remain about the sim-to-real transfer and many approaches lacking in generalization when \ac{dnn}s are only trained on synthetic data \cite{Doersch2019}. Modern and future render technology and highly realistic simulation software could improve sim-to-real transfer in instance segmentation even more.
		\clearpage
		\paragraph{Input Type:} Since the beginning of instance segmentation, depth data has played a vital role \cite{Silbermann2012}. It comes from the fact that this is a task based heavily on 3D information. Objects can be stacked on each other, and there may be only a small piece of an object to see;
		Thus, 3D information is crucial to precisely segment all objects from each other. Assisting an instance segmentation \ac{dnn}, providing information about the depth is therefore an obvious and understandable approach.\\
		How influential the depth information really is for segmentation accuracy and if RGB-only images do provide not enough 3D information by themselves is an open question; some recent researches show a trend to RGB-only approaches \cite{Raj2023}\cite{Zakeri2024}.\\
		There is also a depth-information-only approach due to the better sim-to-real ability of depth data, which could caused by the simpler nature of this data type \cite{Danielczuk2019}.\\\\
		Widespread is the use of depth as the fourth channel (RGB-D), yet the results are disagreed \cite{Lüling2021}\cite{Zakeri2024}.
		But the research gets even further to the question of the best way to implement depth information in \ac{dnn} \cite{Xiang2021}\cite{Xie2020}\cite{Pei2024}\cite{Yasir2022}\cite{Shao2018}\cite{Ye2017}.\\
		Also, the model itself used, for instance segmentation, changes the effectiveness of the used input types \cite{Xiang2021}.\\
		In conclusion, there is no absolute best input type, which always leads to the best segmentation masks. Factors like domain, brightness, \ac{dnn} architecture, and data quality could play a vital role in the choice of the input type, which makes optimizing instance segmentation even harder. The success of RGB-only input type in recent research \cite{Raj2023}\cite{Zakeri2024} can maybe be explained by the increasing quality of synthetic data but still needs more research to be confirmed.
		\paragraph{Shape Texture Amount:} Finally, no research was found that investigates the influence of shape or texture quantity on the accuracy of instance segmentation. Even if this question seems unimportant at first, a closer look is worth it. It appears unambiguous that more shapes and textures lead to better generalization and accuracy under the assumption of enough data amount. However, the quantity of how much it improves the accuracy could be very intriguing. 
		

	\clearpage
	\section{Related Work}
	\label{sec:related-work}
		%Investigating the amount of shape and texture with depth data towards generalization, bias, and sim-to-real seems to be novel and unique. Therefore, no comparable work was found. \\
		%This paper shows, among other things, practical results and shows if research in generalization, bias, and sim-to-real is reproducible in this specific practical use-case.\\
		Related work in this context is \citetitle{Gavrikov2024} \cite{Gavrikov2024}. The work from \citeauthor{Gavrikov2024} analyzes the relation between bias and generalization, as well as this work also (indirectly) investigates in the influence of bias towards generalization. This study will only focus on shape-texture bias, while \citetitle{Gavrikov2024} also analyzes spectral bias and critical bands bias. Moreover, this work examines the task of instance segmentation while the paper from \citeauthor{Gavrikov2024} is focused on classification tasks.\\
		Another related work is \citetitle{Raj2023} \cite{Raj2023}, which also addresses instance segmentation with Mask R-CNN for bin-picking with focus on the input-type as this study also do. The train data is also synthetic and so the sim-to-real challenge also appears. The work from \citeauthor{Raj2023} delimitates itself from this work, by focusing more on domain randomness in the data and also includes a grasp pose planner, which both are not covered in this study.\\
		The paper \citetitle{Tabak2023} from \citeauthor{Tabak2023} \cite{Tabak2023}, investigates also in the the shape-texture bias and found that segmentation models are biased towards texture but still include shape-information in their decision-making. Other than paper \cite{Tabak2023} this work uses the bin-picking domain while the previously mentioned paper takes part in a biological domain. interestingly the work from \citeauthor{Tabak2023} does not try to overcome the texture-bias and uses highly realistic synthetic data to achieve a precise segmentation result and to overcome the sim-to-real task, which is also an important part of this work. This work, in contrast, does not try to utilize the texture-bias specifically, else the bias the investigated. Both works have in common that they use highly realistic synthetic data. This work uses Unreal Engine 5 for data generation while the paper just mentioned uses Blender. The paper shows that models trained on only synthetic data can achieve a \ac{iou} between 45\% and 63\%.\\
		Next, the paper \citetitle{Geirhos2022} \cite{Geirhos2022} have foundings towards the improved accuracy when increasing shape-biased in \ac{cnn}-based models. This is an interesting founding which will be indirectly also checked by this work. The cited paper also claims that \ac{cnn} base models are biased towards texture when trained on ImageNet on the object classification task. This work will proof this on the 3xM dataset within the task of instance segmentation. This work uses the Mask R-CNN which is \ac{cnn}-based.
		\clearpage
		For the sim-to-real transfer with the here proposed dataset, a comparison with another study's results are not possible due to the fact, that it is a novel, here proposed, dataset. The also used OCID dataset \cite{Suchi2019} is used more widely and most similar is the paper \citetitle{Xiang2021} \cite{Xiang2021}. It also uses Mask R-CNN \cite{Kaiming2017} and varies with RGB and RGBD as input data. The methodology and metrics in this paper are used differently, thus a comparison is not possible. %, but it can give a rough impression and help through the evaluation.
		
		% with the "Learning RGB-D Feature Embeddings for Unseen Object Instance Segmentation" \cite{Xiang2021} paper is done because the work also uses Mask R-CNN \cite{Kaiming2017} and varies with RGB and RGBD as input data. Moreover, it calculates metrics on the OCID dataset \cite{Suchi2019}, an available dataset suitable for bin-picking. The quantity variation is missing.
	
	\section{Hypothesis Statement}
	\label{sec:hypothesis-statement}	
		Through 3 experiments, this study will try to prove or disprove the following two claims. One for the core focus depth and one claim for the core focus shape-texture quantity: 
		\begin{enumerate}
			\item "Combining RGB and depth data combines texture and shape bias, leading to better generalization and sim-to-real ability." % better instance segmentation results
			\item "More distinction in shapes leads to higher shape awareness, better generalization, and sim-to-real ability. Furthermore, more distinction in texture leads to higher texture awareness, better generalization, and sim-to-real ability."
			%\item "When using RGB-D as input, less unique shapes and textures are needed for the same instance segmentation accuracy as using only RGB as Input."
		\end{enumerate}
		The 1. claim was chosen through the findings of a bias towards texture in instance segmentation \cite{Theodoridis2022} and the fact that depth data only contains shape information. In that way, the \ac{dnn} could be encouraged to use more shape information for decision-making. In addition, more balanced decision-making between texture and shape information could lead to better results \cite{Li2021}\cite{Co2021}\cite{Chung2023}.\\
		For the 2. claim, there is no foundation for the claim, apart from the fact that \ac{dnn}s (and \ac{cnn}s specifically) need a huge amount of data to generalize well \cite{Cho2016}\cite{Luca2022}. That could mean that more different shapes and textures also increase the ability to generalize, bridge from sim to real, and probably increase the decision-making towards shape/texture.
		%There is also no foundation for the 3. claim, except due to the additional information about shape and the probably increased shape awareness, the \ac{dnn} may need fewer different shapes and textures to achieve a similar precision and generalization as the \ac{dnn} trained with only RGB images and more different shapes and textures in the train-data.
	
	



			 % Externe Datei einbinden
\chapter{Methodology and Implementation}
\label{chap:kapitel3}
	% To find answers about the previous stated hypotheses in section \ref{sec:hypothesis-statement}, an experimental research design was chosen. An experimental research design suits well to this kind of hypothesis, and other designs would not fit such practical questions. 
	This study proposes a new synthetic dataset for instance segmentation, focusing on bin-picking as domain. The dataset uses the Unreal Engine 5.4.4 \cite{ue5} with hyper-realistic textures and shapes to create the data for the experiments as described in section \ref{sec:data}.\\
	To apply instance segmentation the widely used Mask R-CNN \cite{Kaiming2017} from PyTorch \cite{pytorch} is used \ref{sec:ai-model}.\\
	The coding for data preparation, \ac{dnn} training, and inference was done in Visual Studio Code \cite{vscode}, a flexible and feature-rich \ac{ide}. The data generation was programmed in the blueprint system from Unreal Engine 5 \cite{ue5} in the form of visual programming \cite{Romero2022}.\\
	For managing the Python environments, anaconda \cite{anaconda} got used and greatly impacted the workflow.\\
	The training and inference of all \ac{dnn}s and experiments have been done on three separate remote computers. One Linux-based computer with SSH and X11vnc remote connection from Optonic \cite{optonic} with an NVIDIA RTX 4090 as \ac{gpu}. And two Windows-based computers from Shadow-Tech \cite{shadow}, both with an NVIDIA RTX A4500.\\
	Furthermore this study proposes four in-distribution datasets and one real-world dataset for testing performance and sim-to-real ability, as described in more detail in section \ref{sec:test-data}.

	% \section{Tools and Environment}    % Tools and Environment
	% \label{sec:tools-and-environment}
	
	
	
	\section{AI-Model}
	\label{sec:ai-model}
		Starting with a summary about the used \ac{dnn}, Mask R-CNN \cite{Kaiming2017}. Notice that every implementation of Mask R-CNN can vary in a few points, but the core functionality remains the same.
		Mask R-CNN is the name of a \ac{dnn}, which is historical for instance segmentation. It builds up from Faster R-CNN \cite{Ren2016}, uses the ability to detect objects, and adds the creation of masks for every object. It is a widely adopted architecture and shines with its flexible nature and precise masks.\\
		To understand the fundamentals of this \ac{dnn}, it follows brief explanations about the underlying functionality of the Mask R-CNN followed by a summary to understand the whole process.\\
		\textbf{Residual Networks} (short ResNet) are the first part of Mask R-CNN's architecture. The network is a special type of \ac{cnn}s. \ac{cnn} are key architectures in modern computer vision. They detect spatial features and consist of convolution layers, which convolute images using filters or kernels to detect features like edges, texture, or other patterns, and of pooling layers, which reduce the size of the features by taking only the maximal value from every area and uses activation functions to be able to map non-linear functions \cite{Oshea2015}. The ResNet takes on the feature extraction and is most likely pre-trained. It uses skip connections to ensure the ability to generalize well in deeper layers. Output is a dense version of the origin image, which contains the features from it, called feature-map \cite{He2015}. Thus, the ResNet is the so-called backbone of the Mask R-CNN.\\
		Next, these feature maps get extended through a \textbf{\acl{fpn}} (short \ac{fpn}). The \ac{fpn} is used to get the feature maps on different scales to extract features from larger and smaller objects. The output from multiple ResNet layers is used with \ac{cnn}s to calculate new feature maps. In addition, it creates (also with \ac{cnn}s) a hierarchical structure (like a pyramid) with features in different resolutions and semantic \cite{Lin2017}. The \ac{fpn} also belongs to the backbone of the Mask R-CNN, since it also extracts features.\\
		After extracting the feature maps on different scales, a \textbf{\acl{rpn}} (short \ac{rpn}) uses these feature maps to build proposals that could contain objects called \ac{roi}s. First, the \ac{rpn} creates many anchor boxes for every X pixel. Then, a neural network classifies probabilities for every anchor box and only keeps the ones with high probability. Next up, a regression model refines the remaining anchor boxes. At last, the \ac{roi}s get filtered through \ac{nms}, which first filters the boxes with their scores, then removes boxes with too high \ac{iou}. This step is important because one object can have multiple anchor boxes, which is not wanted \cite{Ren2016}.\\
		One step is left to predict the mask, bounding box, and object class. These predictions require a fixed size of input images (feature maps/\ac{roi}s), but the \ac{roi}s can have different sizes, so a \textbf{\ac{roi} Align} is used. The \ac{roi} Align first combines the anchor boxes with the Feature-Maps (here just called \ac{roi}s). \ac{roi} Align is one method to resize images without much information loss, which is vital for further precise processing. It lays grids over the \ac{roi}s and uses precise float values for the positions, which leads to less information loss. Then a bilinear interpolation is applied to get the new resized \ac{roi}s \cite{Kaiming2017}.
		\clearpage
		One of the heads of Mask R-CNN is the {Object classification and bounding box regression}. A \ac{fc} is used for classification and bounding box regression. The \ac{fc} have a softmax layer to predict the class of the \ac{roi}s (often only background and one other object class). Moreover, a linear regression layer improves the bounding boxes for every \ac{roi} \cite{Ren2016}.\\
		% Since neural networks need the same input size and the \ac{roi}s can differ in width and height, the \ac{roi}s need to get resized (aligned) with as few information reductions as possible.
		Lastly, a \textbf{fully convolutional neural network} (short \ac{fcn}) is used for the prediction of the \ac{roi}'s masks \cite{Kang2014}.\\
		\\
		The procedure of instance segmentation with Mask R-CNN in short:
		\begin{enumerate}
			\item Feature-Map creation with Backbone (ResNet + FPN).
			\item \ac{roi} creation with \ac{rpn} including \ac{nms}.
			\item \ac{roi} align, for equal size conditions.
			\item Creation of the box head, using \ac{fc} for object classification and bounding box refinement.
			\item Creation of mask head, using \ac{fcn}.
		\end{enumerate}
		This architecture can be inspected in the \ref{img:maskrcnn} visualization.\\
		More information can be found here: \cite{Kaiming2017}\cite{Ramesh2021}. 
		
		\begin{figure}[h]
			\centering
			\includegraphics[width=\textwidth]{kapitel3/maskrcnn.png}
			\caption[Visualized Architecture of Mask R-CNN by Tobia Ippolito]{Visualized Architecture of Mask R-CNN}
			\label{img:maskrcnn}
		\end{figure}
		
		This work uses the torchvision implementation from PyTorch \cite{Torchvision}. It is a stable, professional implementation and works with modern graphic cards.\\
		Previously, the famous tensorflow implementation from Matterport \cite{Matterport} was used, but the implementation changed due to issues with newer graphic cards. Section \ref{sec:challenges} covers more of the challenges and learnings.
		
		\clearpage
		There was still much coding to make the \ac{dnn} work as intended. The Python file can be controlled by parameters and is well structured.\\
		One important part is the data loader, which first verifies the data, meaning it checks if there is a solution mask for every image and a depth image (if using depth). Finally, it finds the pixel value used for the background.\\
		During the training process, it loads the RGB image, the solution mask, and the depth image optionally. It also extracts a list of objects and bounding boxes from the solution mask. Data augmentations are significant for the robustness of the \ac{dnn}. These are random transformations and manipulation steps, including flipping, cropping, rotating, blurring, and adding noise to the image. A special data augmentation was applied to add more variation to the background since the background has only two different colors. The background augmentation is listed in \ref{lst:bg-augmentation} since it is not a custom data augmentation. The background augmentation can add noise, a checkerboard pattern, a color gradient, or a color shift. The chosen effect will only be applied to the background using the segmentation masks. \\
		Lastly, the data loader must prepare the input data for the right form for training, resizing, and transforming it into a tensor from PyTorch.
		
		\begin{lstlisting}[language=Python,caption=Random Augmentation of the Background using cv2 in Python, label=lst:bg-augmentation]
class Random_Background_Modification:
		...
		
		def __call__(self, images):
				rgb_img, depth_img, mask_img = images
				rgb_img, depth_img, mask_img = pil_to_cv2([rgb_img, depth_img, 
																																		mask_img])
				
				if random.random() < self.probability:
						mode = random.choice(["noise", "checkerboard",     
											  						 "gradient pattern", "color shift"])
						
						if mode == "noise":
								background_pattern = np.random.randint(0, 256, 
																								(self.height, self.width, 3), 
																								dtype=np.uint8)
						elif mode == "checkerboard":
								checker_size = random.choice([5, 10, 25, 50])
								color1 = random.randint(180, 255)
								color1 = [color1, color1, color1]    # Brighter Color
								color2 = random.randint(0, 130)
								color2 = [color2, color2, color2]    # Darker Color
							
								# Create the checkerboard pattern
								background_pattern = np.zeros((self.height, self.width, 3)
																														, dtype=np.uint8)
								for i in range(0, self.height, checker_size):
										for j in range(0, self.width, checker_size):
												color = color1 if (i // checker_size + 
																							j // checker_size) 
																							% 2 == 0 else color2
												background_pattern[i:i+checker_size, 
																							j:j+checker_size] = color
						elif mode == "gradient pattern":
								background_pattern = np.zeros((self.height, self.width, 3)
																														, dtype=np.uint8)
							
								# Generate a gradient
								if random.random() > 0.5:
										for i in range(self.height):
												color_value = int(255 * (i / self.height))
												background_pattern[i, :] = [color_value, 
																												color_value, 
																												color_value]
								else:
										for i in range(self.width):
												color_value = int(255 * (i / self.width))
												background_pattern[:, i] = [color_value, 
																												color_value, 
																												color_value]
						else:
								B, G, R = cv2.split(rgb_img)
							
								# create shift
								add_B = np.full(B.shape, random.randint(10, 150), 
																												dtype=np.uint8)
								add_G = np.full(G.shape, random.randint(10, 150), 
																												dtype=np.uint8)
								add_R = np.full(R.shape, random.randint(10, 150), 
																												dtype=np.uint8)
								
								# make shift
								shifted_B = cv2.add(B, add_B) if random.random() > 0.5 
																								else cv2.subtract(B, add_B)
								
								shifted_G = cv2.add(G, add_G) if random.random() > 0.5 
																								else cv2.subtract(G, add_G)
								
								shifted_R = cv2.add(R, add_R) if random.random() > 0.5 
																								else cv2.subtract(R, add_R)
								
								# apply shift
								background_pattern = cv2.merge((shifted_B, 
																									shifted_G, 
																									shifted_R))
								
						# apply pattern only on background:
						
						# get pattern in right size
						background_pattern = cv2.resize(background_pattern, 
																	(rgb_img.shape[1], rgb_img.shape[0]))
						
						# Create mask for background and objects
						bg_mask = (mask_img == self.bg_value).astype(np.uint8)
						fg_mask = 1 - bg_mask
						
						# Combine the original image and generated pattern
						background_with_pattern = cv2.bitwise_and(background_pattern, 
																													background_pattern, 
																													mask=bg_mask)
						objects_only = cv2.bitwise_and(rgb_img, rgb_img, mask=fg_mask)
						
						# Overlay the generated pattern and the original objects
						result = cv2.add(background_with_pattern, objects_only)
				else:
						result = rgb_img
				
				# Convert back to cv2
				result, depth_img, mask_img = cv2_to_pil([result, 
																										depth_img, 
																										mask_img])
				return result, depth_img, mask_img
		\end{lstlisting}
		
		\clearpage
		The training function needed experiment tracking, logging, printouts, learn rate scheduling, optimization, model loading, and more. For experiment tracking, mlflow and tensorboard were implemented. Both are helpful tools for tracking training and optimizing the training process. For learn rate scheduling, a simple custom scheduler was chosen with warm-up steps to first make small steps in the correct direction and then be able to increase the learning rate. Choosing a high learning rate at the beginning would lead to swinging up weight over-adjustments. \\
		The model loading and creation process was done without effort through the simple and well-functioning PyTorch implementation. The first layer gets optionally adjusted to add a depth channel to the input image. There is also an adjustment at the loss weights to weigh the segmentation more. The \acl{nms} got adjusted due to the clutter environments in bin picking. Listing \ref{lst:model-loading} shows the loading and adjustment. Notice how simple it is to adjust the architecture. The \ac{dnn} loading function also prints out every layer and a summary of the layers, which is a fascinating insight into the architecture.
		
		\begin{lstlisting}[language=Python,caption=Loading function of Mask R-CNN using torchvision, label=lst:model-loading]
def load_maskrcnn(
				weights_path=None, 
				use_4_channels=False, 
				pretrained=True,
				image_mean=[0.485, 0.456, 0.406, 0.5], 
				image_std=[0.229, 0.224, 0.225, 0.5],    
				min_size=1080, 
				max_size=1920, 
				log_path=None, 
				should_log=False, 
				should_print=True):
		
		backbone = resnet_fpn_backbone(backbone_name='resnet50',
																			weights=ResNet50_Weights.IMAGENET1K_V2
																			) 
		model = MaskRCNN(backbone, num_classes=2)  
		
		if use_4_channels:
				# Change the first Conv2d-Layer for 4 Channels
				in_features = model.backbone.body.conv1.in_channels    
				out_features = model.backbone.body.conv1.out_channels
				kernel_size = model.backbone.body.conv1.kernel_size
				stride = model.backbone.body.conv1.stride
				padding = model.backbone.body.conv1.padding
				
				
				# Create new conv layer with 4 channels
				new_conv1 = torch.nn.Conv2d(4, out_features,
																				kernel_size=kernel_size, 
																				stride=stride, 
																				padding=padding)
				
				# copy the existing weights from the first 3 Channels
				with torch.no_grad():
				new_conv1.weight[:, :3, :, :] = model.backbone.body.conv1.weight  # Copy old 3 Channels
				new_conv1.weight[:, 3:, :, :] = model.backbone.body.conv1.weight
																														[:, :1, :, :]
				
				
				model.backbone.body.conv1 = new_conv1
				
				# Modify the transform to handle 4 channels
				model.transform = GeneralizedRCNNTransform(min_size, max_size, 
																										image_mean, image_std)
		
		# adjust loss weights
		model.rpn.rpn_cls_loss_weight = 1.0
		model.rpn.rpn_bbox_loss_weight = 2.0
		model.roi_heads.mask_loss_weight = 2.0
		model.roi_heads.box_loss_weight = 1.0
		model.roi_heads.classification_loss_weight = 1.0
		
		# adjust non-maximum suppression
		model.roi_heads.nms_thresh = 0.4
		model.roi_heads.box_predictor.nms_thresh = 0.4  
		model.roi_heads.mask_predictor.mask_nms_thresh = 0.4
		model.roi_heads.score_thresh = 0.4
		
		
		# load weights
		if weights_path:
				model.load_state_dict(state_dict=torch.load(weights_path, 
																												weights_only=True)) 
		
		
		
		# printing the architecture
		model_str = "Parameter of Mask R-CNN:"
		model_parts = dict()
		[...]
		
		log(log_path, model_str, should_log=should_log, 
																should_print=should_print)
		
		return model
		\end{lstlisting}
		
		\ac{sgd} with Nesterov-Momentum \cite{Botev2016} was used as learn rate optimizer. \ac{sgd} is known for its good generalization and preciseness but also for its slowness. The Nesterov-Momentum was used to increase convergence speed and stabilize the process.\\
		The train loop itself is a typical PyTorch train loop. \\
		Lastly, the inference is essential to use the \ac{dnn}. The same data loader can be used for the inference due to its flexibility. The inference can process multiple images and create optional visualizations from the created mask and the ground truth (if existing). Exceptional is the feature of visualizing in-between inference steps, as the feature maps from the \ac{fpn} or the \ac{roi} align. For that, hooks were implemented, which call a function when the layer runs. The hook function saves the output of the layer in a dictionary. For the visualization, it was tricky to handle all the different sizes. It is clear to see every described step of the network and makes it sufficiently coherent; thus, listing \ref{lst:inference-insight} shows the described code for the hooking and the visualization, and appendix \ref{appendix:inference-insights} shows samples from one example inference with the network insights.\\
		Additionally, given the ground truth, many different metrics can be calculated during the inference. The inference result is saved as a gray image or a numpy array. The whole process has a print-out for information about the progress.\\
		The coding for the training and inference includes about 66 defined functions, 14 defined classes, one global keyword, 33 imports, 794 called functions and classes, 50 for-loops, 223 if statements, 194 bool operations, and 182 arithmetic operations.
	
		\clearpage
	
	
	\section{Train Data}
	\label{sec:data}
		Nine datasets were created with Unreal Engine 5 \cite{ue5} for the experiments. Each has RGB, mask, and depth images; there are 20.000 images per dataset (60.000 images counting RGB, depth and mask separately). All datasets together need 500-600 GB memory space. The resolution of the images is Full-HD (1920 x 1080). One dataset needed about two days to complete the generation process. Appendix \ref{appendix:traindata-examples} holds examples for every generated dataset.\\
		The visual program in Unreal Engine 5 first makes adjustments to the background, bin box, lighting, and camera. Then, random materials and shapes gets combined and spawned in the bin box. The Materials and Shapes are used from Quixel Megascans \cite{Quixel}, a collection of high-quality assets. Samples of these shapes are available at the appendix \ref{appendix:shapes-for-training} and sample materials/textures are viewed in appendix \ref{appendix:materials-for-training}.\\
		An RGB and depth image is rendered and saved using a structural name convention. The depth image is scaled and converted to the bit-depth of 8; where the ground of the box a high value, near or equal 255, has and the closer objects come, the closer they get to 0 (black).\\
		Next, every object gets a unique color and a new material with that color, and the background and bin-box get a black color. The image is now rendered with a base color renderer, which only takes the base color of every object without any exposure or other effect; the result is an RGB image with 0 as the background, and every object has a unique color. This process will be repeated 20.00 times, and the list of materials and shapes is fixed during this time. The list of materials and shapes is adjusted by switching to the next dataset, and the process starts from new.\\
		\\
		It follows a more in-depth description of visual code for data generation. Only the most important parts can be covered. The code is too massive to show everything in this work here.\\
		The data generator has 24 parameters for customization, as listed in appendix \ref{appendix:custom-params-ue5}.\\
		The initialization nodes only run once after starting the data generation process. First, it checks if the previously listed user parameters are correct. Verification is beneficial to detect early problems and mistakes. Then, every variable receives a reset to ensure everything is ready for the data generation. Figure \ref{img:ue5_init} shows this initial code.
		
		\begin{figure}[h]
			\centering
			\includegraphics[width=\textwidth]{kapitel3/init.jpeg}
			\caption[Visual Code for Initialization in Unreal Engine 5 by Tobia Ippolito]{Visual Code for Initialization in Unreal Engine 5}
			\label{img:ue5_init}
		\end{figure}
		
		According to the initialization, the main loop is called every frame. Many variables and branches are used to determine which steps are already made and which should be done next.\\
		The main loops start with changing the background material and the material of the bin box. After that, the lights get adjusted, which consist of 3 rectangular light boxes. Next, three renderers are created and adjusted to the camera: one for RGB capture, one for depth capture, and one for segmentation mask capture. The next step is noteworthy; here, a random amount of objects with random materials and shapes are scaled, created, and spawned by the bin box. If an object falls out of the box, it automatically gets respawned in the box. A RGB image is taken if all objects stop moving or a given time limit is surpassed. It continues with taking a depth image. Before the segmentation mask image can be made, every object gets a new material with a unique plain color, and the background and bin box are assigned to a black material. The segmentation image can now be taken, and the scene is over. Figure \ref{img:ue5_main} shows a small part of the main loop.
		
		\begin{figure}[h]
			\centering
			\includegraphics[width=\textwidth]{kapitel3/main.jpeg}
			\caption[Visual Code for creating a scene and taking an RGB, depth, and mask image in Unreal Engine 5 by Tobia Ippolito]{Visual Code for creating a scene and taking an RGB, depth, and mask image in Unreal Engine 5}
			\label{img:ue5_main}
		\end{figure}
		
		After creating a scene, every important variable is set to the default value, and the image counter is increased. Then, the program checks whether to change the dataset (changing the material/and shape amount) and reset the image counter or continue with the current dataset. This behavior is presented in figure \ref{img:ue5_dataset_loop}.\\
		When switching to another dataset, there is always a selection of materials and shapes for the whole dataset. The selection can be random or ordered, depending on the chosen parameters. The selection process is viewed in figure \ref{img:ue5_material_shape_choice}.
		
		\begin{figure}[h]
			\centering
			\includegraphics[width=\textwidth]{kapitel3/dataset_loop.jpeg}
			\caption[Visual Code for changing the dataset and prepare the next scene in Unreal Engine 5 by Tobia Ippolito]{Visual Code for changing the dataset and prepare the next scene in Unreal Engine 5}
			\label{img:ue5_dataset_loop}
		\end{figure}
		
		\begin{figure}[h]
			\centering
			\includegraphics[width=\textwidth]{kapitel3/shape_material_choice.jpeg}
			\caption[Visual Code for drawing a subset of a given material and shape list in Unreal Engine 5 by Tobia Ippolito]{Visual Code for drawing a subset of a given material and shape list in Unreal Engine 5}
			\label{img:ue5_material_shape_choice}
		\end{figure}
		
		During the data generation process, Python scripts were also highly required for processing shapes and materials, post-processing the segmentation masks to convert them to gray images, and unzipping a large number of folders.\\
		\\
		The nine in this work proposed datasets called \textbf{3xM} ("triple m") for \textbf{m}odel-\textbf{m}aterial-\textbf{m}ixture which refers to the combinations of different unique amounts of shapes and textures.
		

	\section{Test Data}
	\label{sec:test-data}
		Four in-distribution test datasets with 100 RGB, depth, and mask images were created with the methodology from section \ref{sec:data}.
		The datasets differ in the novelness of their shapes and textures. There is one test dataset with known shapes and known textures, one with unknown shapes and known textures, one with known shapes and unknown textures, and one with unknown shapes and unknown textures. This allows to test not only the performance on in-distribution data but also shows the generalization ability of shapes and textures.\\
		\\
		This study also proposes a real-world dataset in the bin-picking domain. The dataset consists of 70 labeled RGB-D images as shown in appendix \ref{appendix:testdata-examples-simtoreal}. All images were made with the \href{https://www.optonic.com/produkte/ensenso/b57/}{B57-2 camera} from Optonic GmbH \cite{optonic} as shown in figure \ref{img:camera} and consist of 2472 x 2064 pixels.
		\begin{figure}[h]
			\centering
			\includegraphics[width=0.6\textwidth]{kapitel3/camera_B57-4.jpg}
			\caption[RGB-D Camera Ensenso B57-2 from Optonic. More informations on: \url{https://www.optonic.com/produkte/ensenso/b57/}]{RGB-D Camera Ensenso B57-2 from Optonic. More informations on: \url{https://www.optonic.com/produkte/ensenso/b57/}}
			\label{img:camera}
		\end{figure}
		\FloatBarrier
		The setup was a simple black bin with a mat (foam-like) inside and the camera attached to a table, as shown in figure \ref{img:real-data-setup}.
		Twenty-two unique objects were used in an organized way to produce the data scenes: twelve industrial parts and ten everyday consumer goods. In total, 62 objects were used. The scenes are organized in cluttered/uncluttered, solo/mixed (only one type of unique object or different unique object types), and industrial/product/industrial \& product.
		\clearpage
		The objects contain:
		\begin{itemize}
			\item 12x Brass Parts
			\item 9x Complex White Plastic Parts
			\item 8x Round White Plastic Parts
			\item 7x Black Metal Mounting Plates
			\item 5x White T-Form Plastic Parts
			\item 4x Chocolate Candies (different shapes)
			\item 3x Smaller Metal Parts
			\item 3x Small Black Metal Mounting Plates
			\item 2x Metal Parts
			\item 1x USB Stick
			\item 1x Hard Disk
			\item 1x Screwdriver
			\item 1x Pen
			\item 1x Tea
			\item 1x Instant Noodles
			\item 1x Nut Mix
			\item 1x Italian Herbs
			\item 1x Tissues
		\end{itemize}
		Figure \ref{img:real-data-objects} shows all used objects.
		\begin{figure}[h]
			\centering
			\includegraphics[width=0.85\textwidth]{kapitel3/real_data_objects.jpg}
			\caption[All used objects for real-world data collection.]{All used objects for real-world data collection.}
			\label{img:real-data-objects}
		\end{figure}
		
		The connection to the camera happened through the network with the NxView Software from Optonic GmbH. Then, every scene was saved in ZIP file format and later loaded as a file camera in NxView, which behaves similarly to a real camera and was used by the Python script in listing \ref{lst:file-camera}. The listing \ref{lst:file-camera} also shows the used scaling for depth information. The depth scaling starts with decreasing all positive values greater than zero to zero since zero should be the ground of the bin (the camera was calibrated before). Next, the depth data is converted from negative to positive space and missing values are replaced with linear interpolation. If there are values greater than 255, a min-max normalization is applied. A stretching method uses the lowest and highest values to cover the whole range of values. Then, the depth data can be converted to integer values with 8-bit depth, which can be saved and used.
		
		\begin{lstlisting}[language=Python,caption=Capturing images from a file camera with depth image scaling, label=lst:file-camera]
import os
import json
import zipfile

import numpy as np
import cv2
from scipy.interpolate import griddata

from nxlib import NxLib, Camera, NxLibItem
from nxlib.context import NxLib, StereoCamera
from nxlib.command import NxLibCommand
from nxlib.constants import *

with open("./params.json", "r") as file:
		params = json.loads(file.read())

os.makedirs("./data", exist_ok=True)
os.makedirs("./data/rgb", exist_ok=True)
os.makedirs("./data/depth", exist_ok=True)

def interpolate_nan(image):
		x, y = np.indices(image.shape)
		nan_mask = np.isnan(image)
		
		# Coordinates of valid (non-NaN) points
		valid_coords = np.array((x[~nan_mask], y[~nan_mask])).T
		# Values of valid points
		valid_values = image[~nan_mask]
		# Coordinates of NaN points
		nan_coords = np.array((x[nan_mask], y[nan_mask])).T
		
		# Interpolate NaN points
		interpolated_values = griddata(valid_coords, valid_values, nan_coords, method='linear')
		
		# Fill the interpolated values back into the image
		result = image.copy()
		result[nan_mask] = interpolated_values
		
		return result

serial_number = "all_scenes_1"

with NxLib(), StereoCamera(serial_number) as camera:
		camera.get_node()[ITM_PARAMETERS].set_json(json.dumps(params[ITM_PARAMETERS]), True)
		
		first_image = None
		counter = 0
		while True:
				camera.capture()
				camera.rectify()
				camera.compute_disparity_map()
				camera.compute_point_map()
				
				# get unique name
				name = "optonic_bin-picking_dataset_00.png"
				name_counter = 1
				while name in os.listdir("./data/rgb"):
						name = f"optonic_bin-picking_dataset_{name_counter:02}.png"
						name_counter += 1
				
				# calc depth map
				points = camera.get_point_map()
				depth_data = points[:,:,2]
				depth_data = np.minimum(depth_data, 0)
				depth_data = depth_data + abs(np.nanmin(depth_data))
				depth_data = interpolate_nan(depth_data)
				depth_data = depth_data.astype(np.int16)
				
				
				if np.nanmax(depth_data) <= 255:
						depth_data = depth_data.astype(np.uint8)
				else:
						depth_data = ((depth_data - np.nanmax(depth_data)) / 
														(np.nanmax(depth_data) - np.nanmin(depth_data))
														).astype(np.uint8)
						depth_data = (depth_data * 255).astype(np.uint8)
				
				min_val = np.min(depth_data)
				max_val = np.max(depth_data)
				
				# Apply linear scaling to stretch the pixel values
				depth_data = ((depth_data - min_val) / (max_val - min_val)) * 255
				depth_data = np.uint8(depth_data)
				
				# get rgb
				image = camera.get_texture()
				image = cv2.cvtColor(image, cv2.COLOR_BGR2RGB)
				
				if first_image is None:
						first_image = image
				elif np.array_equal(first_image, image):
						break
				
				# save images
				cv2.imwrite(f"./data/depth/{name}", depth_data)
				cv2.imwrite(f"./data/rgb/{name}", image)
				print(f"Successfull extracted image {counter+1}")
				counter += 1
		\end{lstlisting}
		
		The author hand-labeled every image using a semi-automatic annotation approach with CVAT \cite{cvat}. All images were pre-labeled with the SAM model \cite{Kirillov2023} and manually corrected and adjusted. The annotation software runs entirely on the web and does not require any installments. This procedure enabled a quick annotation process without any troubles.
	
		
		\begin{figure}[h]
			\centering
			\includegraphics[width=\textwidth]{kapitel3/real_data_setup.jpg}
			\caption[Setup for collecting real-world bin-picking data.]{Setup for collecting real-world bin-picking data.}
			\label{img:real-data-setup}
		\end{figure}





\chapter{Experiment Setup and Description}
\label{chap:kapitel4}

	
	
	\section{Experiment Design}
	\label{sec:experiment-design}
	
	
	
	\section{Experiment Metrics}
	\label{sec:experiment-metrics}
	
	
	
	
	



\chapter{Results and Discussion}
\label{chap:kapitel5}

	\section{Results}
	\label{sec:results}



	\section{Interpretation of Results}
	\label{sec:interpretation-of-results}
	
	
	
	% \section{Hypotheses Confirmation}
	% \label{sec:hypotheses-conformation}
	
	
	
	\section{Discussion} % Limitations and Challenges
	\label{sec:doscussion}
	
	% Hypotheses Confirmation
	
	\iffalse
	Diskussion - Bias Experiment: In the bias experiment it is difficult to say if a prediction is biased towards texture or shape. All confusing data is only confusing on texture level, so a texture biased model should perform poorly in comparison to a shape biased model. But also a shape biased model will use texture information and the question is how much and how the impact really is. In addition there are other influences, like amount of objects per scene, brightness, reflective texture, novel texture, novel shape and all of these can influence the result. -> my opinion? what does the results look like?
	It seems like that there are learned shape and texture dependent decisions
	\fi
	
	
	









\chapter{Conclusion}
\label{chap:kapitel6}
	This work proposes nine labeled RGBD datasets, each with 20,000 images, for training with varying amounts of shapes and textures; four labeled RGBD datasets for testing with varying novelness of shapes and textures, each with 100 images; one small set of 8 exceptional scenarios with puzzling textures; and one real-world RGBD dataset for bin-picking with 70 images.\\
	Furthermore, this work presents findings regarding depth information and the amount of shapes and textures in instance segmentation, as described in section \ref{sec:summary-of-findings}.


	\section{Summary of Findings}
	\label{sec:summary-of-findings}
		This study investigated the influence of depth information and the amount of shape and texture towards shape-texture bias and performance on instance segmentation using models trained with the 3xM datasets. Several key findings emerged:
		\begin{enumerate}
			\item \textbf{Shape vs. Texture Bias:} RGB-only models displayed a clear bias toward texture, while RGB-D models shifted slightly toward shape bias but still exhibited texture bias. Interestingly, increasing the number of shapes often decreased shape bias, challenging the hypothesis that more diverse shapes lead to stronger shape bias.
			
			\item \textbf{In-Distribution Performance:} Models trained with both RGB and depth data achieved a slight improvement (1.5\% mean IoU) over RGB-only models. However, these improvements were context-dependent, with significant variations across different experimental setups. The experiments show that additional depth information can improve the accuracy of Mask R-CNN by unknown textures.\\
			Increasing the number of shapes and textures generally improved performance. The performance especially increases with a higher number of shapes with unknown shapes; also with a higher amount of unknown textures, increasing the texture amount in the training data improves the accuracy of the \ac{dnn}'s mask prediction. 
			\\
			\\
			\item \textbf{Sim-to-Real Performance:} Sim-to-real experiments revealed divergent results. RGB-D models excelled on the OCID dataset, achieving 21.34\% higher mean IoU than RGB-only models, but performed worse on the Optonic Bin-Picking dataset. This suggests that dataset-specific factors such as sharpness, texture quality, and lighting conditions heavily influence performance. Notably, RGB-only models outperformed RGB-D models by 22.26\% on the Optonic Bin-Picking dataset, likely due to differences in data quality and sharpness.\\
			The rising accuracy using increasing shape amounts could depend on the number of unknown shapes. There also seems to be a relation between the rising accuracy in sim-to-real from increasing the number of textures and the number of unknown textures in the real-world dataset.
			
			\item \textbf{Generalization:} The hypothesis that depth data improves generalization was not fully supported. While depth data mitigated the lack of novel textures, it did not enhance generalization toward novel shapes. The quality of both RGB and depth data appears to play a crucial role, highlighting the importance of image sharpness and resolution.\\
			The number of shapes appears to influence the network's ability to generalize to new shapes, particularly when encountering unfamiliar ones. Similarly, the number of textures tends to enhance the generalization of textures, especially when novel textures are present in the data.
		\end{enumerate}
		
		These findings demonstrate the complex interplay of depth information and shape-texture amount in influencing the performance and generalization of \ac{dnn}s in instance segmentation tasks. They underscore the importance of dataset diversity and quality in shaping model biases and achieving robust performance and generalization across varied contexts.
	
	
	\section{Implications and Future Work}
	\label{sec:implications-and-future-work}
		Although this work acquired new empirical values and trends, there are many open questions to answer. This section describes open questions and recommended future research.
		
		\textbf{Depth Information Quality} could be a crucial factor for influencing the performance and sim-to-real ability, as stated in section \ref{sec:interpretation-of-results}. While the influence of RGB-image quality is already covered by research, the influence of depth information quality remains open and should be further investigated. The ambiguous result from the sim-to-real experiment in \ref{sec:results} hints at the importance of depth information quality but still can not prove it. Further sim-to-real experiments could help to confirm or reject this hypothesis. If this hypothesis is true, then research about augmentation to reduce the negative impact of noisy depth sensors would be very important.
		\clearpage
		The influence of the \textbf{\ac{dnn} Architecture} is not part of the experiments of this work. It could lead to intriguing results to run the here used experiments on other \ac{dnn} architectures. In detail, it would be interesting to see the results from different kinds of architectures, like a comparison between \ac{cnn} based, transformer, and reinforcement architectures. This could yield novel insights into the different architectures and their processing differences. Is the influence of depth information and shape-texture amount similar over different \ac{dnn} architectures?
		
		A fascinating research would be the investigation of the influence of different \textbf{integrations of depth information} towards in-distribution performance, sim-to-real ability, and generalization. The proposed experiments with different depth information handling \ac{dnn}s could give insights into the performance of different integrations of depth information and how the shape-texture amount can influence them. Depth information can be provided as the fourth channel (the here used approach) but also can be incorporated through specialized architectures, such as separate depth-processing branches, solely use or fusion layers, to improve the usage of depth information's unique spatial and geometric properties.
		
		Further \textbf{Sim-To-Real} experiments in other domains could assess the applicability of this study's conclusions.
		
		In addition, the Shape vs. Texture Attention Test could be extended with \textbf{explainability} methods to better understand the decision-making processes, which could lead to more precise results regarding the influence of depth information and shape-texture amount on the shape-texture bias of Mask R-CNN in instance segmentation.
		
		Moreover, a similar investigation with \textbf{higher numbers} of shapes and textures could lead to novel insights about the limits and potentials of higher amounts. This work suggests that further shapes and textures would increase the generalization ability for shapes and textures even more. The quality of this rise is uncertain.
		
		Finally, this work recommends an investigation into the roles of image sharpness (of RGB and depth data), lighting conditions, and texture quality in influencing sim-to-real results. It would also be valuable to determine more essential factors and how they influence the instance segmentation performance.
	
		% using depth in another way? => same results?
		% ...
		% Investigating other input types, like only depth images, would be a interesting comparison and could give more insights.
		% ...
		% It also cold be interesting to make a similar investigation towards robustness. Robustness is a import factor of a \ac{dnn}, especially in sensitive fields, like autonomous driving, where new factors like new light settings happens all the time and live could depend on the robustness of a \ac{dnn}.
		
	\clearpage
	\section{Challenges and Learnings}    % Learnings
	\label{sec:challenges}
		During this study, many challenges occur, and maybe the solutions of them can help in the future.\\
		The journey started with an NVIDIA GeForce GTX 1080 Ti graphic card and Mask R-CNN implementation from Matterport \cite{Matterport}. The implementation from Matterport only works with old versions of Tensorflow (Tensorflow was one given restriction), so the first task was to update the code from Matterport to a newer version. A strange, unwanted behavior occurred, where loading the same weights led to different, random results. Debugging was difficult due to the massive size of the \acl{dnn}. \\
		Switching to an already finished upgraded Mask R-CNN seemed to make more sense, but all five tried implementations did not want to work properly. So, the next approach was to try the old version of Mask R-CNN from Matterport. This old implementation came with massive problems with the newer Linux system and the old Python version needed for the Tensorflow version. The solution was virtual python environments with Anaconda \cite{anaconda}. These conda environments worked like a breeze and were also easy to replicate. This version worked, but the training had to stop since the graphic card changed to the newer NVIDIA RTX4090, and the Tensorflow version would not work on this new \ac{gpu}. It was finally time to switch to another architecture. YOLACT \cite{Bolya2019} was now attempted. First, an unofficial Tensorflow version, but quickly changed to the official PyTorch version; thus, there was already much time spent, and it seemed to be not working very well again, so the restriction of Tensorflow was lifted. This official YOLACT implementation also did not work with the new \ac{gpu} and was upgraded. During the upgrade, the code underwent tremendous changes to make it more accessible in Python; before, it was programmed to be used with parameters and not within Python. After all that work was done, the \ac{dnn} seemed to work, but it turned out that something was wrong with the network. It always found too many masks, and debugging was a large endeavor. At that time, only about one to two months were left (the data generation also needed much time to get work), so the decision was to try out the Mask R-CNN again, but with PyTorch's official implementation. PyTorch's implementation worked surprisingly well and quickly. \\
		The next challenge is the most common all over the world: time. There was little time left, but the \ac{dnn} needed much time to train. The ideal would be 100 to 500 epochs for the best results. With 18 \ac{dnn} to train and a time of 3 days with 100, the study would take 54 days only for the training.\\
		So, a second remote computer got leased from \cite{shadow}, and epochs were reduced to 20. The results were not optimal and not acceptable. So, a third remote computer was leased, and the epochs increased to 40. Three remote computers, each six \ac{dnn} to train with about 1 to 2 days computation time per network, make about two weeks to complete the whole training process. \\
		Many other challenges appeared between these challenges, like operating system failures, changing hardware, SSH remote connection issues, storage shortage, \ac{gpu} tribe difficulties,  system-specific inconsistencies, and many more.\\
		Another challenging part of this study was the creation of the synthetic datasets. The proposed data generator in Unreal Engine 5 needed much work and a finishing touch to be in a ready-to-use state. It started with collecting enough shapes and materials of a high quality and different appearance and continued with the programming of the data generator itself. It was challenging to get along with the - for games designed - Unreal Engine 5 and to render and save a virtual camera. Also, the creation of the segmentation masks took work. Moreover, many bugs occurred with the spawning objects, like rotating too fast, wrong scaling and falling out of the box. Most bugs were fixed, but the data generation is still imperfect.\\
		In the end, the generation of 9 datasets needed a long time, about ten days, and every small change needed a restart of the whole generation process.\\
		To carry it to the extreme, in the middle of this study, a personal challenge from the author popped up out of nowhere. The right side of the author's face experienced complete paralysis called peripheral facial paralysis. He visits many medical institutions for several months, from hospitals to house doctors, otologists, alternative practitioners, acupuncture, and neurologists. The right eye was at risk of drying out, and the sense of taste was strange due to the numbness. \\
		In the end, the author is grateful to have faced all these challenges with a lot of learnings and completed this work despite all its difficulties.\\
		In summary, these challenges provided many learnings. First, effective time and prioritization management are fundamental to achieving goals in time and with limited resources. With a well-thought-out time and prioritization management, time can quickly be well-spent.\\
		Secondly, overcoming unexpected obstacles and finding creative solutions with flexibility and persistence is important. The change to other architectures during this study took too long and caused much trouble at the end of this work. Also, clear and open communication is helpful and worthwhile. Miscommunication or a lack of communication can cause much extra work and cause new problems. 
		More specifically, avoiding unforeseen (but often occurring) issues with different operating systems can be tackled with virtual environments. Virtual environments can lead to mainly deterministic behavior; thus, they should be integrated into the working and experiment pipeline. Software like Anaconda or Docker can achieve such deterministic behavior. Docker was not used during this work, but it would lead to much fewer issues with different operating systems. This work used Anaconda, as described, as a virtual Python environment, which has already made this work much easier.\\
		The last learning is to \textit{fail fast}. Failing is part of everybody's life and every project. Failing can be full of learning and improvement, but it is important to fail fast in a time-limited project. During debugging, it is recommended that the code be designed and adjusted so that it fails fast if it even fails.
		
		% Learning...
		
	
	
	
	
	
	



% ------------------------------------------------------------------

\label{lastpage}

% Neue Seite
\cleardoublepage

% Backmatter mit normalem Zeilenabstand setzen
\singlespacing

% Römische Ziffern für die "Back-Matter", fortlaufend mit "Front-Matter"
\pagenumbering{roman}
%\setcounter{page}{\value{frontmatterpage}}
\setcounter{page}{0}
% Abkürzungsverzeichnis
\addchap{\hsmaabbreviations}
\begin{acronym}[IEEE]
	\acro{ood}[OOD]{Out Of Distribution}
\end{acronym}


% Tabellenverzeichnis erzeugen
%\cleardoublepage
%\phantomsection
%\addcontentsline{toc}{chapter}{\hsmalistoftables}
%\listoftables

% Abbildungsverzeichnis erzeugen
\cleardoublepage
\phantomsection
\addcontentsline{toc}{chapter}{\hsmalistoffigures}
\listoffigures

% Listingverzeichnis erzeugen
\cleardoublepage
\phantomsection
\addcontentsline{toc}{chapter}{\hsmalistings}
\lstlistoflistings

% Literaturverzeichnis erzeugen
\begin{flushleft}
\printbibliography
\end{flushleft}

% Index ausgeben. Wenn Sie keinen Index haben, entfernen Sie einfach
% diesen Teil.
%\cleardoublepage
%\phantomsection
%\addcontentsline{toc}{chapter}{\hsmaindex}
%\printindex

% Anhang. Wenn Sie keinen Anhang haben, entfernen Sie einfach
% diesen Teil.
\appendix
\chapter{Used Hyperparameters in Training}
\label{appendix:hyperparameter}
	Every model in this study used following hyperparameters.

	\begin{itemize}
		\item \textbf{\ac{dnn}:} Mask R-CNN from torchvision
		\item \textbf{Epochs:} 40
		\item \textbf{Data amount:} 20000
		\item \textbf{Width and height:} 1920 x 1080
		\item \textbf{Warm-up iterations:} 2000
		\item \textbf{Learning rate:} 0.003
		\item \textbf{Scheduler:} Simple custom scheduler with warm-up and down-regulation
		\item \textbf{Optimizer:} \acl{sgd} with Nesterov momentum \cite{Botev2016}
		\item \textbf{Momentum:} 0.9
		\item \textbf{Batch size:} 5
		\item \textbf{Shuffle data:} True
		\item \textbf{Data Augmentations:} Random flip, rotation, crop, brightness contrast, noise, blur, scale and background modification
	\end{itemize}
\chapter{Inference Examples on Optonic Bin-Picking Dataset}
\label{appendix:inference-examples}
	Three example inferences on the Optonic Bin-Picking Dataset are provided in this appendix using a model trained 40 epochs on only synthetic RGB train data (3xM Dataset) with 160 shapes and 160 textures.\\
	A training on the real data could quickly lead to precise results without the frequently occurring artifacts, detected in the background of the scene and could also improve the performance with cluttered objects.

	\begin{figure}[H]
		\centering
		\includegraphics[width=0.8\textwidth]{anhang/inference-examples/optonic-example-rgb-160-160-1.jpg}
		\caption[An examle inference on an observation from the Optonic Bin-Picking Dataset using a model trained 40 epochs on only synthetic RGB train data with 160 shapes and 160 textures.]{An examle inference on an observation from the Optonic Bin-Picking Dataset using a model trained 40 epochs on only synthetic RGB train data with 160 shapes and 160 textures.}
	\end{figure}
	
	\begin{figure}[H]
		\centering
		\includegraphics[width=0.8\textwidth]{anhang/inference-examples/optonic-example-rgb-160-160-2.jpg}
		\caption[An examle inference on an observation from the Optonic Bin-Picking Dataset using a model trained 40 epochs on only synthetic RGB train data with 160 shapes and 160 textures.]{An examle inference on an observation from the Optonic Bin-Picking Dataset using a model trained 40 epochs on only synthetic RGB train data with 160 shapes and 160 textures.}
	\end{figure}
	
	\begin{figure}[H]
		\centering
		\includegraphics[width=0.8\textwidth]{anhang/inference-examples/optonic-example-rgb-160-160-3.jpg}
		\caption[An examle inference on an observation from the Optonic Bin-Picking Dataset using a model trained 40 epochs on only synthetic RGB train data with 160 shapes and 160 textures.]{An examle inference on an observation from the Optonic Bin-Picking Dataset using a model trained 40 epochs on only synthetic RGB train data with 160 shapes and 160 textures.}
	\end{figure}



\chapter{Shapes for Training}
\label{appendix:shapes-for-training}

	The amount of unique shapes are one essential part of this work, thus it is important to see some examples of the used shapes.\\
	Here are 20 different shapes, used in the data generation.
	
	\begin{figure}[h]
		\centering
		\begin{subfigure}{0.45\textwidth}
			\centering
			\includegraphics[width=\textwidth]{anhang/shape-1.png}
			\caption[Example of used Shape, created by \cite{Quixel}]{Example of used Shape}
		\end{subfigure}
		\begin{subfigure}{0.45\textwidth}
			\centering
			\includegraphics[width=\textwidth]{anhang/shape-2.png}
			\caption[Example of used Shape, created by \cite{Quixel}]{Example of used Shape}
		\end{subfigure}
		
		\begin{subfigure}{0.45\textwidth}
			\centering
			\includegraphics[width=\textwidth]{anhang/shape-3.png}
			\caption[Example of used Shape, created by \cite{Quixel}]{Example of used Shape}
		\end{subfigure}
		\begin{subfigure}{0.45\textwidth}
			\centering
			\includegraphics[width=\textwidth]{anhang/shape-4.png}
			\caption[Example of used Shape, created by \cite{Quixel}]{Example of used Shape}
		\end{subfigure}
		
		\begin{subfigure}{0.45\textwidth}
			\centering
			\includegraphics[width=\textwidth]{anhang/shape-5.png}
			\caption[Example of used Shape, created by \cite{Quixel}]{Example of used Shape}
		\end{subfigure}
		\begin{subfigure}{0.45\textwidth}
			\centering
			\includegraphics[width=\textwidth]{anhang/shape-6.png}
			\caption[Example of used Shape, created by \cite{Quixel}]{Example of used Shape}
		\end{subfigure}
	
	\end{figure}
	
	
	
	\begin{figure}[h]
		\centering
		\begin{subfigure}{0.45\textwidth}
			\centering
			\includegraphics[width=\textwidth]{anhang/shape-7.png}
			\caption[Example of used Shape, created by \cite{Quixel}]{Example of used Shape}
		\end{subfigure}
		\begin{subfigure}{0.45\textwidth}
			\centering
			\includegraphics[width=\textwidth]{anhang/shape-8.png}
			\caption[Example of used Shape, created by \cite{Quixel}]{Example of used Shape}
		\end{subfigure}
		
		\begin{subfigure}{0.45\textwidth}
			\centering
			\includegraphics[width=\textwidth]{anhang/shape-9.png}
			\caption[Example of used Shape, created by \cite{Quixel}]{Example of used Shape}
		\end{subfigure}
		\begin{subfigure}{0.45\textwidth}
			\centering
			\includegraphics[width=\textwidth]{anhang/shape-10.png}
			\caption[Example of used Shape, created by \cite{Quixel}]{Example of used Shape}
		\end{subfigure}
		
		\begin{subfigure}{0.45\textwidth}
			\centering
			\includegraphics[width=\textwidth]{anhang/shape-11.png}
			\caption[Example of used Shape, created by \cite{Quixel}]{Example of used Shape}
		\end{subfigure}
		\begin{subfigure}{0.45\textwidth}
			\centering
			\includegraphics[width=\textwidth]{anhang/shape-12.png}
			\caption[Example of used Shape, created by \cite{Quixel}]{Example of used Shape}
		\end{subfigure}
		
		\begin{subfigure}{0.45\textwidth}
			\centering
			\includegraphics[width=\textwidth]{anhang/shape-13.png}
			\caption[Example of used Shape, created by \cite{Quixel}]{Example of used Shape}
		\end{subfigure}
		\begin{subfigure}{0.45\textwidth}
			\centering
			\includegraphics[width=\textwidth]{anhang/shape-14.png}
			\caption[Example of used Shape, created by \cite{Quixel}]{Example of used Shape}
		\end{subfigure}
		
	\end{figure}
	
	
	
	\begin{figure}[h]
		\centering
		\begin{subfigure}{0.45\textwidth}
			\centering
			\includegraphics[width=\textwidth]{anhang/shape-15.png}
			\caption[Example of used Shape, created by \cite{Quixel}]{Example of used Shape}
		\end{subfigure}
		\begin{subfigure}{0.45\textwidth}
			\centering
			\includegraphics[width=\textwidth]{anhang/shape-16.png}
			\caption[Example of used Shape, created by \cite{Quixel}]{Example of used Shape}
		\end{subfigure}
	
		\begin{subfigure}{0.45\textwidth}
			\centering
			\includegraphics[width=\textwidth]{anhang/shape-17.png}
			\caption[Example of used Shape, created by \cite{Quixel}]{Example of used Shape}
		\end{subfigure}
		\begin{subfigure}{0.45\textwidth}
			\centering
			\includegraphics[width=\textwidth]{anhang/shape-18.png}
			\caption[Example of used Shape, created by \cite{Quixel}]{Example of used Shape}
		\end{subfigure}
		
		\begin{subfigure}{0.45\textwidth}
			\centering
			\includegraphics[width=\textwidth]{anhang/shape-19.png}
			\caption[Example of used Shape, created by \cite{Quixel}]{Example of used Shape}
		\end{subfigure}
		\begin{subfigure}{0.45\textwidth}
			\centering
			\includegraphics[width=\textwidth]{anhang/shape-20.png}
			\caption[Example of used Shape, created by \cite{Quixel}]{Example of used Shape}
		\end{subfigure}
	
	\end{figure}

\chapter{Materials for Training}
\label{appendix:materials-for-training}

	The amount of unique materials are one essential part of this work, thus it is important to see some examples of the used materials.\\
	Here are 20 different materials, used in the data generation.
	
	\begin{figure}[h]
		\centering
		\begin{subfigure}{0.45\textwidth}
			\centering
			\includegraphics[width=\textwidth]{anhang/material-1.png}
			\caption[Example of used Material, created by \cite{Quixel}]{Example of used Material}
		\end{subfigure}
		\begin{subfigure}{0.45\textwidth}
			\centering
			\includegraphics[width=\textwidth]{anhang/material-2.png}
			\caption[Example of used Material, created by \cite{Quixel}]{Example of used Material}
		\end{subfigure}
		
		\begin{subfigure}{0.45\textwidth}
			\centering
			\includegraphics[width=\textwidth]{anhang/material-3.png}
			\caption[Example of used Material, created by \cite{Quixel}]{Example of used Material}
		\end{subfigure}
		\begin{subfigure}{0.45\textwidth}
			\centering
			\includegraphics[width=\textwidth]{anhang/material-4.png}
			\caption[Example of used Material, created by \cite{Quixel}]{Example of used Material}
		\end{subfigure}
		
		\begin{subfigure}{0.45\textwidth}
			\centering
			\includegraphics[width=\textwidth]{anhang/material-5.png}
			\caption[Example of used Material, created by \cite{Quixel}]{Example of used Material}
		\end{subfigure}
		\begin{subfigure}{0.45\textwidth}
			\centering
			\includegraphics[width=\textwidth]{anhang/material-6.png}
			\caption[Example of used Material, created by \cite{Quixel}]{Example of used Material}
		\end{subfigure}
	\end{figure}
	
	
	
	\begin{figure}[h]
		\centering
		\begin{subfigure}{0.45\textwidth}
			\centering
			\includegraphics[width=\textwidth]{anhang/material-7.png}
			\caption[Example of used Material, created by \cite{Quixel}]{Example of used Material}
		\end{subfigure}
		\begin{subfigure}{0.45\textwidth}
			\centering
			\includegraphics[width=\textwidth]{anhang/material-8.png}
			\caption[Example of used Material, created by \cite{Quixel}]{Example of used Material}
		\end{subfigure}
		
		\begin{subfigure}{0.45\textwidth}
			\centering
			\includegraphics[width=\textwidth]{anhang/material-9.png}
			\caption[Example of used Material, created by \cite{Quixel}]{Example of used Material}
		\end{subfigure}
		\begin{subfigure}{0.45\textwidth}
			\centering
			\includegraphics[width=\textwidth]{anhang/material-10.png}
			\caption[Example of used Material, created by \cite{Quixel}]{Example of used Material}
		\end{subfigure}
		
		\begin{subfigure}{0.45\textwidth}
			\centering
			\includegraphics[width=\textwidth]{anhang/material-11.png}
			\caption[Example of used Material, created by \cite{Quixel}]{Example of used Material}
		\end{subfigure}
		\begin{subfigure}{0.45\textwidth}
			\centering
			\includegraphics[width=\textwidth]{anhang/material-12.png}
			\caption[Example of used Material, created by \cite{Quixel}]{Example of used Material}
		\end{subfigure}
		
		\begin{subfigure}{0.45\textwidth}
			\centering
			\includegraphics[width=\textwidth]{anhang/material-13.png}
			\caption[Example of used Material, created by \cite{Quixel}]{Example of used Material}
		\end{subfigure}
		\begin{subfigure}{0.45\textwidth}
			\centering
			\includegraphics[width=\textwidth]{anhang/material-14.png}
			\caption[Example of used Material, created by \cite{Quixel}]{Example of used Material}
		\end{subfigure}
	\end{figure}
	
	
	
	\begin{figure}[h]
		\centering
		\begin{subfigure}{0.45\textwidth}
			\centering
			\includegraphics[width=\textwidth]{anhang/material-15.png}
			\caption[Example of used Material, created by \cite{Quixel}]{Example of used Material}
		\end{subfigure}
		\begin{subfigure}{0.45\textwidth}
			\centering
			\includegraphics[width=\textwidth]{anhang/material-16.png}
			\caption[Example of used Material, created by \cite{Quixel}]{Example of used Material}
		\end{subfigure}
		
		\begin{subfigure}{0.45\textwidth}
			\centering
			\includegraphics[width=\textwidth]{anhang/material-17.png}
			\caption[Example of used Material, created by \cite{Quixel}]{Example of used Material}
		\end{subfigure}
		\begin{subfigure}{0.45\textwidth}
			\centering
			\includegraphics[width=\textwidth]{anhang/material-18.png}
			\caption[Example of used Material, created by \cite{Quixel}]{Example of used Material}
		\end{subfigure}
		
		\begin{subfigure}{0.45\textwidth}
			\centering
			\includegraphics[width=\textwidth]{anhang/material-19.png}
			\caption[Example of used Material, created by \cite{Quixel}]{Example of used Material}
		\end{subfigure}
		\begin{subfigure}{0.45\textwidth}
			\centering
			\includegraphics[width=\textwidth]{anhang/material-20.png}
			\caption[Example of used Material, created by \cite{Quixel}]{Example of used Material}
		\end{subfigure}
	\end{figure}

\chapter{Traindata Examples}
\label{appendix:traindata-examples}

	It follows five examples per every dataset which this work proposes. Every dataset were generated in Unreal Engine 5 \cite{ue5} and on a NVIDIA RTX A4500.\\
	\\
	Five example images for the synthetic dataset with \textbf{10 shapes} and \textbf{10 materials}:
	\begin{figure}[H]
		\centering
		\includegraphics[width=\textwidth]{anhang/example-10-10-1.png}
		\caption[An examle traindata. RGB, depth and the ground truth]{An examle traindata. RGB, depth and the ground truth}
	\end{figure}
	\begin{figure}[H]
		\centering
		\includegraphics[width=\textwidth]{anhang/example-10-10-2.png}
		\caption[An examle traindata. RGB, depth and the ground truth]{An examle traindata. RGB, depth and the ground truth}
	\end{figure}
	\begin{figure}[H]
		\centering
		\includegraphics[width=\textwidth]{anhang/example-10-10-3.png}
		\caption[An examle traindata. RGB, depth and the ground truth]{An examle traindata. RGB, depth and the ground truth}
	\end{figure}
	\begin{figure}[H]
		\centering
		\includegraphics[width=\textwidth]{anhang/example-10-10-4.png}
		\caption[An examle traindata. RGB, depth and the ground truth]{An examle traindata. RGB, depth and the ground truth}
	\end{figure}
	\begin{figure}[H]
		\centering
		\includegraphics[width=\textwidth]{anhang/example-10-10-5.png}
		\caption[An examle traindata. RGB, depth and the ground truth]{An examle traindata. RGB, depth and the ground truth}
	\end{figure}
	
	\FloatBarrier
	% \vspace{3cm}
	\clearpage
	Five example images for the synthetic dataset with \textbf{10 shapes} and \textbf{80 materials}:
	\begin{figure}[H]
		\centering
		\includegraphics[width=\textwidth]{anhang/example-10-80-1.png}
		\caption[An examle traindata. RGB, depth and the ground truth]{An examle traindata. RGB, depth and the ground truth}
	\end{figure}
	\begin{figure}[H]
		\centering
		\includegraphics[width=\textwidth]{anhang/example-10-80-2.png}
		\caption[An examle traindata. RGB, depth and the ground truth]{An examle traindata. RGB, depth and the ground truth}
	\end{figure}
	\begin{figure}[H]
		\centering
		\includegraphics[width=\textwidth]{anhang/example-10-80-3.png}
		\caption[An examle traindata. RGB, depth and the ground truth]{An examle traindata. RGB, depth and the ground truth}
	\end{figure}
	\begin{figure}[H]
		\centering
		\includegraphics[width=\textwidth]{anhang/example-10-80-4.png}
		\caption[An examle traindata. RGB, depth and the ground truth]{An examle traindata. RGB, depth and the ground truth}
	\end{figure}
	\begin{figure}[H]
		\centering
		\includegraphics[width=\textwidth]{anhang/example-10-80-5.png}
		\caption[An examle traindata. RGB, depth and the ground truth]{An examle traindata. RGB, depth and the ground truth}
	\end{figure}
	
	\FloatBarrier
	\clearpage
	Five example images for the synthetic dataset with \textbf{10 shapes} and \textbf{160 materials}:
	\begin{figure}[H]
		\centering
		\includegraphics[width=\textwidth]{anhang/example-10-160-1.png}
		\caption[An examle traindata. RGB, depth and the ground truth]{An examle traindata. RGB, depth and the ground truth}
	\end{figure}
	\begin{figure}[H]
		\centering
		\includegraphics[width=\textwidth]{anhang/example-10-160-2.png}
		\caption[An examle traindata. RGB, depth and the ground truth]{An examle traindata. RGB, depth and the ground truth}
	\end{figure}
	\begin{figure}[H]
		\centering
		\includegraphics[width=\textwidth]{anhang/example-10-160-3.png}
		\caption[An examle traindata. RGB, depth and the ground truth]{An examle traindata. RGB, depth and the ground truth}
	\end{figure}
	\begin{figure}[H]
		\centering
		\includegraphics[width=\textwidth]{anhang/example-10-160-4.png}
		\caption[An examle traindata. RGB, depth and the ground truth]{An examle traindata. RGB, depth and the ground truth}
	\end{figure}
	\begin{figure}[H]
		\centering
		\includegraphics[width=\textwidth]{anhang/example-10-160-5.png}
		\caption[An examle traindata. RGB, depth and the ground truth]{An examle traindata. RGB, depth and the ground truth}
	\end{figure}
	
	\FloatBarrier
	\clearpage
	Five example images for the synthetic dataset with \textbf{80 shapes} and \textbf{10 materials}:
	\begin{figure}[H]
		\centering
		\includegraphics[width=\textwidth]{anhang/example-80-10-1.png}
		\caption[An examle traindata. RGB, depth and the ground truth]{An examle traindata. RGB, depth and the ground truth}
	\end{figure}
	\begin{figure}[H]
		\centering
		\includegraphics[width=\textwidth]{anhang/example-80-10-2.png}
		\caption[An examle traindata. RGB, depth and the ground truth]{An examle traindata. RGB, depth and the ground truth}
	\end{figure}
	\begin{figure}[H]
		\centering
		\includegraphics[width=\textwidth]{anhang/example-80-10-3.png}
		\caption[An examle traindata. RGB, depth and the ground truth]{An examle traindata. RGB, depth and the ground truth}
	\end{figure}
	\begin{figure}[H]
		\centering
		\includegraphics[width=\textwidth]{anhang/example-80-10-4.png}
		\caption[An examle traindata. RGB, depth and the ground truth]{An examle traindata. RGB, depth and the ground truth}
	\end{figure}
	\begin{figure}[H]
		\centering
		\includegraphics[width=\textwidth]{anhang/example-80-10-5.png}
		\caption[An examle traindata. RGB, depth and the ground truth]{An examle traindata. RGB, depth and the ground truth}
	\end{figure}
	
	\FloatBarrier
	\clearpage
	Five example images for the synthetic dataset with \textbf{80 shapes} and \textbf{80 materials}:
	\begin{figure}[H]
		\centering
		\includegraphics[width=\textwidth]{anhang/example-80-80-1.png}
		\caption[An examle traindata. RGB, depth and the ground truth]{An examle traindata. RGB, depth and the ground truth}
	\end{figure}
	\begin{figure}[H]
		\centering
		\includegraphics[width=\textwidth]{anhang/example-80-80-2.png}
		\caption[An examle traindata. RGB, depth and the ground truth]{An examle traindata. RGB, depth and the ground truth}
	\end{figure}
	\begin{figure}[H]
		\centering
		\includegraphics[width=\textwidth]{anhang/example-80-80-3.png}
		\caption[An examle traindata. RGB, depth and the ground truth]{An examle traindata. RGB, depth and the ground truth}
	\end{figure}
	\begin{figure}[H]
		\centering
		\includegraphics[width=\textwidth]{anhang/example-80-80-4.png}
		\caption[An examle traindata. RGB, depth and the ground truth]{An examle traindata. RGB, depth and the ground truth}
	\end{figure}
	\begin{figure}[H]
		\centering
		\includegraphics[width=\textwidth]{anhang/example-80-80-5.png}
		\caption[An examle traindata. RGB, depth and the ground truth]{An examle traindata. RGB, depth and the ground truth}
	\end{figure}
	
	\FloatBarrier
	\clearpage
	Five example images for the synthetic dataset with \textbf{80 shapes} and \textbf{160 materials}:
	\begin{figure}[H]
		\centering
		\includegraphics[width=\textwidth]{anhang/example-80-160-1.png}
		\caption[An examle traindata. RGB, depth and the ground truth]{An examle traindata. RGB, depth and the ground truth}
	\end{figure}
	\begin{figure}[H]
		\centering
		\includegraphics[width=\textwidth]{anhang/example-80-160-2.png}
		\caption[An examle traindata. RGB, depth and the ground truth]{An examle traindata. RGB, depth and the ground truth}
	\end{figure}
	\begin{figure}[H]
		\centering
		\includegraphics[width=\textwidth]{anhang/example-80-160-3.png}
		\caption[An examle traindata. RGB, depth and the ground truth]{An examle traindata. RGB, depth and the ground truth}
	\end{figure}
	\begin{figure}[H]
		\centering
		\includegraphics[width=\textwidth]{anhang/example-80-160-4.png}
		\caption[An examle traindata. RGB, depth and the ground truth]{An examle traindata. RGB, depth and the ground truth}
	\end{figure}
	\begin{figure}[H]
		\centering
		\includegraphics[width=\textwidth]{anhang/example-80-160-5.png}
		\caption[An examle traindata. RGB, depth and the ground truth]{An examle traindata. RGB, depth and the ground truth}
	\end{figure}
	
	\FloatBarrier
	\clearpage
	Five example images for the synthetic dataset with \textbf{160 shapes} and \textbf{10 materials}:
	\begin{figure}[H]
		\centering
		\includegraphics[width=\textwidth]{anhang/example-160-10-1.png}
		\caption[An examle traindata. RGB, depth and the ground truth]{An examle traindata. RGB, depth and the ground truth}
	\end{figure}
	\begin{figure}[H]
		\centering
		\includegraphics[width=\textwidth]{anhang/example-160-10-2.png}
		\caption[An examle traindata. RGB, depth and the ground truth]{An examle traindata. RGB, depth and the ground truth}
	\end{figure}
	\begin{figure}[H]
		\centering
		\includegraphics[width=\textwidth]{anhang/example-160-10-3.png}
		\caption[An examle traindata. RGB, depth and the ground truth]{An examle traindata. RGB, depth and the ground truth}
	\end{figure}
	\begin{figure}[H]
		\centering
		\includegraphics[width=\textwidth]{anhang/example-160-10-4.png}
		\caption[An examle traindata. RGB, depth and the ground truth]{An examle traindata. RGB, depth and the ground truth}
	\end{figure}
	\begin{figure}[H]
		\centering
		\includegraphics[width=\textwidth]{anhang/example-160-10-5.png}
		\caption[An examle traindata. RGB, depth and the ground truth]{An examle traindata. RGB, depth and the ground truth}
	\end{figure}
	
	\FloatBarrier
	\clearpage
	Five example images for the synthetic dataset with \textbf{160 shapes} and \textbf{80 materials}:
	\begin{figure}[H]
		\centering
		\includegraphics[width=\textwidth]{anhang/example-160-80-1.png}
		\caption[An examle traindata. RGB, depth and the ground truth]{An examle traindata. RGB, depth and the ground truth}
	\end{figure}
	\begin{figure}[H]
		\centering
		\includegraphics[width=\textwidth]{anhang/example-160-80-2.png}
		\caption[An examle traindata. RGB, depth and the ground truth]{An examle traindata. RGB, depth and the ground truth}
	\end{figure}
	\begin{figure}[H]
		\centering
		\includegraphics[width=\textwidth]{anhang/example-160-80-3.png}
		\caption[An examle traindata. RGB, depth and the ground truth]{An examle traindata. RGB, depth and the ground truth}
	\end{figure}
	\begin{figure}[H]
		\centering
		\includegraphics[width=\textwidth]{anhang/example-160-80-4.png}
		\caption[An examle traindata. RGB, depth and the ground truth]{An examle traindata. RGB, depth and the ground truth}
	\end{figure}
	\begin{figure}[H]
		\centering
		\includegraphics[width=\textwidth]{anhang/example-160-80-5.png}
		\caption[An examle traindata. RGB, depth and the ground truth]{An examle traindata. RGB, depth and the ground truth}
	\end{figure}
	
	\FloatBarrier
	\clearpage
	Five example images for the synthetic dataset with \textbf{160 shapes} and \textbf{160 materials}:
	\begin{figure}[H]
		\centering
		\includegraphics[width=\textwidth]{anhang/example-160-160-1.png}
		\caption[An examle traindata. RGB, depth and the ground truth]{An examle traindata. RGB, depth and the ground truth}
	\end{figure}
	\begin{figure}[H]
		\centering
		\includegraphics[width=\textwidth]{anhang/example-160-160-2.png}
		\caption[An examle traindata. RGB, depth and the ground truth]{An examle traindata. RGB, depth and the ground truth}
	\end{figure}
	\begin{figure}[H]
		\centering
		\includegraphics[width=\textwidth]{anhang/example-160-160-3.png}
		\caption[An examle traindata. RGB, depth and the ground truth]{An examle traindata. RGB, depth and the ground truth}
	\end{figure}
	\begin{figure}[H]
		\centering
		\includegraphics[width=\textwidth]{anhang/example-160-160-4.png}
		\caption[An examle traindata. RGB, depth and the ground truth]{An examle traindata. RGB, depth and the ground truth}
	\end{figure}
	\begin{figure}[H]
		\centering
		\includegraphics[width=\textwidth]{anhang/example-160-160-5.png}
		\caption[An examle traindata. RGB, depth and the ground truth]{An examle traindata. RGB, depth and the ground truth}
	\end{figure}


\chapter{Shape vs. Texture Attention Test Data}
\label{appendix:testdata-examples-bias}

	This appendix presents samples of the handcrafted test images for the Shape vs. Texture Attention Test \ref{sec:shape-texutre-attention-test}.\\
	The test data is separated into images with multiple textures and only one texture for all objects. There is an equal distribution of images using known textures and known shapes, known textures and unknown shapes, unknown textures and known shapes, and unknown textures and unknown shapes.\\
	Starting with \textbf{multiple} textures on every object.
	
	\begin{figure}[H]
		\centering
		\includegraphics[width=0.9\textwidth]{anhang/bias/multiple_textures_known_known_1.png}
		\caption[An examle testdata for Shape vs. Texture Attention Test with multiple textures. RGB and depth images]{An examle testdata for Shape vs. Texture Attention Test with multiple textures. RGB and depth images}
	\end{figure}
	\begin{figure}[H]
		\centering
		\includegraphics[width=0.9\textwidth]{anhang/bias/multiple_textures_known_known_2.png}
		\caption[An examle testdata for Shape vs. Texture Attention Test with multiple textures. RGB and depth images]{An examle testdata for Shape vs. Texture Attention Test with multiple textures. RGB and depth images}
	\end{figure}
	\begin{figure}[H]
		\centering
		\includegraphics[width=0.9\textwidth]{anhang/bias/multiple_textures_known_unknown_1.png}
		\caption[An examle testdata for Shape vs. Texture Attention Test with multiple textures. RGB and depth images]{An examle testdata for Shape vs. Texture Attention Test with multiple textures. RGB and depth images}
	\end{figure}
	\begin{figure}[H]
		\centering
		\includegraphics[width=0.9\textwidth]{anhang/bias/multiple_textures_known_unknown_2.png}
		\caption[An examle testdata for Shape vs. Texture Attention Test with multiple textures. RGB and depth images]{An examle testdata for Shape vs. Texture Attention Test with multiple textures. RGB and depth images}
	\end{figure}
	\begin{figure}[H]
		\centering
		\includegraphics[width=0.9\textwidth]{anhang/bias/multiple_textures_unknown_known_1.png}
		\caption[An examle testdata for Shape vs. Texture Attention Test with multiple textures. RGB and depth images]{An examle testdata for Shape vs. Texture Attention Test with multiple textures. RGB and depth images}
	\end{figure}
	\begin{figure}[H]
		\centering
		\includegraphics[width=0.9\textwidth]{anhang/bias/multiple_textures_unknown_unknown_1.png}
		\caption[An examle testdata for Shape vs. Texture Attention Test with multiple textures. RGB and depth images]{An examle testdata for Shape vs. Texture Attention Test with multiple textures. RGB and depth images}
	\end{figure}
	
	\iffalse
	\begin{figure}[H]
		\centering
		\includegraphics[width=0.9\textwidth]{anhang/bias/multiple_textures_known_known_1.png}
		\caption[An examle testdata for Shape vs. Texture Attention Test with multiple textures. RGB and depth images]{An examle testdata for Shape vs. Texture Attention Test with multiple textures. RGB and depth images}
	\end{figure}
	\begin{figure}[H]
		\centering
		\includegraphics[width=0.9\textwidth]{anhang/bias/multiple_textures_known_known_2.png}
		\caption[An examle testdata for Shape vs. Texture Attention Test with multiple textures. RGB and depth images]{An examle testdata for Shape vs. Texture Attention Test with multiple textures. RGB and depth images}
	\end{figure}
	\begin{figure}[H]
		\centering
		\includegraphics[width=0.9\textwidth]{anhang/bias/multiple_textures_known_known_3.png}
		\caption[An examle testdata for Shape vs. Texture Attention Test with multiple textures. RGB and depth images]{An examle testdata for Shape vs. Texture Attention Test with multiple textures. RGB and depth images}
	\end{figure}
	
	\begin{figure}[H]
		\centering
		\includegraphics[width=0.9\textwidth]{anhang/bias/multiple_textures_known_unknown_1.png}
		\caption[An examle testdata for Shape vs. Texture Attention Test with multiple textures. RGB and depth images]{An examle testdata for Shape vs. Texture Attention Test with multiple textures. RGB and depth images}
	\end{figure}
	\begin{figure}[H]
		\centering
		\includegraphics[width=0.9\textwidth]{anhang/bias/multiple_textures_known_unknown_2.png}
		\caption[An examle testdata for Shape vs. Texture Attention Test with multiple textures. RGB and depth images]{An examle testdata for Shape vs. Texture Attention Test with multiple textures. RGB and depth images}
	\end{figure}
	\begin{figure}[H]
		\centering
		\includegraphics[width=0.9\textwidth]{anhang/bias/multiple_textures_known_unknown_3.png}
		\caption[An examle testdata for Shape vs. Texture Attention Test with multiple textures. RGB and depth images]{An examle testdata for Shape vs. Texture Attention Test with multiple textures. RGB and depth images}
	\end{figure}
	
	\begin{figure}[H]
		\centering
		\includegraphics[width=0.9\textwidth]{anhang/bias/multiple_textures_unknown_known_1.png}
		\caption[An examle testdata for Shape vs. Texture Attention Test with multiple textures. RGB and depth images]{An examle testdata for Shape vs. Texture Attention Test with multiple textures. RGB and depth images}
	\end{figure}
	\begin{figure}[H]
		\centering
		\includegraphics[width=0.9\textwidth]{anhang/bias/multiple_textures_unknown_known_2.png}
		\caption[An examle testdata for Shape vs. Texture Attention Test with multiple textures. RGB and depth images]{An examle testdata for Shape vs. Texture Attention Test with multiple textures. RGB and depth images}
	\end{figure}
	\begin{figure}[H]
		\centering
		\includegraphics[width=0.9\textwidth]{anhang/bias/multiple_textures_unknown_known_3.png}
		\caption[An examle testdata for Shape vs. Texture Attention Test with multiple textures. RGB and depth images]{An examle testdata for Shape vs. Texture Attention Test with multiple textures. RGB and depth images}
	\end{figure}
	
	\begin{figure}[H]
		\centering
		\includegraphics[width=0.9\textwidth]{anhang/bias/multiple_textures_unknown_unknown_1.png}
		\caption[An examle testdata for Shape vs. Texture Attention Test with multiple textures. RGB and depth images]{An examle testdata for Shape vs. Texture Attention Test with multiple textures. RGB and depth images}
	\end{figure}
	\begin{figure}[H]
		\centering
		\includegraphics[width=0.9\textwidth]{anhang/bias/multiple_textures_unknown_unknown_2.png}
		\caption[An examle testdata for Shape vs. Texture Attention Test with multiple textures. RGB and depth images]{An examle testdata for Shape vs. Texture Attention Test with multiple textures. RGB and depth images}
	\end{figure}
	\begin{figure}[H]
		\centering
		\includegraphics[width=0.9\textwidth]{anhang/bias/multiple_textures_unknown_unknown_3.png}
		\caption[An examle testdata for Shape vs. Texture Attention Test with multiple textures. RGB and depth images]{An examle testdata for Shape vs. Texture Attention Test with multiple textures. RGB and depth images}
	\end{figure}
	\fi
	

	\FloatBarrier
	% \vspace{3cm}
	\clearpage

	4 images with \textbf{one} equal texture.
	
	\begin{figure}[H]
		\centering
		\includegraphics[width=0.9\textwidth]{anhang/bias/one_textures_known_known_1.png}
		\caption[An examle testdata for Shape vs. Texture Attention Test with one texture. RGB and depth images]{An examle testdata for Shape vs. Texture Attention Test with one texture. RGB and depth images}
	\end{figure}
	\begin{figure}[H]
		\centering
		\includegraphics[width=0.9\textwidth]{anhang/bias/one_textures_known_unknown_1.png}
		\caption[An examle testdata for Shape vs. Texture Attention Test with one texture. RGB and depth images]{An examle testdata for Shape vs. Texture Attention Test with one texture. RGB and depth images}
	\end{figure}
	\begin{figure}[H]
		\centering
		\includegraphics[width=0.9\textwidth]{anhang/bias/one_textures_unknown_known_1.png}
		\caption[An examle testdata for Shape vs. Texture Attention Test with one texture. RGB and depth images]{An examle testdata for Shape vs. Texture Attention Test with one texture. RGB and depth images}
	\end{figure}
	\begin{figure}[H]
		\centering
		\includegraphics[width=0.9\textwidth]{anhang/bias/one_textures_unknown_unknown_1.png}
		\caption[An examle testdata for Shape vs. Texture Attention Test with one texture. RGB and depth images]{An examle testdata for Shape vs. Texture Attention Test with one texture. RGB and depth images}
	\end{figure}
	
	
	\iffalse
	\begin{figure}[H]
		\centering
		\includegraphics[width=0.9\textwidth]{anhang/bias/one_textures_known_known_1.png}
		\caption[An examle testdata for Shape vs. Texture Attention Test with one texture. RGB and depth images]{An examle testdata for Shape vs. Texture Attention Test with one texture. RGB and depth images}
	\end{figure}
	\begin{figure}[H]
		\centering
		\includegraphics[width=0.9\textwidth]{anhang/bias/one_textures_known_known_2.png}
		\caption[An examle testdata for Shape vs. Texture Attention Test with one texture. RGB and depth images]{An examle testdata for Shape vs. Texture Attention Test with one texture. RGB and depth images}
	\end{figure}
	\begin{figure}[H]
		\centering
		\includegraphics[width=0.9\textwidth]{anhang/bias/one_textures_known_known_3.png}
		\caption[An examle testdata for Shape vs. Texture Attention Test with one texture. RGB and depth images]{An examle testdata for Shape vs. Texture Attention Test with one texture. RGB and depth images}
	\end{figure}
	
	\begin{figure}[H]
		\centering
		\includegraphics[width=0.9\textwidth]{anhang/bias/one_textures_known_unknown_1.png}
		\caption[An examle testdata for Shape vs. Texture Attention Test with one texture. RGB and depth images]{An examle testdata for Shape vs. Texture Attention Test with one texture. RGB and depth images}
	\end{figure}
	\begin{figure}[H]
		\centering
		\includegraphics[width=0.9\textwidth]{anhang/bias/one_textures_known_unknown_2.png}
		\caption[An examle testdata for Shape vs. Texture Attention Test with one texture. RGB and depth images]{An examle testdata for Shape vs. Texture Attention Test with one texture. RGB and depth images}
	\end{figure}
	\begin{figure}[H]
		\centering
		\includegraphics[width=0.9\textwidth]{anhang/bias/one_textures_known_unknown_3.png}
		\caption[An examle testdata for Shape vs. Texture Attention Test with one texture. RGB and depth images]{An examle testdata for Shape vs. Texture Attention Test with one texture. RGB and depth images}
	\end{figure}
	
	\begin{figure}[H]
		\centering
		\includegraphics[width=0.9\textwidth]{anhang/bias/one_textures_unknown_known_1.png}
		\caption[An examle testdata for Shape vs. Texture Attention Test with one texture. RGB and depth images]{An examle testdata for Shape vs. Texture Attention Test with one texture. RGB and depth images}
	\end{figure}
	\begin{figure}[H]
		\centering
		\includegraphics[width=0.9\textwidth]{anhang/bias/one_textures_unknown_known_2.png}
		\caption[An examle testdata for Shape vs. Texture Attention Test with one texture. RGB and depth images]{An examle testdata for Shape vs. Texture Attention Test with one texture. RGB and depth images}
	\end{figure}
	\begin{figure}[H]
		\centering
		\includegraphics[width=0.9\textwidth]{anhang/bias/one_textures_unknown_known_3.png}
		\caption[An examle testdata for Shape vs. Texture Attention Test with one texture. RGB and depth images]{An examle testdata for Shape vs. Texture Attention Test with one texture. RGB and depth images}
	\end{figure}
	
	\begin{figure}[H]
		\centering
		\includegraphics[width=0.9\textwidth]{anhang/bias/one_textures_unknown_unknown_1.png}
		\caption[An examle testdata for Shape vs. Texture Attention Test with one texture. RGB and depth images]{An examle testdata for Shape vs. Texture Attention Test with one texture. RGB and depth images}
	\end{figure}
	\begin{figure}[H]
		\centering
		\includegraphics[width=0.9\textwidth]{anhang/bias/one_textures_unknown_unknown_2.png}
		\caption[An examle testdata for Shape vs. Texture Attention Test with one texture. RGB and depth images]{An examle testdata for Shape vs. Texture Attention Test with one texture. RGB and depth images}
	\end{figure}
	\begin{figure}[H]
		\centering
		\includegraphics[width=0.9\textwidth]{anhang/bias/one_textures_unknown_unknown_3.png}
		\caption[An examle testdata for Shape vs. Texture Attention Test with one texture. RGB and depth images]{An examle testdata for Shape vs. Texture Attention Test with one texture. RGB and depth images}
	\end{figure}
	\fi
	
	
	
	
	
	
	
	

\chapter{In-Distribution Performance and Generalization Test Data Examples}
\label{appendix:testdata-examples-in-distribution}

	This appendix shows 5 example data points for every of the 4 test datasets from the In-Distribution Performance and Generalization Test \ref{sec:in-distribution-performance-generalization}.
	
	Five examples from in-distribution testdata \ref{sec:in-distribution-performance-generalization} for \textbf{known shapes} and \textbf{known textures}:
	\begin{figure}[H]
		\centering
		\includegraphics[width=\textwidth]{anhang/test-example-known_known-1.png}
		\caption[An examle testdata for in-distribution performance experiment. RGB, depth and mask images]{An examle testdata for in-distribution performance experiment. RGB, depth and mask images}
	\end{figure}
	\begin{figure}[H]
		\centering
		\includegraphics[width=\textwidth]{anhang/test-example-known_known-2.png}
		\caption[An examle testdata for in-distribution performance experiment. RGB, depth and mask images]{An examle testdata for in-distribution performance experiment. RGB, depth and mask images}
	\end{figure}
	\begin{figure}[H]
		\centering
		\includegraphics[width=\textwidth]{anhang/test-example-known_known-3.png}
		\caption[An examle testdata for in-distribution performance experiment. RGB, depth and mask images]{An examle testdata for in-distribution performance experiment. RGB, depth and mask images}
	\end{figure}
	\begin{figure}[H]
		\centering
		\includegraphics[width=\textwidth]{anhang/test-example-known_known-4.png}
		\caption[An examle testdata for in-distribution performance experiment. RGB, depth and mask images]{An examle testdata for in-distribution performance experiment. RGB, depth and mask images}
	\end{figure}
	\begin{figure}[H]
		\centering
		\includegraphics[width=\textwidth]{anhang/test-example-known_known-5.png}
		\caption[An examle testdata for in-distribution performance experiment. RGB, depth and mask images]{An examle testdata for in-distribution performance experiment. RGB, depth and mask images}
	\end{figure}
	
	
	\FloatBarrier
	% \vspace{3cm}
	\clearpage
	Five examples from in-distribution testdata \ref{sec:in-distribution-performance-generalization} for \textbf{known shapes} and \textbf{unknow textures}:
	\begin{figure}[H]
		\centering
		\includegraphics[width=\textwidth]{anhang/test-example-known_unknown-1.png}
		\caption[An examle testdata for in-distribution performance experiment. RGB, depth and mask images]{An examle testdata for in-distribution performance experiment. RGB, depth and mask images}
	\end{figure}
	\begin{figure}[H]
		\centering
		\includegraphics[width=\textwidth]{anhang/test-example-known_unknown-2.png}
		\caption[An examle testdata for in-distribution performance experiment. RGB, depth and mask images]{An examle testdata for in-distribution performance experiment. RGB, depth and mask images}
	\end{figure}
	\begin{figure}[H]
		\centering
		\includegraphics[width=\textwidth]{anhang/test-example-known_unknown-3.png}
		\caption[An examle testdata for in-distribution performance experiment. RGB, depth and mask images]{An examle testdata for in-distribution performance experiment. RGB, depth and mask images}
	\end{figure}
	\begin{figure}[H]
		\centering
		\includegraphics[width=\textwidth]{anhang/test-example-known_unknown-4.png}
		\caption[An examle testdata for in-distribution performance experiment. RGB, depth and mask images]{An examle testdata for in-distribution performance experiment. RGB, depth and mask images}
	\end{figure}
	\begin{figure}[H]
		\centering
		\includegraphics[width=\textwidth]{anhang/test-example-known_unknown-5.png}
		\caption[An examle testdata for in-distribution performance experiment. RGB, depth and mask images]{An examle testdata for in-distribution performance experiment. RGB, depth and mask images}
	\end{figure}
	
	
	
	\FloatBarrier
	% \vspace{3cm}
	\clearpage
	Five examples from in-distribution testdata \ref{sec:in-distribution-performance-generalization} for \textbf{unknown shapes} and \textbf{known textures}:
	\begin{figure}[H]
		\centering
		\includegraphics[width=\textwidth]{anhang/test-example-unknown_known-1.png}
		\caption[An examle testdata for in-distribution performance experiment. RGB, depth and mask images]{An examle testdata for in-distribution performance experiment. RGB, depth and mask images}
	\end{figure}
	\begin{figure}[H]
		\centering
		\includegraphics[width=\textwidth]{anhang/test-example-unknown_known-2.png}
		\caption[An examle testdata for in-distribution performance experiment. RGB, depth and mask images]{An examle testdata for in-distribution performance experiment. RGB, depth and mask images}
	\end{figure}
	\begin{figure}[H]
		\centering
		\includegraphics[width=\textwidth]{anhang/test-example-unknown_known-3.png}
		\caption[An examle testdata for in-distribution performance experiment. RGB, depth and mask images]{An examle testdata for in-distribution performance experiment. RGB, depth and mask images}
	\end{figure}
	\begin{figure}[H]
		\centering
		\includegraphics[width=\textwidth]{anhang/test-example-unknown_known-4.png}
		\caption[An examle testdata for in-distribution performance experiment. RGB, depth and mask images]{An examle testdata for in-distribution performance experiment. RGB, depth and mask images}
	\end{figure}
	\begin{figure}[H]
		\centering
		\includegraphics[width=\textwidth]{anhang/test-example-unknown_known-5.png}
		\caption[An examle testdata for in-distribution performance experiment. RGB, depth and mask images]{An examle testdata for in-distribution performance experiment. RGB, depth and mask images}
	\end{figure}
	
	
	
	\FloatBarrier
	% \vspace{3cm}
	\clearpage
	Five examples from in-distribution testdata \ref{sec:in-distribution-performance-generalization} for \textbf{unknown shapes} and \textbf{unknown textures}:
	\begin{figure}[H]
		\centering
		\includegraphics[width=\textwidth]{anhang/test-example-unknown_unknown-1.png}
		\caption[An examle testdata for in-distribution performance experiment. RGB, depth and mask images]{An examle testdata for in-distribution performance experiment. RGB, depth and mask images}
	\end{figure}
	\begin{figure}[H]
		\centering
		\includegraphics[width=\textwidth]{anhang/test-example-unknown_unknown-2.png}
		\caption[An examle testdata for in-distribution performance experiment. RGB, depth and mask images]{An examle testdata for in-distribution performance experiment. RGB, depth and mask images}
	\end{figure}
	\begin{figure}[H]
		\centering
		\includegraphics[width=\textwidth]{anhang/test-example-unknown_unknown-3.png}
		\caption[An examle testdata for in-distribution performance experiment. RGB, depth and mask images]{An examle testdata for in-distribution performance experiment. RGB, depth and mask images}
	\end{figure}
	\begin{figure}[H]
		\centering
		\includegraphics[width=\textwidth]{anhang/test-example-unknown_unknown-4.png}
		\caption[An examle testdata for in-distribution performance experiment. RGB, depth and mask images]{An examle testdata for in-distribution performance experiment. RGB, depth and mask images}
	\end{figure}
	\begin{figure}[H]
		\centering
		\includegraphics[width=\textwidth]{anhang/test-example-unknown_unknown-5.png}
		\caption[An examle testdata for in-distribution performance experiment. RGB, depth and mask images]{An examle testdata for in-distribution performance experiment. RGB, depth and mask images}
	\end{figure}
	

	

\chapter{Sim-to-Real}
\label{appendix:testdata-examples-simtoreal}

	It follows five examples of the real-world test data from \cite{Suchi2019}, used to measure the sim-to-real ability in this study.
	
	Five examples from sim-to-real \ref{sec:sim-to-real-performance-test} OCID testdata:
	\begin{figure}[H]
		\centering
		\includegraphics[width=\textwidth]{anhang/test-example-ocid-1.png}
		\caption[An examle testdata for sim-to-real performance experiment. RGB, depth and mask images]{An examle testdata for sim-to-real performance experiment. RGB, depth and mask images}
	\end{figure}
	\begin{figure}[H]
		\centering
		\includegraphics[width=\textwidth]{anhang/test-example-ocid-2.png}
		\caption[An examle testdata for sim-to-real performance experiment. RGB, depth and mask images]{An examle testdata for sim-to-real performance experiment. RGB, depth and mask images}
	\end{figure}
	\begin{figure}[H]
		\centering
		\includegraphics[width=\textwidth]{anhang/test-example-ocid-3.png}
		\caption[An examle testdata for sim-to-real performance experiment. RGB, depth and mask images]{An examle testdata for sim-to-real performance experiment. RGB, depth and mask images}
	\end{figure}
	\begin{figure}[H]
		\centering
		\includegraphics[width=\textwidth]{anhang/test-example-ocid-4.png}
		\caption[An examle testdata for sim-to-real performance experiment. RGB, depth and mask images]{An examle testdata for sim-to-real performance experiment. RGB, depth and mask images}
	\end{figure}
	\begin{figure}[H]
		\centering
		\includegraphics[width=\textwidth]{anhang/test-example-ocid-5.png}
		\caption[An examle testdata for sim-to-real performance experiment. RGB, depth and mask images]{An examle testdata for sim-to-real performance experiment. RGB, depth and mask images}
	\end{figure}
	
	
	\iffalse
	\FloatBarrier
	% \vspace{3cm}
	\clearpage
	Five examples from sim-to-real \ref{sec:sim-to-real-performance-test} WISDOM real testdata:
	\begin{figure}[H]
		\centering
		\includegraphics[width=\textwidth]{anhang/test-example-wisdom-1.png}
		\caption[An examle testdata for sim-to-real performance experiment. RGB, depth and mask images]{An examle testdata for sim-to-real performance experiment. RGB, depth and mask images}
	\end{figure}
	\begin{figure}[H]
		\centering
		\includegraphics[width=\textwidth]{anhang/test-example-wisdom-2.png}
		\caption[An examle testdata for sim-to-real performance experiment. RGB, depth and mask images]{An examle testdata for sim-to-real performance experiment. RGB, depth and mask images}
	\end{figure}
	\begin{figure}[H]
		\centering
		\includegraphics[width=\textwidth]{anhang/test-example-wisdom-3.png}
		\caption[An examle testdata for sim-to-real performance experiment. RGB, depth and mask images]{An examle testdata for sim-to-real performance experiment. RGB, depth and mask images}
	\end{figure}
	\begin{figure}[H]
		\centering
		\includegraphics[width=\textwidth]{anhang/test-example-wisdom-4.png}
		\caption[An examle testdata for sim-to-real performance experiment. RGB, depth and mask images]{An examle testdata for sim-to-real performance experiment. RGB, depth and mask images}
	\end{figure}
	\begin{figure}[H]
		\centering
		\includegraphics[width=\textwidth]{anhang/test-example-wisdom-5.png}
		\caption[An examle testdata for sim-to-real performance experiment. RGB, depth and mask images]{An examle testdata for sim-to-real performance experiment. RGB, depth and mask images}
	\end{figure}
	\fi
	

\input{kapitel/appendix_test_ocid}
\chapter{Inference Insights}
\label{appendix:inference-insights}

	This appendix will show one example inference with outputs from every layer. The visualizations are only samples because there would be too many outputs. At the end of this appendix, there will be the essential code to hook the in-between outputs and how to visualize them.\\
	First presented is the input image.
	
	\begin{figure}[H]
		\centering
		\includegraphics[width=\textwidth]{anhang/insights/Insight_3xM_10000_10_80.png}
		\caption[Insight Inference Input]{Insight Inference Input}
	\end{figure}
	\FloatBarrier
	
	\clearpage
	The first output in Mask R-CNN comes from ResNet. ResNet extracts features from the image; thus, it is called the backbone of Mask R-CNN.
	
	\begin{figure}[H]
		\centering
		\includegraphics[width=\textwidth]{anhang/insights/3xM_10000_10_80_01_resnet_layer1_conv1.jpg}
		\caption[ResNet Layer 1 Convolution 1 Output]{ResNet Layer 1 Convolution 1 Output}
	\end{figure}
	\begin{figure}[H]
		\centering
		\includegraphics[width=\textwidth]{anhang/insights/3xM_10000_10_80_02_resnet_layer1_conv2.jpg}
		\caption[ResNet Layer 1 Convolution 2 Output]{ResNet Layer 1 Convolution 2 Output}
	\end{figure}
	\begin{figure}[H]
		\centering
		\includegraphics[width=\textwidth]{anhang/insights/3xM_10000_10_80_03_resnet_layer1_conv3.jpg}
		\caption[ResNet Layer 1 Convolution 3 Output]{ResNet Layer 1 Convolution 3 Output}
	\end{figure}
	
	
	\begin{figure}[H]
		\centering
		\includegraphics[width=\textwidth]{anhang/insights/3xM_10000_10_80_04_resnet_layer2_conv1.jpg}
		\caption[ResNet Layer 2 Convolution 1 Output]{ResNet Layer 2 Convolution 1 Output}
	\end{figure}
	\begin{figure}[H]
		\centering
		\includegraphics[width=\textwidth]{anhang/insights/3xM_10000_10_80_05_resnet_layer2_conv2.jpg}
		\caption[ResNet Layer 2 Convolution 2 Output]{ResNet Layer 2 Convolution 2 Output}
	\end{figure}
	\begin{figure}[H]
		\centering
		\includegraphics[width=\textwidth]{anhang/insights/3xM_10000_10_80_06_resnet_layer2_conv3.jpg}
		\caption[ResNet Layer 2 Convolution 3 Output]{ResNet Layer 2 Convolution 3 Output}
	\end{figure}
	
	
	\begin{figure}[H]
		\centering
		\includegraphics[width=\textwidth]{anhang/insights/3xM_10000_10_80_07_resnet_layer3_conv1.jpg}
		\caption[ResNet Layer 3 Convolution 1 Output]{ResNet Layer 3 Convolution 1 Output}
	\end{figure}
	\begin{figure}[H]
		\centering
		\includegraphics[width=\textwidth]{anhang/insights/3xM_10000_10_80_08_resnet_layer3_conv2.jpg}
		\caption[ResNet Layer 3 Convolution 2 Output]{ResNet Layer 3 Convolution 2 Output}
	\end{figure}
	\begin{figure}[H]
		\centering
		\includegraphics[width=\textwidth]{anhang/insights/3xM_10000_10_80_09_resnet_layer3_conv3.jpg}
		\caption[ResNet Layer 3 Convolution 3 Output]{ResNet Layer 3 Convolution 3 Output}
	\end{figure}
	
	\begin{figure}[H]
		\centering
		\includegraphics[width=\textwidth]{anhang/insights/3xM_10000_10_80_10_resnet_layer4_conv1.jpg}
		\caption[ResNet Layer 4 Convolution 1 Output]{ResNet Layer 4 Convolution 1 Output}
	\end{figure}
	\begin{figure}[H]
		\centering
		\includegraphics[width=\textwidth]{anhang/insights/3xM_10000_10_80_11_resnet_layer4_conv2.jpg}
		\caption[ResNet Layer 4 Convolution 2 Output]{ResNet Layer 4 Convolution 2 Output}
	\end{figure}
	\begin{figure}[H]
		\centering
		\includegraphics[width=\textwidth]{anhang/insights/3xM_10000_10_80_12_resnet_layer4_conv3.jpg}
		\caption[ResNet Layer 4 Convolution 3 Output]{ResNet Layer 4 Convolution 3 Output}
	\end{figure}
	
	
	\FloatBarrier
	\clearpage
	
	After the ResNet comes the \ac{fpn}. The created feature maps from ResNet will be used on different levels to extract extra output. First, it takes the outputs from ResNet, extracts feature maps from the feature maps, and creates different feature maps on different resolution levels called lateral layers.
	
	\begin{figure}[H]
		\centering
		\includegraphics[width=\textwidth]{anhang/insights/3xM_10000_10_80_19_fpn_lateral_layer4.jpg}
		\caption[\ac{fpn} lateral layer of ResNet Layer 4]{\ac{fpn} lateral layer of ResNet Layer 4}
	\end{figure}
	
	\begin{figure}[H]
		\centering
		\includegraphics[width=\textwidth]{anhang/insights/3xM_10000_10_80_21_fpn_lateral_layer3.jpg}
		\caption[\ac{fpn} lateral layer of ResNet Layer 3]{\ac{fpn} lateral layer of ResNet Layer 3}
	\end{figure}
	
	\begin{figure}[H]
		\centering
		\includegraphics[width=\textwidth]{anhang/insights/3xM_10000_10_80_23_fpn_lateral_layer2.jpg}
		\caption[\ac{fpn} lateral layer of ResNet Layer 2]{\ac{fpn} lateral layer of ResNet Layer 2}
	\end{figure}
	
	\begin{figure}[H]
		\centering
		\includegraphics[width=\textwidth]{anhang/insights/3xM_10000_10_80_25_fpn_lateral_layer1.jpg}
		\caption[\ac{fpn} lateral layer of ResNet Layer 1]{\ac{fpn} lateral layer of ResNet Layer 1}
	\end{figure}
	
	\FloatBarrier
	\clearpage
	The \ac{fpn} uses these feature maps to create the final feature maps on different resolution levels and combines features from different levels.
	
	\begin{figure}[H]
		\centering
		\includegraphics[width=\textwidth]{anhang/insights/3xM_10000_10_80_20_fpn_output_layer4.jpg}
		\caption[\ac{fpn} output layer of ResNet Layer 4]{\ac{fpn} output layer of ResNet Layer 4}
	\end{figure}
	
	\begin{figure}[H]
		\centering
		\includegraphics[width=\textwidth]{anhang/insights/3xM_10000_10_80_22_fpn_output_layer3.jpg}
		\caption[\ac{fpn} output layer of ResNet Layer 3]{\ac{fpn} output layer of ResNet Layer 3}
	\end{figure}
	
	\begin{figure}[H]
		\centering
		\includegraphics[width=\textwidth]{anhang/insights/3xM_10000_10_80_24_fpn_output_layer2.jpg}
		\caption[\ac{fpn} output layer of ResNet Layer 2]{\ac{fpn} output layer of ResNet Layer 2}
	\end{figure}
	
	\begin{figure}[H]
		\centering
		\includegraphics[width=\textwidth]{anhang/insights/3xM_10000_10_80_26_fpn_output_layer1.jpg}
		\caption[\ac{fpn} output layer of ResNet Layer 1]{\ac{fpn} output layer of ResNet Layer 1}
	\end{figure}
	
	
	\FloatBarrier
	
	\clearpage
	Afterward, creating feature maps, the \ac{rpn} localizes object proposals and uses \ac{nms} to remove redundant proposals.\\
	The \ac{rpn} head first uses the feature maps to prepare for object detection and creating proposals, areas within potential objects.\\
	The \ac{rpn} object classification predicts whether a proposal contains an object.\\
	The \ac{rpn} bounding box predictor defines a bounding box for every proposal. The outputs are \ac{roi}s.
	
	\begin{figure}[H]
		\centering
		\includegraphics[width=0.9\textwidth]{anhang/insights/3xM_10000_10_80_27_rpn_head_conv.jpg}
		\caption[\ac{rpn} head convolution. One example Proposal.]{\ac{rpn} head convolution. One example Proposal.}
	\end{figure}
	\begin{figure}[H]
		\centering
		\includegraphics[width=0.9\textwidth]{anhang/insights/3xM_10000_10_80_28_rpn_cls_logits.jpg}
		\caption[\ac{rpn} object classification for one example Proposal]{\ac{rpn} object classification for one example Proposal}
	\end{figure}
	\begin{figure}[H]
		\centering
		\includegraphics[width=0.9\textwidth]{anhang/insights/3xM_10000_10_80_29_rpn_bbox_pred.jpg}
		\caption[\ac{rpn} bounding box prediction for one example Proposal]{\ac{rpn} bounding box prediction for one example Proposal}
	\end{figure}
	
	
	\FloatBarrier
	
	It follows the \ac{roi}-align, where the \ac{roi}s from \ac{rpn} are getting resized.
	
	\begin{figure}[H]
		\centering
		\includegraphics[width=\textwidth]{anhang/insights/3xM_10000_10_80_30_roi_box_pool.jpg}
		\caption[\ac{roi}-align for three example \ac{roi}s]{\ac{roi}-align for three example \ac{roi}s}
	\end{figure}
	
	\FloatBarrier
	\clearpage
	At last, the masks get prepared (refined) by the mask head and then get predicted \ac{roi}-wise.
	
	\begin{figure}[H]
		\centering
		\includegraphics[width=\textwidth]{anhang/insights/3xM_10000_10_80_35_mask_head_conv1.jpg}
		\caption[The mask head uses the standardized \ac{roi}s and refines them for mask prediction]{The mask head uses the standardized \ac{roi}s and refines them for mask prediction}
	\end{figure}
	\begin{figure}[H]
		\centering
		\includegraphics[width=\textwidth]{anhang/insights/3xM_10000_10_80_36_mask_predictor_fcn.jpg}
		\caption[The mask predictor predicts the masks for every polished \ac{roi}]{The mask predictor predicts the masks for every polished \ac{roi}}
	\end{figure}
	
	\FloatBarrier
	
	Hint: The bounding box refinement and object classification are also included at the end, but this insight wanted to focus on mask creation.
	
	
	Following code collect and visualize these fascinating insights.

	\begin{lstlisting}[language=Python,caption=Hooking insight informations from Mask R-CNN, label=lst:inference-insight]
DNN_INSIGHTS = {}

def hook_func(module, input, output, name):
		try:
				DNN_INSIGHTS[name] = {
						'output': output.detach().cpu()
				}
		except AttributeError as e:
				print(f"Error: {e} Data: {input}")

def register_maskrcnn_hooks(model):
		################
		# ResNet hooks #
		################
		model.backbone.body.layer1[0].conv1.register_forward_hook(
				lambda m, i, o: hook_func(m, i, o, 'resnet_layer1_conv1')
		)
		
		model.backbone.body.layer1[0].conv2.register_forward_hook(
				lambda m, i, o: hook_func(m, i, o, 'resnet_layer1_conv2')
		)
		
		model.backbone.body.layer1[0].conv3.register_forward_hook(
				lambda m, i, o: hook_func(m, i, o, 'resnet_layer1_conv3')
		)
		
		model.backbone.body.layer2[0].conv1.register_forward_hook(
				lambda m, i, o: hook_func(m, i, o, 'resnet_layer2_conv1')
		)
		
		model.backbone.body.layer2[0].conv2.register_forward_hook(
				lambda m, i, o: hook_func(m, i, o, 'resnet_layer2_conv2')
		)
		
		model.backbone.body.layer2[0].conv3.register_forward_hook(
				lambda m, i, o: hook_func(m, i, o, 'resnet_layer2_conv3')
		)
		
		model.backbone.body.layer3[0].conv1.register_forward_hook(
				lambda m, i, o: hook_func(m, i, o, 'resnet_layer3_conv1')
		)
		
		model.backbone.body.layer3[0].conv2.register_forward_hook(
				lambda m, i, o: hook_func(m, i, o, 'resnet_layer3_conv2')
		)
		
		model.backbone.body.layer3[0].conv3.register_forward_hook(
				lambda m, i, o: hook_func(m, i, o, 'resnet_layer3_conv3')
		)
		
		model.backbone.body.layer4[0].conv1.register_forward_hook(
				lambda m, i, o: hook_func(m, i, o, 'resnet_layer4_conv1')
		)
		
		model.backbone.body.layer4[0].conv2.register_forward_hook(
				lambda m, i, o: hook_func(m, i, o, 'resnet_layer4_conv2')
		)
		
		model.backbone.body.layer4[0].conv3.register_forward_hook(
				lambda m, i, o: hook_func(m, i, o, 'resnet_layer4_conv3')
		)
		
		#######################################
		# Feature Pyramid Network (FPN) hooks #
		#######################################
		model.backbone.fpn.inner_blocks[0].register_forward_hook(
				lambda m, i, o: hook_func(m, i, o, 'fpn_lateral_layer1')
		)
		
		model.backbone.fpn.inner_blocks[1].register_forward_hook(
				lambda m, i, o: hook_func(m, i, o, 'fpn_lateral_layer2')
		)
		
		model.backbone.fpn.inner_blocks[2].register_forward_hook(
				lambda m, i, o: hook_func(m, i, o, 'fpn_lateral_layer3')
		)
		
		model.backbone.fpn.inner_blocks[3].register_forward_hook(
				lambda m, i, o: hook_func(m, i, o, 'fpn_lateral_layer4')
		)
		
		model.backbone.fpn.layer_blocks[0].register_forward_hook(
				lambda m, i, o: hook_func(m, i, o, 'fpn_output_layer1')
		)
		
		model.backbone.fpn.layer_blocks[1].register_forward_hook(
				lambda m, i, o: hook_func(m, i, o, 'fpn_output_layer2')
		)
		
		model.backbone.fpn.layer_blocks[2].register_forward_hook(
				lambda m, i, o: hook_func(m, i, o, 'fpn_output_layer3')
		)
		
		model.backbone.fpn.layer_blocks[3].register_forward_hook(
				lambda m, i, o: hook_func(m, i, o, 'fpn_output_layer4')
		)
		
		
		#######################################
		# Region Proposal Network (RPN) hooks #
		#######################################
		model.rpn.head.conv.register_forward_hook(
				lambda m, i, o: hook_func(m, i, o, 'rpn_head_conv')
		)
		
		model.rpn.head.cls_logits.register_forward_hook(
				lambda m, i, o: hook_func(m, i, o, 'rpn_cls_logits')
		)
		
		model.rpn.head.bbox_pred.register_forward_hook(
				lambda m, i, o: hook_func(m, i, o, 'rpn_bbox_pred')
		)
		
		###################
		# RoI Align hooks #
		###################
		model.roi_heads.box_roi_pool.register_forward_hook(
				lambda m, i, o: hook_func(m, i, o, 'roi_box_pool')
		)
		
		#############
		# Mask Head #
		#############
		model.roi_heads.mask_head[0].register_forward_hook(
				lambda m, i, o: hook_func(m, i, o, 'mask_head_conv1')
		)
		
		model.roi_heads.mask_predictor.mask_fcn_logits.register_forward_hook(
				lambda m, i, o: hook_func(m, i, o, 'mask_predictor_fcn_logits')
		)




def plot_feature_map(tensor, should_save, save_path, should_show, title="Feature Map"):
		if len(tensor.shape) == 4:
				num_cols = min(tensor.size()[0], 3)
				num_rows = min(tensor.size()[1], 1)
				
				if num_rows > num_cols:
						num_rows = num_cols
				
				fig, all_ax = plt.subplots(num_rows, num_cols, figsize=(15, 10))
				
				
				
				
				
				for cur_col in range(num_cols):
						for cur_row in range(num_rows):
								feature_map_index_1 = cur_col
								feature_map_index_2 = cur_row
								
								if "mask_predictor" in title:
								title = title.replace("_logits", "")
								feature_map_index_1 = feature_map_index_1
								feature_map_index_2 = feature_map_index_2+1
								
								# Extract the feature map for the ith sample
								cur_image = tensor[feature_map_index_1, feature_map_index_2].detach().cpu().numpy()  # Select channel 0, detach from graph
								
								if num_cols > 1 and num_rows == 1:
										ax = all_ax[cur_col]
								elif num_cols == 1 and num_rows == 1:
										ax = all_ax
								elif num_cols == 1 and num_rows > 1:
										ax = all_ax[cur_row]
								elif num_cols > 1 and num_rows > 1:
										ax = all_ax[cur_row][cur_col]
								else:
										raise Exception(f"Error during col: {cur_col}, row: {cur_row}")
								
								ax.imshow(cur_image, cmap='viridis')
								ax.set_title(f'{title} {cur_col+1} {cur_row+1}')
								ax.axis('off')
		elif len(tensor.shape) == 2:
				plot_image = tensor.cpu().numpy()
				plt.figure(figsize=(15, 10))
				plt.imshow(plot_image, cmap='viridis')
				plt.title(title)
				plt.axis('off')
		else:
				raise ValueException(f"Tensor with shape: {tensor.shape} can't be plottet.")
		
		
		if should_save:
				plt.savefig(os.path.join(save_path, f"{title}.jpg"))
		
		if should_show:
				plt.show()
		
		plt.axis('off')
		plt.clf()
		plt.close()


def visualize_insights(insights, should_save, save_path, name, should_show, max_aspect_ratio=5.0, max_cols=3, channel_limit=3, batch_limit=1):
		counter = 1
		for layer_name, data in insights.items():
				try:
						plot_feature_map(data['output'], should_save, save_path, should_show, title=f"{name}_{counter:02}_{layer_name}")
				except Exception as e:
						print(f"Error during insight visualization of: {layer_name} with error: {e} and tensor: {data['output'].size()}")
				counter += 1
	\end{lstlisting}
	



\chapter{Wilcoxon Test Example}
\label{appendix:wilcoxon-test-example}
	It follows a toy example to see the process of the Wilcoxon Test.\\
	The Wilcoxon Test tries to reject the null hypothesis. The null hypothesis is that the difference between both sequences can be described with random fluctuations. Here are the steps for trying to reject the null hypothesis.\\
	\\
	\textbf{Dataset 1 (Model A \ac{iou})} and \textbf{Dataset 2 (Model B \ac{iou}):}
	\[
	\mathbf{X} = [0.75, 0.82, 0.78, 0.88, 0.91], \mathbf{Y} = [0.72, 0.84, 0.76, 0.85, 0.89]
	\]
	
	\textbf{Step 1: Calculate Differences}
	\[
	d_i = X_i - Y_i = [0.03, -0.02, 0.02, 0.03, 0.02]
	\]
	
	\textbf{Step 2: Compute Rankings}\\
	Absolute Values for ranking:
	\[
	|d_i| = [0.03, 0.02, 0.02, 0.03, 0.02]
	\]
	Sorted and Ranked:
	\[
	[0.02, 0.02, 0.02, 0.03, 0.03] = [1, 2, 3, 4, 5]
	\]
	Calc mean ranks for the same rank:
	\[
	\text{Mean rank for } 0.02 = \frac{1 + 2 + 3}{3} = 2.0
	\]
	\[
	\text{Mean rank for } 0.03 = \frac{4 + 5}{2} = 4.5
	\]
	Assign ranking results:
	\[
	\text{Ranks} = [4.5, 2.0, 2.0, 4.5, 2.0]
	\]
	
	\textbf{Step 3: Assign Signs to Ranks}
	\[
	\text{Signed Ranks} = [+4.5, -2.0, +2.0, +4.5, +2.0]
	\]
	
	\clearpage
	\textbf{Step 4: Compute Test Statistic}\\
	The sum of positive ranks is:
	\[
	W^+ = 4.5 + 2.0 + 4.5 + 2.0 = 13.0
	\]
	
	The sum of negative ranks is:
	\[
	W^- = 2.0
	\]
	
	The test statistic is the smaller of the two:
	\[
	W = \min(W^+, W^-) = 2.0
	\]
	
	\textbf{Step 5: Determine Significance}\\
	Using \href{https://de.wikipedia.org/wiki/Wilcoxon-Vorzeichen-Rang-Test#Teststatistik}{the Wilcoxon signed-rank table} for \( n = 5 \) and a two-tailed test at \( \alpha = 0.05 \), the critical value is \( W_\text{critical} = 0 \). Since \( W = 2.0 \) is greater than \( W_\text{critical} \), we fail to reject the null hypothesis. So the values does not differ significant from each other and also could be a random fluctuation.\\
	\\
	The described process is the core functionality of the Wilcoxon Test, but it gets a bit more complicated when applied to data with more observations. Then, a standardized test statistic is calculated and used to reject the null hypothesis.



\chapter{Customization Parameters in Unreal Engine 5 Data Generator}
\label{appendix:custom-params-ue5}
	The data generator in Unreal Engine 5 is very flexible and usable for many different cases. It has 24 parameters, which allow customized usage for data generation. The 24 parameters are listed below, each with a short description of its functionality/effect.

	\begin{itemize}
		\item \textbf{Single Shot:} Is a bool value that decides if the data generator should only take a single picture with the objects currently laying in the bin-box or make a standard data generation with object spawning. The single-shot mode allows a custom scene without the features of the data generator.
		\item \textbf{Mesh Table:} Defines the data table of shapes. This adds more flexibility to the usage.
		\item \textbf{Material Table:} Defines the data table of materials. This table allows a quick switch of the materials.
		\item \textbf{Background Table:} A data table of materials for the background and the bin-box.
		\item \textbf{Random Indexing:} Decides if the materials and shapes of the objects should be chosen randomly or in order.
		\item \textbf{Duplicate Meshes:} Whether to allow duplicate shapes. It only matters if the shape data table has duplicate elements.
		\item \textbf{Duplicate Materials:} Whether to allow duplicate materials. It only matters if the material data table has duplicate elements.
		\item  \textbf{Start Image Counter:} Defines the image to start with. This is useful when exiting the data generator and continuing with an already existing dataset.
		\item \textbf{Start Material Amount Index:} Sets the start point in the **Material Amounts** array. This helps skip an already existing dataset.
		\item \textbf{Start Mesh Amount Index:} Sets the start point in the **Mesh Amounts** array. This helps skip an already existing dataset.
		\item \textbf{Model Amounts:} An array of integer values defines the number of shapes. For example, [1, 10] will create two datasets, one using only one shape and another using ten shapes.
		\item \textbf{Material Amounts:} An array of integer values defines the number of materials. For example, [1, 10] will create two datasets, one using only one material and another using ten materials.
		\item \textbf{Model Min Amounts:} An array that defines the start index for the shapes. So, it is possible to define a range of shapes from the mesh table.
		\item \textbf{Material Min Amounts:} An array that defines the start index for the materials. So, it is possible to define a range of materials from the material table.
		\item \textbf{Object Amount MIN:} Defines the minimum amount of objects per scene.
		\item \textbf{Object Amount MAX:} Defines the maximum amount of objects per scene.
		\item \textbf{Data Amount Per Dataset:} Sets the amount of scenes per dataset. 
		\item \textbf{Image Width:} Sets the width of the rendered image.
		\item \textbf{Image Height:} Sets the height of the rendered image.
		\item \textbf{Use Dual Dir Format:} This boolean decides whether to create only three folders and every image have a unique name or to create a folder for every scene.
		\item \textbf{Time Limit:} Defines a time limit in seconds before the scene gets frozen and a picture is taken.
		\item \textbf{Max Size:} Defines the maximum size of an object.
		\item \textbf{Min Size:} Defines the minimum size of an object.
		\item \textbf{Data Save Path:} Describes the path where the datasets should get created as a string.
	\end{itemize}

\end{document}
