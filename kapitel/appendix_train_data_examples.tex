\chapter{Traindata Examples}
\label{appendix:traindata-examples}

	It follows five examples per every dataset which this work proposes. Every dataset were generated in Unreal Engine 5 \cite{ue5} and on a NVIDIA RTX A4500.\\
	\\
	Five example images for the synthetic dataset with \textbf{10 shapes} and \textbf{10 materials}:
	\begin{figure}[H]
		\centering
		\includegraphics[width=\textwidth]{anhang/example-10-10-1.png}
		\caption[An examle traindata. RGB, depth and the ground truth]{An examle traindata. RGB, depth and the ground truth}
	\end{figure}
	\begin{figure}[H]
		\centering
		\includegraphics[width=\textwidth]{anhang/example-10-10-2.png}
		\caption[An examle traindata. RGB, depth and the ground truth]{An examle traindata. RGB, depth and the ground truth}
	\end{figure}
	\begin{figure}[H]
		\centering
		\includegraphics[width=\textwidth]{anhang/example-10-10-3.png}
		\caption[An examle traindata. RGB, depth and the ground truth]{An examle traindata. RGB, depth and the ground truth}
	\end{figure}
	\begin{figure}[H]
		\centering
		\includegraphics[width=\textwidth]{anhang/example-10-10-4.png}
		\caption[An examle traindata. RGB, depth and the ground truth]{An examle traindata. RGB, depth and the ground truth}
	\end{figure}
	\begin{figure}[H]
		\centering
		\includegraphics[width=\textwidth]{anhang/example-10-10-5.png}
		\caption[An examle traindata. RGB, depth and the ground truth]{An examle traindata. RGB, depth and the ground truth}
	\end{figure}
	
	\FloatBarrier
	% \vspace{3cm}
	\clearpage
	Five example images for the synthetic dataset with \textbf{10 shapes} and \textbf{80 materials}:
	\begin{figure}[H]
		\centering
		\includegraphics[width=\textwidth]{anhang/example-10-80-1.png}
		\caption[An examle traindata. RGB, depth and the ground truth]{An examle traindata. RGB, depth and the ground truth}
	\end{figure}
	\begin{figure}[H]
		\centering
		\includegraphics[width=\textwidth]{anhang/example-10-80-2.png}
		\caption[An examle traindata. RGB, depth and the ground truth]{An examle traindata. RGB, depth and the ground truth}
	\end{figure}
	\begin{figure}[H]
		\centering
		\includegraphics[width=\textwidth]{anhang/example-10-80-3.png}
		\caption[An examle traindata. RGB, depth and the ground truth]{An examle traindata. RGB, depth and the ground truth}
	\end{figure}
	\begin{figure}[H]
		\centering
		\includegraphics[width=\textwidth]{anhang/example-10-80-4.png}
		\caption[An examle traindata. RGB, depth and the ground truth]{An examle traindata. RGB, depth and the ground truth}
	\end{figure}
	\begin{figure}[H]
		\centering
		\includegraphics[width=\textwidth]{anhang/example-10-80-5.png}
		\caption[An examle traindata. RGB, depth and the ground truth]{An examle traindata. RGB, depth and the ground truth}
	\end{figure}
	
	\FloatBarrier
	\clearpage
	Five example images for the synthetic dataset with \textbf{10 shapes} and \textbf{160 materials}:
	\begin{figure}[H]
		\centering
		\includegraphics[width=\textwidth]{anhang/example-10-160-1.png}
		\caption[An examle traindata. RGB, depth and the ground truth]{An examle traindata. RGB, depth and the ground truth}
	\end{figure}
	\begin{figure}[H]
		\centering
		\includegraphics[width=\textwidth]{anhang/example-10-160-2.png}
		\caption[An examle traindata. RGB, depth and the ground truth]{An examle traindata. RGB, depth and the ground truth}
	\end{figure}
	\begin{figure}[H]
		\centering
		\includegraphics[width=\textwidth]{anhang/example-10-160-3.png}
		\caption[An examle traindata. RGB, depth and the ground truth]{An examle traindata. RGB, depth and the ground truth}
	\end{figure}
	\begin{figure}[H]
		\centering
		\includegraphics[width=\textwidth]{anhang/example-10-160-4.png}
		\caption[An examle traindata. RGB, depth and the ground truth]{An examle traindata. RGB, depth and the ground truth}
	\end{figure}
	\begin{figure}[H]
		\centering
		\includegraphics[width=\textwidth]{anhang/example-10-160-5.png}
		\caption[An examle traindata. RGB, depth and the ground truth]{An examle traindata. RGB, depth and the ground truth}
	\end{figure}
	
	\FloatBarrier
	\clearpage
	Five example images for the synthetic dataset with \textbf{80 shapes} and \textbf{10 materials}:
	\begin{figure}[H]
		\centering
		\includegraphics[width=\textwidth]{anhang/example-80-10-1.png}
		\caption[An examle traindata. RGB, depth and the ground truth]{An examle traindata. RGB, depth and the ground truth}
	\end{figure}
	\begin{figure}[H]
		\centering
		\includegraphics[width=\textwidth]{anhang/example-80-10-2.png}
		\caption[An examle traindata. RGB, depth and the ground truth]{An examle traindata. RGB, depth and the ground truth}
	\end{figure}
	\begin{figure}[H]
		\centering
		\includegraphics[width=\textwidth]{anhang/example-80-10-3.png}
		\caption[An examle traindata. RGB, depth and the ground truth]{An examle traindata. RGB, depth and the ground truth}
	\end{figure}
	\begin{figure}[H]
		\centering
		\includegraphics[width=\textwidth]{anhang/example-80-10-4.png}
		\caption[An examle traindata. RGB, depth and the ground truth]{An examle traindata. RGB, depth and the ground truth}
	\end{figure}
	\begin{figure}[H]
		\centering
		\includegraphics[width=\textwidth]{anhang/example-80-10-5.png}
		\caption[An examle traindata. RGB, depth and the ground truth]{An examle traindata. RGB, depth and the ground truth}
	\end{figure}
	
	\FloatBarrier
	\clearpage
	Five example images for the synthetic dataset with \textbf{80 shapes} and \textbf{80 materials}:
	\begin{figure}[H]
		\centering
		\includegraphics[width=\textwidth]{anhang/example-80-80-1.png}
		\caption[An examle traindata. RGB, depth and the ground truth]{An examle traindata. RGB, depth and the ground truth}
	\end{figure}
	\begin{figure}[H]
		\centering
		\includegraphics[width=\textwidth]{anhang/example-80-80-2.png}
		\caption[An examle traindata. RGB, depth and the ground truth]{An examle traindata. RGB, depth and the ground truth}
	\end{figure}
	\begin{figure}[H]
		\centering
		\includegraphics[width=\textwidth]{anhang/example-80-80-3.png}
		\caption[An examle traindata. RGB, depth and the ground truth]{An examle traindata. RGB, depth and the ground truth}
	\end{figure}
	\begin{figure}[H]
		\centering
		\includegraphics[width=\textwidth]{anhang/example-80-80-4.png}
		\caption[An examle traindata. RGB, depth and the ground truth]{An examle traindata. RGB, depth and the ground truth}
	\end{figure}
	\begin{figure}[H]
		\centering
		\includegraphics[width=\textwidth]{anhang/example-80-80-5.png}
		\caption[An examle traindata. RGB, depth and the ground truth]{An examle traindata. RGB, depth and the ground truth}
	\end{figure}
	
	\FloatBarrier
	\clearpage
	Five example images for the synthetic dataset with \textbf{80 shapes} and \textbf{160 materials}:
	\begin{figure}[H]
		\centering
		\includegraphics[width=\textwidth]{anhang/example-80-160-1.png}
		\caption[An examle traindata. RGB, depth and the ground truth]{An examle traindata. RGB, depth and the ground truth}
	\end{figure}
	\begin{figure}[H]
		\centering
		\includegraphics[width=\textwidth]{anhang/example-80-160-2.png}
		\caption[An examle traindata. RGB, depth and the ground truth]{An examle traindata. RGB, depth and the ground truth}
	\end{figure}
	\begin{figure}[H]
		\centering
		\includegraphics[width=\textwidth]{anhang/example-80-160-3.png}
		\caption[An examle traindata. RGB, depth and the ground truth]{An examle traindata. RGB, depth and the ground truth}
	\end{figure}
	\begin{figure}[H]
		\centering
		\includegraphics[width=\textwidth]{anhang/example-80-160-4.png}
		\caption[An examle traindata. RGB, depth and the ground truth]{An examle traindata. RGB, depth and the ground truth}
	\end{figure}
	\begin{figure}[H]
		\centering
		\includegraphics[width=\textwidth]{anhang/example-80-160-5.png}
		\caption[An examle traindata. RGB, depth and the ground truth]{An examle traindata. RGB, depth and the ground truth}
	\end{figure}
	
	\FloatBarrier
	\clearpage
	Five example images for the synthetic dataset with \textbf{160 shapes} and \textbf{10 materials}:
	\begin{figure}[H]
		\centering
		\includegraphics[width=\textwidth]{anhang/example-160-10-1.png}
		\caption[An examle traindata. RGB, depth and the ground truth]{An examle traindata. RGB, depth and the ground truth}
	\end{figure}
	\begin{figure}[H]
		\centering
		\includegraphics[width=\textwidth]{anhang/example-160-10-2.png}
		\caption[An examle traindata. RGB, depth and the ground truth]{An examle traindata. RGB, depth and the ground truth}
	\end{figure}
	\begin{figure}[H]
		\centering
		\includegraphics[width=\textwidth]{anhang/example-160-10-3.png}
		\caption[An examle traindata. RGB, depth and the ground truth]{An examle traindata. RGB, depth and the ground truth}
	\end{figure}
	\begin{figure}[H]
		\centering
		\includegraphics[width=\textwidth]{anhang/example-160-10-4.png}
		\caption[An examle traindata. RGB, depth and the ground truth]{An examle traindata. RGB, depth and the ground truth}
	\end{figure}
	\begin{figure}[H]
		\centering
		\includegraphics[width=\textwidth]{anhang/example-160-10-5.png}
		\caption[An examle traindata. RGB, depth and the ground truth]{An examle traindata. RGB, depth and the ground truth}
	\end{figure}
	
	\FloatBarrier
	\clearpage
	Five example images for the synthetic dataset with \textbf{160 shapes} and \textbf{80 materials}:
	\begin{figure}[H]
		\centering
		\includegraphics[width=\textwidth]{anhang/example-160-80-1.png}
		\caption[An examle traindata. RGB, depth and the ground truth]{An examle traindata. RGB, depth and the ground truth}
	\end{figure}
	\begin{figure}[H]
		\centering
		\includegraphics[width=\textwidth]{anhang/example-160-80-2.png}
		\caption[An examle traindata. RGB, depth and the ground truth]{An examle traindata. RGB, depth and the ground truth}
	\end{figure}
	\begin{figure}[H]
		\centering
		\includegraphics[width=\textwidth]{anhang/example-160-80-3.png}
		\caption[An examle traindata. RGB, depth and the ground truth]{An examle traindata. RGB, depth and the ground truth}
	\end{figure}
	\begin{figure}[H]
		\centering
		\includegraphics[width=\textwidth]{anhang/example-160-80-4.png}
		\caption[An examle traindata. RGB, depth and the ground truth]{An examle traindata. RGB, depth and the ground truth}
	\end{figure}
	\begin{figure}[H]
		\centering
		\includegraphics[width=\textwidth]{anhang/example-160-80-5.png}
		\caption[An examle traindata. RGB, depth and the ground truth]{An examle traindata. RGB, depth and the ground truth}
	\end{figure}
	
	\FloatBarrier
	\clearpage
	Five example images for the synthetic dataset with \textbf{160 shapes} and \textbf{160 materials}:
	\begin{figure}[H]
		\centering
		\includegraphics[width=\textwidth]{anhang/example-160-160-1.png}
		\caption[An examle traindata. RGB, depth and the ground truth]{An examle traindata. RGB, depth and the ground truth}
	\end{figure}
	\begin{figure}[H]
		\centering
		\includegraphics[width=\textwidth]{anhang/example-160-160-2.png}
		\caption[An examle traindata. RGB, depth and the ground truth]{An examle traindata. RGB, depth and the ground truth}
	\end{figure}
	\begin{figure}[H]
		\centering
		\includegraphics[width=\textwidth]{anhang/example-160-160-3.png}
		\caption[An examle traindata. RGB, depth and the ground truth]{An examle traindata. RGB, depth and the ground truth}
	\end{figure}
	\begin{figure}[H]
		\centering
		\includegraphics[width=\textwidth]{anhang/example-160-160-4.png}
		\caption[An examle traindata. RGB, depth and the ground truth]{An examle traindata. RGB, depth and the ground truth}
	\end{figure}
	\begin{figure}[H]
		\centering
		\includegraphics[width=\textwidth]{anhang/example-160-160-5.png}
		\caption[An examle traindata. RGB, depth and the ground truth]{An examle traindata. RGB, depth and the ground truth}
	\end{figure}

