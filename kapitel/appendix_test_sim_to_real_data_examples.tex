\chapter{Optonic Bin-Picking Dataset Examples}
\label{appendix:testdata-examples-simtoreal}

	This study proposes a real-world dataset for instance-segmentation as described in section \ref{sec:test-data}. Here are samples from this dataset with RGB, depth and (colored) mask. In addition, this appendix shows the python code for capturing the images in listing \ref{lst:file-camera}.
	
	\begin{figure}[H]
		\centering
		\includegraphics[width=\textwidth]{anhang/real/real_data_sample_1.png}
		\caption[An sample observation from the sim-to-real test dataset.]{An sample observation from the sim-to-real test dataset}
	\end{figure}
	\begin{figure}[H]
		\centering
		\includegraphics[width=\textwidth]{anhang/real/real_data_sample_2.png}
		\caption[An sample observation from the sim-to-real test dataset.]{An sample observation from the sim-to-real test dataset}
	\end{figure}
	\begin{figure}[H]
		\centering
		\includegraphics[width=\textwidth]{anhang/real/real_data_sample_3.png}
		\caption[An sample observation from the sim-to-real test dataset.]{An sample observation from the sim-to-real test dataset}
	\end{figure}
	\begin{figure}[H]
		\centering
		\includegraphics[width=\textwidth]{anhang/real/real_data_sample_4.png}
		\caption[An sample observation from the sim-to-real test dataset.]{An sample observation from the sim-to-real test dataset}
	\end{figure}
	\begin{figure}[H]
		\centering
		\includegraphics[width=\textwidth]{anhang/real/real_data_sample_5.png}
		\caption[An sample observation from the sim-to-real test dataset.]{An sample observation from the sim-to-real test dataset}
	\end{figure}
	\begin{figure}[H]
		\centering
		\includegraphics[width=\textwidth]{anhang/real/real_data_sample_6.png}
		\caption[An sample observation from the sim-to-real test dataset.]{An sample observation from the sim-to-real test dataset}
	\end{figure}
	\begin{figure}[H]
		\centering
		\includegraphics[width=\textwidth]{anhang/real/real_data_sample_7.png}
		\caption[An sample observation from the sim-to-real test dataset.]{An sample observation from the sim-to-real test dataset}
	\end{figure}
	\begin{figure}[H]
		\centering
		\includegraphics[width=\textwidth]{anhang/real/real_data_sample_8.png}
		\caption[An sample observation from the sim-to-real test dataset.]{An sample observation from the sim-to-real test dataset}
	\end{figure}
	\begin{figure}[H]
		\centering
		\includegraphics[width=\textwidth]{anhang/real/real_data_sample_9.png}
		\caption[An sample observation from the sim-to-real test dataset.]{An sample observation from the sim-to-real test dataset}
	\end{figure}
	\begin{figure}[H]
		\centering
		\includegraphics[width=\textwidth]{anhang/real/real_data_sample_10.png}
		\caption[An sample observation from the sim-to-real test dataset.]{An sample observation from the sim-to-real test dataset}
	\end{figure}
	\begin{figure}[H]
		\centering
		\includegraphics[width=\textwidth]{anhang/real/real_data_sample_11.png}
		\caption[An sample observation from the sim-to-real test dataset.]{An sample observation from the sim-to-real test dataset}
	\end{figure}
	\begin{figure}[H]
		\centering
		\includegraphics[width=\textwidth]{anhang/real/real_data_sample_12.png}
		\caption[An sample observation from the sim-to-real test dataset.]{An sample observation from the sim-to-real test dataset}
	\end{figure}
	\begin{figure}[H]
		\centering
		\includegraphics[width=\textwidth]{anhang/real/real_data_sample_13.png}
		\caption[An sample observation from the sim-to-real test dataset.]{An sample observation from the sim-to-real test dataset}
	\end{figure}
	\begin{figure}[H]
		\centering
		\includegraphics[width=\textwidth]{anhang/real/real_data_sample_14.png}
		\caption[An sample observation from the sim-to-real test dataset.]{An sample observation from the sim-to-real test dataset}
	\end{figure}
	\begin{figure}[H]
		\centering
		\includegraphics[width=\textwidth]{anhang/real/real_data_sample_15.png}
		\caption[An sample observation from the sim-to-real test dataset.]{An sample observation from the sim-to-real test dataset}
	\end{figure}
	
	\clearpage
	\begin{lstlisting}[language=Python,caption=Capturing images from a file camera with depth image scaling using Ensenso Software from Optonic GmbH \cite{optonic}, label=lst:file-camera]
import os
import json
import zipfile

import numpy as np
import cv2
from scipy.interpolate import griddata

from nxlib import NxLib, Camera, NxLibItem
from nxlib.context import NxLib, StereoCamera
from nxlib.command import NxLibCommand
from nxlib.constants import *

with open("./params.json", "r") as file:
		params = json.loads(file.read())

os.makedirs("./data", exist_ok=True)
os.makedirs("./data/rgb", exist_ok=True)
os.makedirs("./data/depth", exist_ok=True)

def interpolate_nan(image):
		x, y = np.indices(image.shape)
		nan_mask = np.isnan(image)
		
		# Coordinates of valid (non-NaN) points
		valid_coords = np.array((x[~nan_mask], y[~nan_mask])).T
		# Values of valid points
		valid_values = image[~nan_mask]
		# Coordinates of NaN points
		nan_coords = np.array((x[nan_mask], y[nan_mask])).T
		
		# Interpolate NaN points
		interpolated_values = griddata(valid_coords, valid_values,
																				nan_coords, method='linear')
		
		# Fill the interpolated values back into the image
		result = image.copy()
		result[nan_mask] = interpolated_values
		
		return result

serial_number = "all_scenes_1"

with NxLib(), StereoCamera(serial_number) as camera:
		camera.get_node()[ITM_PARAMETERS].set_json(
																					json.dumps(params[ITM_PARAMETERS]), 
																					True
																				)
		
		
		first_image = None
		while True:
				camera.capture()
				camera.rectify()
				camera.compute_disparity_map()
				camera.compute_point_map()
				
				# get unique name
				name = "optonic_bin-picking_dataset_00.png"
				name_counter = 1
				while name in os.listdir("./data/rgb"):
						name = f"optonic_bin-picking_dataset_{name_counter:02}.png"
						name_counter += 1
				
				# calc depth map
				points = camera.get_point_map()
				depth_data = points[:,:,2]
				depth_data = np.minimum(depth_data, 0)
				depth_data = depth_data + abs(np.nanmin(depth_data))
				depth_data = interpolate_nan(depth_data)
				depth_data = depth_data.astype(np.int16)
				
				
				if np.nanmax(depth_data) <= 255:
						depth_data = depth_data.astype(np.uint8)
				else:
						depth_data = ((depth_data - np.nanmax(depth_data)) / 
														(np.nanmax(depth_data) - np.nanmin(depth_data))
														).astype(np.uint8)
				depth_data = (depth_data * 255).astype(np.uint8)
				
				min_val = np.min(depth_data)
				max_val = np.max(depth_data)
				
				# Apply linear scaling to stretch the pixel values
				depth_data = ((depth_data - min_val) / (max_val - min_val)) * 255
				depth_data = np.uint8(depth_data)
				
				# get rgb
				image = camera.get_texture()
				image = cv2.cvtColor(image, cv2.COLOR_BGR2RGB)
				
				if first_image is None:
						first_image = image
				elif np.array_equal(first_image, image):
						break
				
				# save images
				cv2.imwrite(f"./data/depth/{name}", depth_data)
				cv2.imwrite(f"./data/rgb/{name}", image)
	\end{lstlisting}

	
