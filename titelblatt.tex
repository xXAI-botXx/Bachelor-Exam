% -------------------------------------------------------
% In dieser Datei sollten eigentlich keine Veränderungen mehr
% notwendig sein.
% -------------------------------------------------------

\thispagestyle{empty}

% Fakultät
% -------------------------------------------------------
\ifthenelse{\equal{\hsmafakultaet}{EI}}%
  {\newcommand{\hsmafakultaetlangde}{Fakultät Elektrotechnik und Informationstechnik}%
   \newcommand{\hsmafakultaetlangen}{Department of Electrical Engineering and Computer Science}}{}
\ifthenelse{\equal{\hsmafakultaet}{EMI}}%
{\newcommand{\hsmafakultaetlangde}{Fakultät Elektrotechnik, Medizintechnik und Informatik}%
	\newcommand{\hsmafakultaetlangen}{Department of Electrical Engineering, Medical Engineering and Computer Science}}{}



\ifthenelse{\equal{\hsmastudiengang}{AI}}%
{\newcommand{\hsmastudienganglangde}{Angewandte Informatik}%
	\newcommand{\hsmastudienganglangen}{Applied Computer Science}%
	\newcommand{\hsmatypde}{BACHELORARBEIT}%
	\newcommand{\hsmatypen}{BACHELOR THESIS}%
	\newcommand{\hsmagrad}{\hsmabachelor}}{}
	
\ifthenelse{\equal{\hsmastudiengang}{AKI}}%
{\newcommand{\hsmastudienganglangde}{Angewandte Künstliche Intelligenz}%
	\newcommand{\hsmastudienganglangen}{Applied Artificial Intelligent}%
	\newcommand{\hsmatypde}{BACHELORARBEIT}%
	\newcommand{\hsmatypen}{BACHELOR THESIS}%
	\newcommand{\hsmagrad}{\hsmabachelor}}{}
	
\ifthenelse{\equal{\hsmastudiengang}{EI}}%
{\newcommand{\hsmastudienganglangde}{Elektrotechnik/Informationstechnik}%
	\newcommand{\hsmastudienganglangen}{Electrical Engineering/Information Technology}%
	\newcommand{\hsmatypde}{BACHELORARBEIT}%
	\newcommand{\hsmatypen}{BACHELOR THESIS}%
	\newcommand{\hsmagrad}{\hsmabachelor}}{}

\ifthenelse{\equal{\hsmastudiengang}{MK}}%
{\newcommand{\hsmastudienganglangde}{Mechatronik}%
	\newcommand{\hsmastudienganglangen}{Mechatronics}%
	\newcommand{\hsmatypde}{BACHELORARBEIT}%
	\newcommand{\hsmatypen}{BACHELOR THESIS}%
	\newcommand{\hsmagrad}{\hsmabachelor}}{}

\ifthenelse{\equal{\hsmastudiengang}{INFM}}%
  {\newcommand{\hsmastudienganglangde}{Informatik Master}%
  \newcommand{\hsmastudienganglangen}{Computer Science Master}%
  \newcommand{\hsmatypde}{MASTERARBEIT}%
  \newcommand{\hsmatypen}{MASTER THESIS}%
  \newcommand{\hsmagrad}{\hsmamaster}}{}

\newcommand{\hsmamaster}{Master of Science (M.Sc.)}

\newcommand{\hsmabachelor}{Bachelor of Science (B.Sc.)}


\newcommand{\hsmakoerperschaftde}{Hochschule für Technik, Wirtschaft und Medien Offenburg}
\newcommand{\hsmakoerperschaften}{Offenburg University}

\newcommand{\hsmaautorbib}{\hsmaautornname, \hsmaautorvname} % Autor Nachname, Vorname
\newcommand{\hsmaautor}{\hsmaautorvname \ \hsmaautornname} % Autor Vorname Nachname

\ifthenelse{\equal{\hsmasprache}{de}}%
  {\newcommand{\hsmatyp}{\hsmatypde}%
   \newcommand{\hsmathesistype}{zur Erlangung des akademischen Grades \hsmagrad}%
   \newcommand{\hsmakoerperschaft}{\hsmakoerperschaftde}%
   \newcommand{\hsmastudiengangname}{Studiengang \hsmastudienganglangde}%
   \newcommand{\hsmastudienganglang}{\hsmastudienganglangde}%
   \newcommand{\hsmatitel}{\hsmatitelde}%
   \newcommand{\hsmatutor}{Betreuer}%
   \newcommand{\hsmafakultaetlang}{\hsmafakultaetlangde}%
   \newcommand{\hsmalistoftables}{Tabellenverzeichnis}%
   \newcommand{\hsmalistoffigures}{Abbildungsverzeichnis}%
   \newcommand{\hsmalistings}{Quellcodeverzeichnis}%
   \newcommand{\hsmaindex}{Index}%
   \newcommand{\hsmaabbreviations}{Abkürzungsverzeichnis}%   
   \selectlanguage{ngerman}}%
  {\newcommand{\hsmatyp}{\hsmatypen}%
   \newcommand{\hsmathesistype}{for the acquisition of the academic degree \hsmagrad}%
   \newcommand{\hsmakoerperschaft}{\hsmakoerperschaften}%
   \newcommand{\hsmastudiengangname}{Course of Studies: \hsmastudienganglang}%
   \newcommand{\hsmastudienganglang}{\hsmastudienganglangen}%
   \newcommand{\hsmatitel}{\hsmatitelencleaned}%
   \newcommand{\hsmatutor}{Tutors}
   \newcommand{\hsmafakultaetlang}{\hsmafakultaetlangen}%
   \newcommand{\hsmalistoftables}{List of Tables}%
   \newcommand{\hsmalistoffigures}{List of Figures}%
   \newcommand{\hsmalistings}{Listings}%
   \newcommand{\hsmaindex}{Index}%
   \newcommand{\hsmaabbreviations}{List of Abbreviations}%
   \selectlanguage{english}}%


% Daten in die Standard-Felder von KOMA-Script eintragen
\titlehead{\hsmatyp\ in\  \hsmastudienganglang}
\subject{}
\title{\hsmatitel}
\author{\hsmaauthor}
\date{\small{\hsmadatum}}

% Daten für das fertige PDF-Dokument
\hypersetup{
  pdftitle={\hsmatitel},  % Titel des Dokuments
  pdfauthor={\hsmaautor},              % Autor
  pdfsubject={\hsmatyp\ in\ \hsmastudienganglang},                % Thema
  pdfkeywords={\hsmatitel}         % Schlüsselworte
}

\newlength{\bindekorrektur}
\newlength{\seitenanfang}
\newlength{\seitenbreite}
  
\setlength{\bindekorrektur}{-46mm}   % Korrektur der horizontalen Position
\setlength{\seitenanfang}{0mm}       % Korrektur der vertikalen Position
\setlength{\seitenbreite}{297mm}

%\noindent \includegraphics[width=7cm, left]{hso.png}\hfill \includegraphics[width=2cm, right]{edeka.png} \\
\captionsetup[figure]{labelformat=empty}
\noindent 
\begin{figure}
	%\includegraphics[width=10cm,center]{hso.jpg}
% Wenn ein Unternehmenslogo mit abgedruckt werden soll,
% kann dies wie folgt integriert werden.	
	\begin{subfigure}[b]{0.5\textwidth}
		\includegraphics[width=7cm,left]{hso.jpg}
	\end{subfigure} 
	\begin{subfigure}[b]{0.5\textwidth}
		\centering
		\includegraphics[width=4cm,right]{optonic.png}
	\end{subfigure} 
	\caption[]{}
\end{figure}
\captionsetup[figure]{labelformat=simple}
% Titel der Arbeit
\begin{textblock*}{128mm}(41mm,\seitenanfang + 62mm) % 4,5cm vom linken Rand und 6,0cm vom oberen Rand
  \centering\Large\sffamily
  \vspace{12mm} % Kleiner zusätzlicher Abstand oben für bessere Optik
  \textbf{\hsmatitel}
\end{textblock*}%

% Name
\begin{textblock*}{\seitenbreite}(\bindekorrektur,\seitenanfang + 108mm)
  \centering\large\sffamily
  \hsmaautor
\end{textblock*}

% Thesis
\begin{textblock*}{\seitenbreite}(\bindekorrektur,\seitenanfang + 130mm)
  \centering\large\sffamily
  \textbf{\hsmatyp}\\
  \begin{small}\hsmathesistype \end{small}\\
  \vspace{6mm}
  \hsmastudiengangname
\end{textblock*}

% Fakultät
\begin{textblock*}{\seitenbreite}(\bindekorrektur,\seitenanfang + 165mm)
  \centering\large\sffamily
  \hsmafakultaetlang\\
  \vspace{2mm}
  \hsmakoerperschaft
\end{textblock*}

% Datum
\begin{textblock*}{\seitenbreite}(\bindekorrektur,\seitenanfang + 190mm)
  \centering\large 
  \textsf{\hsmadatum}
\end{textblock*}

% Firma
\begin{textblock*}{\seitenbreite}(\bindekorrektur,\seitenanfang + 215mm)
  \centering\large 
  % \textsf{Durchgeführt bei der Firma \hsmafirma}
  \textsf{Performed at the company \hsmafirma}
\end{textblock*}

% Betreuer
\begin{textblock*}{\seitenbreite}(\bindekorrektur,\seitenanfang + 240mm)
  \centering\large\sffamily
  \hsmatutor \\
  \vspace{2mm}
  \hsmabetreuer\\
  \vspace{2mm}
  \hsmazweitkorrektor
\end{textblock*}

% Bibliographische Informationen
\null\newpage
\thispagestyle{empty}
  
\newcommand{\hsmabibde}{\begin{small}\textbf{\hsmaautorbib}: \\ \hsmatitelde \ / \hsmaautor. \ -- \\ \hsmatypde, \hsmaort : \hsmakoerperschaftde, \hsmajahr. \pageref{lastpage} Seiten.\end{small}}

\newcommand{\hsmabiben}{\begin{small}\textbf{\hsmaautorbib}: \\ \hsmatitelen \ / \hsmaautor. \ -- \\ \hsmatypen, \hsmaort : \hsmakoerperschaften, \hsmajahr. \pageref{lastpage} pages. \end{small}}

\ifthenelse{\equal{\hsmasprache}{de}}%
  {\hsmabibde \\ \vspace{0.5cm} \\ \hsmabiben}
  {\hsmabiben \\ \vspace{0.5cm} \\ \hsmabibde}


%Vorwort
\clearpage\setcounter{page}{1}
\thispagestyle{empty}
\textsf{\large\textbf{Acknowledgment}}

This work would not have been possible without the guidance of Optonic GmbH, especially Moritz Sperling and my university supervisor, Prof. Dr.-Ing. Janis Keuper.\\
The journey was challenging; I faced many difficulties. Including technological complications and personal health issues. \\
Although I often tried to solve problems independently, seeking help sooner eased some challenges. In the end, I could not be happier to have accepted these challenges and navigated through them with the expertise and dedication of my supervisors.


% Erklärung
\clearpage
\thispagestyle{empty}
\textsf{\large\textbf{Eidesstattliche Erklärung}}

Hiermit versichere ich eidesstattlich, dass die vorliegende Master-Thesis von mir selbststän-dig und ohne unerlaubte fremde Hilfe angefertigt worden ist, insbesondere, dass ich alle Stel-len, die wörtlich oder annähernd wörtlich oder dem Gedanken nach aus Veröffentlichungen, unveröffentlichten Unterlagen und Gesprächen entnommen worden sind, als solche an den entsprechenden Stellen innerhalb der Arbeit durch Zitate kenntlich gemacht habe, wobei in den Zitaten jeweils der Umfang der entnommenen Originalzitate kenntlich gemacht wurde. Ich bin mir bewusst, dass eine falsche Versicherung rechtliche Folgen haben wird.

Ich bin damit einverstanden, dass meine Arbeit veröffentlicht wird, d.\,h. dass die Arbeit elektronisch gespeichert, in andere Formate konvertiert, auf den Servern der Hochschule Offenburg öffentlich zugänglich gemacht und über das Internet verbreitet werden darf.

%\textsf{\large\textbf{Declaration on oath}}

%I hereby declare on oath that this Bachelor's thesis has been prepared by me independently and without unauthorized external assistance, in particular, that I have identified all passages taken verbatim or approximately verbatim or in spirit from publications, unpublished documents, and conversations as such at the appropriate places within the thesis utilizing quotations, whereby the scope of the original quotations taken has been indicated in the quotations. I am aware that a false statement will have legal consequences.

\vspace{1cm}
\hsmaort, \hsmadatum \\
\hsmaautor

\vspace{2.5cm}

\textsf{\large\textbf{AI-Tool Disclaimer}}

%Nowadays AI-Tools are everywhere and it is important to state what is human-made and what not. In reality, the work of AI and humans is woven together, and therefore, it is even more crucial to name and try to differentiate.\\
% This work was written by a human for humans. 
% The AI-Tool Grammarly \cite{Grammarly} was applied to correct grammar and spelling, for appealing language and best output for the readers. The text itself was not the creation of an AI nor AI assisted.\\
% ChatGPT \cite{ChatGPT} was used for supporting during coding to create and debug tasks efficiently. It still remains the handcraft and thinking of the author.

%I hereby affirm that the AI tool Grammarly \cite{Grammarly} was used solely to correct grammar, spelling, and to enhance the language for optimal clarity and appeal to the reader. The text itself, however, was not generated by AI, nor was it AI-assisted in its creation. Furthermore, I testify that ChatGPT \cite{ChatGPT} was employed as a supportive tool during the coding process, assisting with the creation and debugging of tasks in an efficient manner. Nevertheless, the work reflects the independent craftsmanship and reasoning of the author.
Ich versichere hiermit, dass das KI-Tool Grammarly \cite{Grammarly} ausschließlich dazu verwendet wurde, Grammatik und Rechtschreibung zu korrigieren und die Sprache zu verbessern, um eine optimale Klarheit und Attraktivität für den Leser zu erreichen. Der Text selbst wurde jedoch weder von der KI erstellt. Darüber hinaus bezeuge ich, dass ChatGPT \cite{ChatGPT} während des Programmierens als unterstützendes Werkzeug eingesetzt wurde, das die Erstellung und das Debugging von Aufgaben auf effiziente Weise unterstützte. Nichtsdestotrotz spiegelt die Arbeit das unabhängige Handwerk und die Überlegungen des Autors wider.

\ifthenelse{\boolean{hsmapublizieren} \and \not\boolean{hsmasperrvermerk}}%
{
\vspace{0.5cm}
%I agree that my work may be published, i.e. that the work may be stored electronically, converted into other formats, made publicly accessible on the servers of Offenburg University of Applied Sciences and distributed via the Internet.  
}{}%


\vspace{1cm}
\hsmaort, \hsmadatum \\
%\vspace{1.2cm}						                                      
\hsmaautor

\ifthenelse{\boolean{hsmasperrvermerk}}%
{%
\vspace{5cm}
\color{red}\textsf{\large\textbf{Sperrvermerk}}

Die vorliegende Abschlussarbeit beinhaltet vertrauliche Informationen und interne Daten des Unternehmens \hsmafirma.
Sie darf aus diesem Grund nur zu Prüfzwecken verwendet und ohne ausdrückliche Genehmigung durch die \hsmafirma weder Dritten zugänglich gemacht, noch ganz oder in Auszügen veröffentlicht werden. Die Sperrfrist endet 5 Jahre Jahre nach dem Einreichen der Arbeit bei der Hochschule Offenburg. Unbeschadet hiervon bleibt die Weitergabe der Arbeit und Einsicht in die Arbeit an die mit der Prüfung befassten Mitarbeiter der Hochschule und Prüfer möglich, die ihrerseits zur Geheimhaltung verpflichtet sind, sowie die Verwendung der Arbeit in eventuellen prüfungsrechtlichen Rechtsschutzverfahren nach Maßgabe der geltenden verwaltungsprozessualen Regeln.
\color{black}
}{}

\cleardoublepage

% Abstract
\thispagestyle{empty}
\textsf{\large\textbf{Zusammenfassung}}
\subsubsection*{\hsmatitelde}%\hsmaabstractde
In dieser Studie werden neun synthetische RGBD-Instanzsegmentierungsdatensätze für das Training und zusätzliche Datensätze für das Testen vorgestellt, wobei ein besonderer Schwerpunkt auf der Variation der Anzahl von Formen und Texturen liegt. Zusätzlich wird ein realer Bin-Picking-Datensatz mit unübersichtlichen und übersichtlichen Industrieteilen und Alltagsprodukten vorgestellt. Diese Datensätze und der bekannte OCID-Datensatz werden verwendet, um die Auswirkung von Tiefeninformationen und der Anzahl der Formen und Texturen auf den "Shape-Texture-Bias", die Leistung und die Generalisierung bei der Instanzsegmentierung zu untersuchen.\\
Die Ergebnisse zeigen, dass reine RGB-Modelle einen Bias in Richtung der Textur besitzen, während RGB-D-Modelle eine leichte Verschiebung in Richtung Shape-Bias aufweisen, obwohl der Textur Bias bestehen bleibt. Modelle, die RGB und Tiefe verwenden, schnitten etwas besser ab als mit nur RGB Daten, und ebenso auf einem der beiden realen Datensatz war das Segmentierungsergebnis deutlich besser mit der Tiefe als zusätzlichen Input. Allerdings könnte eine niedrige Qualität der Tiefeninformationen, eventuell die Schärfe, zu verzerrten Ergebnissen führen, die noch schlechter sind als bei reinen RGB-Modellen. Darüber hinaus zeigt diese Studie, dass reine RGB-Modelle massiv von der Qualität (Auflösung) der Eingabebilder beeinflusst werden.\\
Die Erhöhung der Anzahl der einzigartigen Formen in den Trainingsdaten verringert den Shape-Bias, erhöht die allgemeine Instanzsegmentierungsleistung, steigend mit der Anzahl der unbekannten Formen in den Daten und hat einen entscheidenden Einfluss auf die Generalisierungs von Formen. \\
Die Erhöhung der Anzahl der Texturen hatte keine signifikante Auswirkung auf die Shape-Texture-Bias, aber sie zeigt, dass sie auch die allgemeine Leistung mit der steigenden Anzahl unbekannter Texturen erhöht. Diese Studie legt nahe, dass eine höhere Anzahl an Texturen in den Trainingsdaten zu einer höheren Generalisierung von Texturen führt.

\clearpage
\thispagestyle{empty}
\textsf{\large\textbf{Abstract}}
\subsubsection*{\hsmatitelen}%\hsmaabstracten

This study presents nine synthetic RGBD instance segmentation datasets for training and additional datasets for testing with a special focus on varying the number of shapes and textures. Additionally, a real-world bin-picking dataset with cluttered and uncluttered industrial parts and everyday products is proposed. These datasets and the known OCID dataset are used to investigate the impact of depth information and shape-texture amount towards shape-texture bias, performance, and generalization in instance segmentation.\\
The findings reveal that RGB-only models are biased toward texture, while RGB-D models exhibit a slight shift towards shape bias, though texture bias persists. Models using RGB and depth performed slightly better, and one real-world dataset was significantly better. However, the depth information quality, eventually sharpness, could lead to distorted results, even worse than RGB-only. In addition, this study finds that RGB-only models are influenced massively by the quality (resolution) of the input images.\\
Increasing the number of unique shapes in train data decreases the shape bias, increases the general performance, rises with the number of unknown shapes in the data, and has a crucial impact on the generalization of shapes. \\
Increasing the number of textures had no significant effect on the shape-texture bias, but it shows that it also increases the general performance with the rising number of unknown textures. This study suggests that a higher texture amount in train data leads to higher generalization of textures.




